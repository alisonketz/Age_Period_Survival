\documentclass[11pt,]{article}
\usepackage{lmodern}
\usepackage{amssymb,amsmath}
\usepackage{ifxetex,ifluatex}
\usepackage{fixltx2e} % provides \textsubscript
\ifnum 0\ifxetex 1\fi\ifluatex 1\fi=0 % if pdftex
  \usepackage[T1]{fontenc}
  \usepackage[utf8]{inputenc}
\else % if luatex or xelatex
  \ifxetex
    \usepackage{mathspec}
  \else
    \usepackage{fontspec}
  \fi
  \defaultfontfeatures{Ligatures=TeX,Scale=MatchLowercase}
\fi
% use upquote if available, for straight quotes in verbatim environments
\IfFileExists{upquote.sty}{\usepackage{upquote}}{}
% use microtype if available
\IfFileExists{microtype.sty}{%
\usepackage[]{microtype}
\UseMicrotypeSet[protrusion]{basicmath} % disable protrusion for tt fonts
}{}
\PassOptionsToPackage{hyphens}{url} % url is loaded by hyperref
\usepackage[unicode=true]{hyperref}
\PassOptionsToPackage{usenames,dvipsnames}{color} % color is loaded by hyperref
\hypersetup{
            pdftitle={Appendix S4},
            pdfauthor={Alison C. Ketz},
            colorlinks=true,
            linkcolor=Maroon,
            citecolor=Blue,
            urlcolor=Blue,
            breaklinks=true}
\urlstyle{same}  % don't use monospace font for urls
\usepackage[margin=1in]{geometry}
\usepackage{color}
\usepackage{fancyvrb}
\newcommand{\VerbBar}{|}
\newcommand{\VERB}{\Verb[commandchars=\\\{\}]}
\DefineVerbatimEnvironment{Highlighting}{Verbatim}{commandchars=\\\{\}}
% Add ',fontsize=\small' for more characters per line
\usepackage{framed}
\definecolor{shadecolor}{RGB}{248,248,248}
\newenvironment{Shaded}{\begin{snugshade}}{\end{snugshade}}
\newcommand{\KeywordTok}[1]{\textcolor[rgb]{0.13,0.29,0.53}{\textbf{#1}}}
\newcommand{\DataTypeTok}[1]{\textcolor[rgb]{0.13,0.29,0.53}{#1}}
\newcommand{\DecValTok}[1]{\textcolor[rgb]{0.00,0.00,0.81}{#1}}
\newcommand{\BaseNTok}[1]{\textcolor[rgb]{0.00,0.00,0.81}{#1}}
\newcommand{\FloatTok}[1]{\textcolor[rgb]{0.00,0.00,0.81}{#1}}
\newcommand{\ConstantTok}[1]{\textcolor[rgb]{0.00,0.00,0.00}{#1}}
\newcommand{\CharTok}[1]{\textcolor[rgb]{0.31,0.60,0.02}{#1}}
\newcommand{\SpecialCharTok}[1]{\textcolor[rgb]{0.00,0.00,0.00}{#1}}
\newcommand{\StringTok}[1]{\textcolor[rgb]{0.31,0.60,0.02}{#1}}
\newcommand{\VerbatimStringTok}[1]{\textcolor[rgb]{0.31,0.60,0.02}{#1}}
\newcommand{\SpecialStringTok}[1]{\textcolor[rgb]{0.31,0.60,0.02}{#1}}
\newcommand{\ImportTok}[1]{#1}
\newcommand{\CommentTok}[1]{\textcolor[rgb]{0.56,0.35,0.01}{\textit{#1}}}
\newcommand{\DocumentationTok}[1]{\textcolor[rgb]{0.56,0.35,0.01}{\textbf{\textit{#1}}}}
\newcommand{\AnnotationTok}[1]{\textcolor[rgb]{0.56,0.35,0.01}{\textbf{\textit{#1}}}}
\newcommand{\CommentVarTok}[1]{\textcolor[rgb]{0.56,0.35,0.01}{\textbf{\textit{#1}}}}
\newcommand{\OtherTok}[1]{\textcolor[rgb]{0.56,0.35,0.01}{#1}}
\newcommand{\FunctionTok}[1]{\textcolor[rgb]{0.00,0.00,0.00}{#1}}
\newcommand{\VariableTok}[1]{\textcolor[rgb]{0.00,0.00,0.00}{#1}}
\newcommand{\ControlFlowTok}[1]{\textcolor[rgb]{0.13,0.29,0.53}{\textbf{#1}}}
\newcommand{\OperatorTok}[1]{\textcolor[rgb]{0.81,0.36,0.00}{\textbf{#1}}}
\newcommand{\BuiltInTok}[1]{#1}
\newcommand{\ExtensionTok}[1]{#1}
\newcommand{\PreprocessorTok}[1]{\textcolor[rgb]{0.56,0.35,0.01}{\textit{#1}}}
\newcommand{\AttributeTok}[1]{\textcolor[rgb]{0.77,0.63,0.00}{#1}}
\newcommand{\RegionMarkerTok}[1]{#1}
\newcommand{\InformationTok}[1]{\textcolor[rgb]{0.56,0.35,0.01}{\textbf{\textit{#1}}}}
\newcommand{\WarningTok}[1]{\textcolor[rgb]{0.56,0.35,0.01}{\textbf{\textit{#1}}}}
\newcommand{\AlertTok}[1]{\textcolor[rgb]{0.94,0.16,0.16}{#1}}
\newcommand{\ErrorTok}[1]{\textcolor[rgb]{0.64,0.00,0.00}{\textbf{#1}}}
\newcommand{\NormalTok}[1]{#1}
\usepackage{graphicx,grffile}
\makeatletter
\def\maxwidth{\ifdim\Gin@nat@width>\linewidth\linewidth\else\Gin@nat@width\fi}
\def\maxheight{\ifdim\Gin@nat@height>\textheight\textheight\else\Gin@nat@height\fi}
\makeatother
% Scale images if necessary, so that they will not overflow the page
% margins by default, and it is still possible to overwrite the defaults
% using explicit options in \includegraphics[width, height, ...]{}
\setkeys{Gin}{width=\maxwidth,height=\maxheight,keepaspectratio}
\IfFileExists{parskip.sty}{%
\usepackage{parskip}
}{% else
\setlength{\parindent}{0pt}
\setlength{\parskip}{6pt plus 2pt minus 1pt}
}
\setlength{\emergencystretch}{3em}  % prevent overfull lines
\providecommand{\tightlist}{%
  \setlength{\itemsep}{0pt}\setlength{\parskip}{0pt}}
\setcounter{secnumdepth}{5}
% Redefines (sub)paragraphs to behave more like sections
\ifx\paragraph\undefined\else
\let\oldparagraph\paragraph
\renewcommand{\paragraph}[1]{\oldparagraph{#1}\mbox{}}
\fi
\ifx\subparagraph\undefined\else
\let\oldsubparagraph\subparagraph
\renewcommand{\subparagraph}[1]{\oldsubparagraph{#1}\mbox{}}
\fi

% set default figure placement to htbp
\makeatletter
\def\fps@figure{htbp}
\makeatother


\title{Appendix S4}
\author{Alison C. Ketz}
\date{11/10/2021}

\begin{document}
\maketitle

\section{Simulation}\label{simulation}

This appendix provides R code for generating the age-period survival
data with two example hazard functions that were used in the manuscript
for data generation. Each step in the data generating function will be
described and then the full executable function will be provided. Stand
alone executable R files for the data generating function are also
provided. Then we describe how to format the data for fitting all of
these models. The model fitting procedure for a single model is
described in detail. We have provided code for each of the 10 models
that were fit using NIMBLE, where each model has four R scripts
including: 1) A script to execute and run all the additional Rscripts
for each simulation as well as for providing necessary libraries. 2) A
preliminary script that runs the data generating and formatting scripts,
computes the necessary basis functions, and calculates constants needed
for model fitting. 3) A script for running the models in nimble with
additional nimble functions used for intermediary calculations. 4) An R
script for doing model checking and plotting of results. These four R
scripts for each model are provided for both hazard function simulations
- the R-select or K-select simulation.

\subsection{Kselect Data Generation}\label{kselect-data-generation}

First we specify necessary constants, including the intercept for the
log hazard, i.e.~the baseline hazard rate, the number of individuals
that are ``collared'', and the maximum number of age intervals and
period intervals.

\begin{Shaded}
\begin{Highlighting}[]
  \CommentTok{# sample size of individuals from each year}
\NormalTok{  n_yr1 <-}\StringTok{ }\DecValTok{1000}
\NormalTok{  n_yr2 <-}\StringTok{ }\DecValTok{1000}
\NormalTok{  n <-}\StringTok{ }\NormalTok{n_yr1 }\OperatorTok{+}\StringTok{ }\NormalTok{n_yr2}
  
  \CommentTok{# intercept}
\NormalTok{  beta0 <-}\StringTok{ }\OperatorTok{-}\DecValTok{5}

\NormalTok{  ### maximum age interval}
\NormalTok{  nT_age <-}\StringTok{ }\DecValTok{400}

\NormalTok{  ### maximum period interval}
\NormalTok{  nT_period <-}\StringTok{ }\DecValTok{104}
\end{Highlighting}
\end{Shaded}

We generate the ages of individuals upon entry (left\_age) and we must
generate the period of entry (left\_period). We specify individuals
entering the population assuming a stable age distribution, which means
that the left age entry is generated using the same baseline hazard over
the ages as the age effect for the generating hazard function. We
specified staggered entry of equal probability over 8 interval periods
for the period of entry. We also specified staggered entry such that the
approximate entry of collared individuals are split between different
``years'', if we consider that the intervals used for period effects
reflect 2 years of a study, despite the fact that these data are
simulated and the intervals could ultimately be reflecting any study
period of time. The maximum age that an individual in the study can live
is the difference between the maximum period and the entry period
(maxtimes).

\begin{Shaded}
\begin{Highlighting}[]
\NormalTok{  #####################################################}
\NormalTok{  ### Age at entry with staggered entry,}
\NormalTok{  ### drawn from a stable age distribution}
\NormalTok{  ### based on the age effect hazard}
\NormalTok{  ### for 2 'years', and for the age that}
\NormalTok{  ### individuals that go from birth to censoring}
\NormalTok{  #####################################################}

  \CommentTok{#Age effects on log hazard}
  \CommentTok{#baseline is a quadratic function}
\NormalTok{  age <-}\StringTok{ }\KeywordTok{seq}\NormalTok{(}\DecValTok{1}\NormalTok{, nT_age }\OperatorTok{-}\StringTok{ }\DecValTok{1}\NormalTok{, }\DataTypeTok{by =} \DecValTok{1}\NormalTok{) }
\NormalTok{  age_effect <-}\StringTok{ }\FloatTok{5e-05} \OperatorTok{*}\StringTok{ }\NormalTok{age}\OperatorTok{^}\DecValTok{2} \OperatorTok{-}\StringTok{ }\FloatTok{0.02} \OperatorTok{*}\StringTok{ }\NormalTok{age }\OperatorTok{+}\StringTok{ }\DecValTok{1} 

  \CommentTok{#Inducing wiggles during the early age intervals}
\NormalTok{  anthro <-}\StringTok{ }\NormalTok{.}\DecValTok{5} \OperatorTok{*}\StringTok{ }\KeywordTok{cos}\NormalTok{(}\DecValTok{1} \OperatorTok{/}\StringTok{ }\DecValTok{26} \OperatorTok{*}\StringTok{ }\NormalTok{pi }\OperatorTok{*}\StringTok{ }\NormalTok{age)}
\NormalTok{  anthro[}\DecValTok{1}\OperatorTok{:}\DecValTok{19}\NormalTok{] <-}\StringTok{ }\DecValTok{0}
\NormalTok{  anthro[}\DecValTok{20}\OperatorTok{:}\DecValTok{38}\NormalTok{] <-}\StringTok{ }\NormalTok{anthro[}\DecValTok{20}\OperatorTok{:}\DecValTok{38}\NormalTok{] }\OperatorTok{*}\StringTok{ }\KeywordTok{seq}\NormalTok{(}\DecValTok{0}\NormalTok{, }\DecValTok{1}\NormalTok{, }\DataTypeTok{length =} \KeywordTok{length}\NormalTok{(}\DecValTok{20}\OperatorTok{:}\DecValTok{38}\NormalTok{))}
\NormalTok{  anthro[}\DecValTok{50}\OperatorTok{:}\DecValTok{170}\NormalTok{] <-}\StringTok{ }\NormalTok{anthro[}\DecValTok{50}\OperatorTok{:}\DecValTok{170}\NormalTok{] }\OperatorTok{*}\StringTok{ }\KeywordTok{seq}\NormalTok{(}\DecValTok{1}\NormalTok{, }\DecValTok{0}\NormalTok{, }\DataTypeTok{length =} \KeywordTok{length}\NormalTok{(}\DecValTok{50}\OperatorTok{:}\DecValTok{170}\NormalTok{))}
\NormalTok{  anthro[}\DecValTok{171}\OperatorTok{:}\NormalTok{(nT_age }\OperatorTok{-}\StringTok{ }\DecValTok{1}\NormalTok{)] <-}\StringTok{ }\DecValTok{0} 
\NormalTok{  age_effect <-}\StringTok{ }\NormalTok{age_effect }\OperatorTok{+}\StringTok{ }\NormalTok{anthro}
\NormalTok{  age_effect <-}\StringTok{ }\NormalTok{age_effect }\OperatorTok{-}\StringTok{ }\KeywordTok{mean}\NormalTok{(age_effect)}

\NormalTok{  hazard_se <-}\StringTok{ }\OperatorTok{-}\KeywordTok{exp}\NormalTok{((beta0 }\OperatorTok{+}\StringTok{ }\NormalTok{age_effect))}
\NormalTok{  stat_se <-}\StringTok{ }\KeywordTok{rep}\NormalTok{(}\OtherTok{NA}\NormalTok{, nT_age }\OperatorTok{-}\StringTok{ }\NormalTok{nT_period)}
\NormalTok{  stat_se[}\DecValTok{1}\NormalTok{] <-}\StringTok{ }\NormalTok{(}\DecValTok{1} \OperatorTok{-}\StringTok{ }\KeywordTok{exp}\NormalTok{(hazard_se[}\DecValTok{1}\NormalTok{]))}
  \ControlFlowTok{for}\NormalTok{ (j }\ControlFlowTok{in} \DecValTok{2}\OperatorTok{:}\NormalTok{(nT_age }\OperatorTok{-}\StringTok{ }\NormalTok{nT_period)) \{}
\NormalTok{    stat_se[j] <-}\StringTok{ }\NormalTok{(}\DecValTok{1} \OperatorTok{-}\StringTok{ }\KeywordTok{exp}\NormalTok{(hazard_se[j])) }\OperatorTok{*}\StringTok{ }\KeywordTok{exp}\NormalTok{(}\KeywordTok{sum}\NormalTok{(hazard_se[}\DecValTok{1}\OperatorTok{:}\NormalTok{(j }\OperatorTok{-}\StringTok{ }\DecValTok{1}\NormalTok{)]))}
\NormalTok{  \}}
\NormalTok{  left_age <-}\StringTok{ }\NormalTok{nimble}\OperatorTok{::}\KeywordTok{rcat}\NormalTok{(n, stat_se)}

\NormalTok{  ###########################################}
  \CommentTok{# Staggered entry period effects}
\NormalTok{  ###########################################}

\NormalTok{  weeks_entry <-}\StringTok{ }\DecValTok{8}
\NormalTok{  left_yr1 <-}\StringTok{ }\NormalTok{nimble}\OperatorTok{::}\KeywordTok{rcat}\NormalTok{(n_yr1, }\DataTypeTok{prob =} \KeywordTok{rep}\NormalTok{(}\DecValTok{1} \OperatorTok{/}\StringTok{ }\NormalTok{weeks_entry, weeks_entry))}
\NormalTok{  left_yr2 <-}\StringTok{ }\NormalTok{nimble}\OperatorTok{::}\KeywordTok{rcat}\NormalTok{(n_yr2, }\DataTypeTok{prob =} \KeywordTok{rep}\NormalTok{(}\DecValTok{1} \OperatorTok{/}\StringTok{ }\NormalTok{weeks_entry, weeks_entry))}
\NormalTok{  left_period <-}\StringTok{ }\KeywordTok{c}\NormalTok{(left_yr1, left_yr2)}

  \CommentTok{# no staggered entry}
\NormalTok{  maxtimes <-}\StringTok{ }\NormalTok{nT_period }\OperatorTok{-}\StringTok{ }\NormalTok{left_period}
\end{Highlighting}
\end{Shaded}

Then we must specify the hazard functions. These could be any functional
(mathematical) form. We have tried to use reasonable variants that would
reflect hazards for our study systems. For the age effects for the
K-selected species, we used a quadratic shape curve with extra wiggles
that are introduced by the additive functional shape (anthro). We
allowed the additive effect to gradually start and end over the first
half of the ages using the decay sequence. Then we specified the period
effects using a periodic curve.

\begin{Shaded}
\begin{Highlighting}[]
\NormalTok{  ########################################}
  \CommentTok{# Age effects for the log hazard}
\NormalTok{  ########################################}

\NormalTok{  age <-}\StringTok{ }\KeywordTok{seq}\NormalTok{(}\DecValTok{1}\NormalTok{, nT_age }\OperatorTok{-}\StringTok{ }\DecValTok{1}\NormalTok{, }\DataTypeTok{by =} \DecValTok{1}\NormalTok{)}
\NormalTok{  age_effect <-}\StringTok{ }\FloatTok{5e-05} \OperatorTok{*}\StringTok{ }\NormalTok{age}\OperatorTok{^}\DecValTok{2} \OperatorTok{-}\StringTok{ }\FloatTok{0.02} \OperatorTok{*}\StringTok{ }\NormalTok{age }\OperatorTok{+}\StringTok{ }\DecValTok{1}
\NormalTok{  age_effect <-}\StringTok{ }\NormalTok{age_effect }\OperatorTok{+}\StringTok{ }\NormalTok{anthro}
\NormalTok{  age_effect <-}\StringTok{ }\NormalTok{age_effect }\OperatorTok{-}\StringTok{ }\KeywordTok{mean}\NormalTok{(age_effect)}
 
\NormalTok{  ########################################}
\NormalTok{  ### Period effects for the log hazard }
\NormalTok{  ########################################}

\NormalTok{  period <-}\StringTok{ }\KeywordTok{seq}\NormalTok{(}\DecValTok{1}\NormalTok{, nT_period }\OperatorTok{-}\StringTok{ }\DecValTok{1}\NormalTok{, }\DataTypeTok{by =} \DecValTok{1}\NormalTok{)}
\NormalTok{  period_effect <-}\StringTok{ }\DecValTok{1} \OperatorTok{*}\StringTok{ }\KeywordTok{sin}\NormalTok{(}\DecValTok{2}\OperatorTok{/}\DecValTok{52} \OperatorTok{*}\StringTok{ }\NormalTok{pi }\OperatorTok{*}\StringTok{ }\NormalTok{(period) }\OperatorTok{+}\StringTok{ }\DecValTok{1}\NormalTok{)}
\NormalTok{  period_effect <-}\StringTok{ }\NormalTok{period_effect }\OperatorTok{-}\StringTok{ }\KeywordTok{mean}\NormalTok{(period_effect)}
\end{Highlighting}
\end{Shaded}

We calculate the log hazard by the assuming the age effects and period
effects are additive. Therefore, we add the age effects and period
effects to the intercept baseline hazard from the age at entry to the
maximum possible age, and across the period effects from left entry to
the maximum possible period for each individual.

\begin{Shaded}
\begin{Highlighting}[]
\NormalTok{  ########################################}
\NormalTok{  ### Log hazard}
\NormalTok{  ########################################}
\NormalTok{  hazard <-}\StringTok{ }\KeywordTok{matrix}\NormalTok{(}\OtherTok{NA}\NormalTok{, n, nT_age)}
  \ControlFlowTok{for}\NormalTok{ (i }\ControlFlowTok{in} \DecValTok{1}\OperatorTok{:}\NormalTok{n) \{}
\NormalTok{    hazard[i,] <-}\StringTok{ }\KeywordTok{c}\NormalTok{(}\KeywordTok{rep}\NormalTok{(}\DecValTok{0}\NormalTok{,left_age[i] }\OperatorTok{-}\StringTok{ }\DecValTok{1}\NormalTok{), beta0}\OperatorTok{+}
\StringTok{                      }\NormalTok{age_effect[left_age[i]}\OperatorTok{:}\NormalTok{(left_age[i] }\OperatorTok{+}\StringTok{ }\NormalTok{maxtimes[i] }\OperatorTok{-}\StringTok{ }\DecValTok{1}\NormalTok{)] }\OperatorTok{+}
\StringTok{                      }\NormalTok{period_effect[left_period[i]}\OperatorTok{:}\NormalTok{(left_period[i] }\OperatorTok{+}\StringTok{ }\NormalTok{maxtimes[i] }\OperatorTok{-}\StringTok{ }\DecValTok{1}\NormalTok{)],}
                      \KeywordTok{rep}\NormalTok{(}\DecValTok{0}\NormalTok{, nT_age }\OperatorTok{-}\StringTok{ }\NormalTok{(left_age[i] }\OperatorTok{-}\StringTok{ }\DecValTok{1} \OperatorTok{+}\StringTok{ }\NormalTok{maxtimes[i])))}
\NormalTok{  \}}
\end{Highlighting}
\end{Shaded}

Then we must calculate the probability of mortality in each interval
based on the complementary log-log link function. The first period that
an individual enters the study has a probability of mortality during
that period that is a straightforward exponential link transformation.
The subsequent periods depend on the cumulative probability of surviving
the previous intervals.

\begin{Shaded}
\begin{Highlighting}[]
\NormalTok{  ########################################}
\NormalTok{  ### Probability of mortality}
\NormalTok{  ########################################}

\NormalTok{  test_stat <-}\StringTok{ }\KeywordTok{matrix}\NormalTok{(}\DecValTok{0}\NormalTok{, n, nT_age)}
  \ControlFlowTok{for}\NormalTok{ (i }\ControlFlowTok{in} \DecValTok{1}\OperatorTok{:}\NormalTok{n) \{}
    \ControlFlowTok{for}\NormalTok{ (j }\ControlFlowTok{in}\NormalTok{ left_age[i]}\OperatorTok{:}\NormalTok{(left_age[i] }\OperatorTok{+}\StringTok{ }\NormalTok{maxtimes[i] }\OperatorTok{-}\StringTok{ }\DecValTok{1}\NormalTok{)) \{}
      \ControlFlowTok{if}\NormalTok{ (j}\OperatorTok{==}\NormalTok{left_age[i]) \{}
\NormalTok{        test_stat[i, j] <-}\StringTok{ }\NormalTok{(}\DecValTok{1} \OperatorTok{-}\StringTok{ }\KeywordTok{exp}\NormalTok{(}\OperatorTok{-}\KeywordTok{exp}\NormalTok{(hazard[i, j])))}
\NormalTok{      \} }\ControlFlowTok{else}\NormalTok{ \{}
\NormalTok{        test_stat[i, j] <-}
\StringTok{          }\NormalTok{(}\DecValTok{1} \OperatorTok{-}\StringTok{ }\KeywordTok{exp}\NormalTok{(}\OperatorTok{-}\KeywordTok{exp}\NormalTok{(hazard[i, j])))}\OperatorTok{*}\KeywordTok{exp}\NormalTok{(}\OperatorTok{-}\KeywordTok{sum}\NormalTok{(}\KeywordTok{exp}\NormalTok{(hazard[i, left_age[i]}\OperatorTok{:}\NormalTok{(j }\OperatorTok{-}\StringTok{ }\DecValTok{1}\NormalTok{)])))}
\NormalTok{      \}}
\NormalTok{    \}}
\NormalTok{    test_stat[i, left_age[i] }\OperatorTok{+}\StringTok{ }\NormalTok{maxtimes[i]] <-}
\StringTok{          }\KeywordTok{exp}\NormalTok{(}\OperatorTok{-}\KeywordTok{sum}\NormalTok{(}\KeywordTok{exp}\NormalTok{(hazard[i, left_age[i]}\OperatorTok{:}\NormalTok{(left_age[i] }\OperatorTok{+}\StringTok{ }\NormalTok{maxtimes[i] }\OperatorTok{-}\StringTok{ }\DecValTok{1}\NormalTok{)])))}
\NormalTok{  \}}
\end{Highlighting}
\end{Shaded}

Based on the probability of mortality during all of the possible periods
that an individual is considered part of the study, we must draw when
they ``fail'', i.e.~die, using the multinomial distribution (a
categorical distribution could also be used). The age interval when the
failure event occurs, or mortality event, can then be calculated based
on the age at entry. If there is no mortality event, an individual is
right censored. Lastly, based on the age of entry, period of entry, and
right age, we can calculate the period interval of exit from the study,
either through mortality or from right censoring.

\begin{Shaded}
\begin{Highlighting}[]
\NormalTok{  ########################################}
\NormalTok{  ### Calculating right censoring}
\NormalTok{  ########################################}
\NormalTok{  fail_int <-}\StringTok{ }\KeywordTok{rep}\NormalTok{(}\DecValTok{0}\NormalTok{, n)}
\NormalTok{  right_age <-}\StringTok{ }\KeywordTok{rep}\NormalTok{(}\DecValTok{0}\NormalTok{, n)}
\NormalTok{  rt_censor <-}\StringTok{ }\KeywordTok{rep}\NormalTok{(}\DecValTok{0}\NormalTok{, n)}
  \ControlFlowTok{for}\NormalTok{(i }\ControlFlowTok{in} \DecValTok{1}\OperatorTok{:}\NormalTok{n) \{}
\NormalTok{    fail_int[i] <-}\StringTok{ }\KeywordTok{which}\NormalTok{(}\KeywordTok{rmultinom}\NormalTok{(}\DecValTok{1}\NormalTok{, }\DecValTok{1}\NormalTok{, test_stat[i,}\DecValTok{1}\OperatorTok{:}\NormalTok{nT_age]) }\OperatorTok{==}\StringTok{ }\DecValTok{1}\NormalTok{)}
\NormalTok{    right_age[i] <-}\StringTok{ }\KeywordTok{ifelse}\NormalTok{(fail_int[i] }\OperatorTok{>=}\StringTok{ }\NormalTok{left_age[i] }\OperatorTok{+}\StringTok{ }\NormalTok{maxtimes[i],}
\NormalTok{                           left_age[i] }\OperatorTok{+}\StringTok{ }\NormalTok{maxtimes[i], fail_int[i] }\OperatorTok{+}\StringTok{ }\DecValTok{1}\NormalTok{)}
\NormalTok{    rt_censor[i] <-}\StringTok{ }\KeywordTok{ifelse}\NormalTok{(fail_int[i] }\OperatorTok{<}\StringTok{ }\NormalTok{(left_age[i] }\OperatorTok{+}\StringTok{ }\NormalTok{maxtimes[i]), }\DecValTok{0}\NormalTok{, }\DecValTok{1}\NormalTok{)}
\NormalTok{  \}}

\NormalTok{  right_period <-}\StringTok{ }\NormalTok{right_age }\OperatorTok{-}\StringTok{ }\NormalTok{left_age }\OperatorTok{+}\StringTok{ }\NormalTok{left_period}
\end{Highlighting}
\end{Shaded}

Lastly, we return values that were both specified or generated.

\begin{Shaded}
\begin{Highlighting}[]
  \KeywordTok{return}\NormalTok{(}\KeywordTok{list}\NormalTok{(}\DataTypeTok{n =}\NormalTok{ n,}
              \DataTypeTok{nT_age =}\NormalTok{ nT_age,}
              \DataTypeTok{nT_period =}\NormalTok{ nT_period,}
              \DataTypeTok{beta0 =}\NormalTok{ beta0,}
              \DataTypeTok{age_effect =}\NormalTok{ age_effect,}
              \DataTypeTok{period_effect =}\NormalTok{ period_effect,}
              \DataTypeTok{hazard =}\NormalTok{ hazard,}
              \DataTypeTok{test_stat =}\NormalTok{ test_stat,}
              \DataTypeTok{right_age =}\NormalTok{ right_age,}
              \DataTypeTok{left_age =}\NormalTok{ left_age,}
              \DataTypeTok{right_period =}\NormalTok{ right_period,}
              \DataTypeTok{left_period =}\NormalTok{ left_period,}
              \DataTypeTok{rt_censor =}\NormalTok{ rt_censor,}
              \DataTypeTok{prop_right_cens =} \KeywordTok{sum}\NormalTok{(right_age }\OperatorTok{==}\StringTok{ }\NormalTok{nT_age)}\OperatorTok{/}\KeywordTok{length}\NormalTok{(right_age)}
\NormalTok{              )}
\NormalTok{  )}
\end{Highlighting}
\end{Shaded}

The following code provides the full function that can be used to
generate the age-period survival data. It is also provided in the R file
S4\_age\_period\_survival\_generate\_data\_kselect.R.

\begin{Shaded}
\begin{Highlighting}[]
\NormalTok{ageperiod_surv_sim_data_kselect <-}\StringTok{ }\ControlFlowTok{function}\NormalTok{() \{}
  
  \CommentTok{# sample size of individuals from each year}
\NormalTok{  n_yr1 <-}\StringTok{ }\DecValTok{1000}
\NormalTok{  n_yr2 <-}\StringTok{ }\DecValTok{1000}
\NormalTok{  n <-}\StringTok{ }\NormalTok{n_yr1 }\OperatorTok{+}\StringTok{ }\NormalTok{n_yr2}
  
  \CommentTok{# intercept}
\NormalTok{  beta0 <-}\StringTok{ }\OperatorTok{-}\DecValTok{5}

\NormalTok{  ### maximum age interval}
\NormalTok{  nT_age <-}\StringTok{ }\DecValTok{400}

\NormalTok{  ### maximum period interval}
\NormalTok{  nT_period <-}\StringTok{ }\DecValTok{104}

\NormalTok{  #####################################################}
\NormalTok{  ### Age at entry with staggered entry,}
\NormalTok{  ### drawn from a stable age distribution}
\NormalTok{  ### based on the age effect hazard}
\NormalTok{  ### for 2 'years', and for the age that}
\NormalTok{  ### individuals that go from birth to censoring}
\NormalTok{  #####################################################}

  \CommentTok{#Age effects on log hazard}
  \CommentTok{#baseline is a quadratic function}
\NormalTok{  age <-}\StringTok{ }\KeywordTok{seq}\NormalTok{(}\DecValTok{1}\NormalTok{, nT_age }\OperatorTok{-}\StringTok{ }\DecValTok{1}\NormalTok{, }\DataTypeTok{by =} \DecValTok{1}\NormalTok{) }
\NormalTok{  age_effect <-}\StringTok{ }\FloatTok{5e-05} \OperatorTok{*}\StringTok{ }\NormalTok{age}\OperatorTok{^}\DecValTok{2} \OperatorTok{-}\StringTok{ }\FloatTok{0.02} \OperatorTok{*}\StringTok{ }\NormalTok{age }\OperatorTok{+}\StringTok{ }\DecValTok{1} 

  \CommentTok{#Inducing wiggles during the early age intervals}
\NormalTok{  anthro <-}\StringTok{ }\NormalTok{.}\DecValTok{5} \OperatorTok{*}\StringTok{ }\KeywordTok{cos}\NormalTok{(}\DecValTok{1} \OperatorTok{/}\StringTok{ }\DecValTok{26} \OperatorTok{*}\StringTok{ }\NormalTok{pi }\OperatorTok{*}\StringTok{ }\NormalTok{age)}
\NormalTok{  anthro[}\DecValTok{1}\OperatorTok{:}\DecValTok{19}\NormalTok{] <-}\StringTok{ }\DecValTok{0}
\NormalTok{  anthro[}\DecValTok{20}\OperatorTok{:}\DecValTok{38}\NormalTok{] <-}\StringTok{ }\NormalTok{anthro[}\DecValTok{20}\OperatorTok{:}\DecValTok{38}\NormalTok{] }\OperatorTok{*}\StringTok{ }\KeywordTok{seq}\NormalTok{(}\DecValTok{0}\NormalTok{, }\DecValTok{1}\NormalTok{, }\DataTypeTok{length =} \KeywordTok{length}\NormalTok{(}\DecValTok{20}\OperatorTok{:}\DecValTok{38}\NormalTok{))}
\NormalTok{  anthro[}\DecValTok{50}\OperatorTok{:}\DecValTok{170}\NormalTok{] <-}\StringTok{ }\NormalTok{anthro[}\DecValTok{50}\OperatorTok{:}\DecValTok{170}\NormalTok{] }\OperatorTok{*}\StringTok{ }\KeywordTok{seq}\NormalTok{(}\DecValTok{1}\NormalTok{, }\DecValTok{0}\NormalTok{, }\DataTypeTok{length =} \KeywordTok{length}\NormalTok{(}\DecValTok{50}\OperatorTok{:}\DecValTok{170}\NormalTok{))}
\NormalTok{  anthro[}\DecValTok{171}\OperatorTok{:}\NormalTok{(nT_age }\OperatorTok{-}\StringTok{ }\DecValTok{1}\NormalTok{)] <-}\StringTok{ }\DecValTok{0} 
\NormalTok{  age_effect <-}\StringTok{ }\NormalTok{age_effect }\OperatorTok{+}\StringTok{ }\NormalTok{anthro}
\NormalTok{  age_effect <-}\StringTok{ }\NormalTok{age_effect }\OperatorTok{-}\StringTok{ }\KeywordTok{mean}\NormalTok{(age_effect)}

\NormalTok{  hazard_se <-}\StringTok{ }\OperatorTok{-}\KeywordTok{exp}\NormalTok{((beta0 }\OperatorTok{+}\StringTok{ }\NormalTok{age_effect))}
\NormalTok{  stat_se <-}\StringTok{ }\KeywordTok{rep}\NormalTok{(}\OtherTok{NA}\NormalTok{, nT_age }\OperatorTok{-}\StringTok{ }\NormalTok{nT_period)}
\NormalTok{  stat_se[}\DecValTok{1}\NormalTok{] <-}\StringTok{ }\NormalTok{(}\DecValTok{1} \OperatorTok{-}\StringTok{ }\KeywordTok{exp}\NormalTok{(hazard_se[}\DecValTok{1}\NormalTok{]))}
  \ControlFlowTok{for}\NormalTok{ (j }\ControlFlowTok{in} \DecValTok{2}\OperatorTok{:}\NormalTok{(nT_age }\OperatorTok{-}\StringTok{ }\NormalTok{nT_period)) \{}
\NormalTok{    stat_se[j] <-}\StringTok{ }\NormalTok{(}\DecValTok{1} \OperatorTok{-}\StringTok{ }\KeywordTok{exp}\NormalTok{(hazard_se[j])) }\OperatorTok{*}\StringTok{ }\KeywordTok{exp}\NormalTok{(}\KeywordTok{sum}\NormalTok{(hazard_se[}\DecValTok{1}\OperatorTok{:}\NormalTok{(j }\OperatorTok{-}\StringTok{ }\DecValTok{1}\NormalTok{)]))}
\NormalTok{  \}}
\NormalTok{  left_age <-}\StringTok{ }\NormalTok{nimble}\OperatorTok{::}\KeywordTok{rcat}\NormalTok{(n, stat_se)}

\NormalTok{  ###########################################}
  \CommentTok{# Staggered entry period effects}
\NormalTok{  ###########################################}

\NormalTok{  weeks_entry <-}\StringTok{ }\DecValTok{8}
\NormalTok{  left_yr1 <-}\StringTok{ }\NormalTok{nimble}\OperatorTok{::}\KeywordTok{rcat}\NormalTok{(n.yr1, }\DataTypeTok{prob =} \KeywordTok{rep}\NormalTok{(}\DecValTok{1} \OperatorTok{/}\StringTok{ }\NormalTok{weeks_entry, weeks_entry))}
\NormalTok{  left_yr2 <-}\StringTok{ }\NormalTok{nimble}\OperatorTok{::}\KeywordTok{rcat}\NormalTok{(n.yr2, }\DataTypeTok{prob =} \KeywordTok{rep}\NormalTok{(}\DecValTok{1} \OperatorTok{/}\StringTok{ }\NormalTok{weeks_entry, weeks_entry))}
\NormalTok{  left_period <-}\StringTok{ }\KeywordTok{c}\NormalTok{(left_yr1, left_yr2)}

  \CommentTok{# no staggered entry}
\NormalTok{  maxtimes <-}\StringTok{ }\NormalTok{nT_period }\OperatorTok{-}\StringTok{ }\NormalTok{left_period}

\NormalTok{  ########################################}
  \CommentTok{# Age effects for the log hazard}
\NormalTok{  ########################################}

\NormalTok{  age <-}\StringTok{ }\KeywordTok{seq}\NormalTok{(}\DecValTok{1}\NormalTok{, nT_age }\OperatorTok{-}\StringTok{ }\DecValTok{1}\NormalTok{, }\DataTypeTok{by =} \DecValTok{1}\NormalTok{)}
\NormalTok{  age_effect <-}\StringTok{ }\FloatTok{5e-05} \OperatorTok{*}\StringTok{ }\NormalTok{age}\OperatorTok{^}\DecValTok{2} \OperatorTok{-}\StringTok{ }\FloatTok{0.02} \OperatorTok{*}\StringTok{ }\NormalTok{age }\OperatorTok{+}\StringTok{ }\DecValTok{1}
\NormalTok{  age_effect <-}\StringTok{ }\NormalTok{age_effect }\OperatorTok{+}\StringTok{ }\NormalTok{anthro}
\NormalTok{  age_effect <-}\StringTok{ }\NormalTok{age_effect }\OperatorTok{-}\StringTok{ }\KeywordTok{mean}\NormalTok{(age_effect)}
 
\NormalTok{  ########################################}
\NormalTok{  ### Period effects for the log hazard }
\NormalTok{  ########################################}

\NormalTok{  period <-}\StringTok{ }\KeywordTok{seq}\NormalTok{(}\DecValTok{1}\NormalTok{, nT_period }\OperatorTok{-}\StringTok{ }\DecValTok{1}\NormalTok{, }\DataTypeTok{by =} \DecValTok{1}\NormalTok{)}
\NormalTok{  period_effect <-}\StringTok{ }\DecValTok{1} \OperatorTok{*}\StringTok{ }\KeywordTok{sin}\NormalTok{(}\DecValTok{2}\OperatorTok{/}\DecValTok{52} \OperatorTok{*}\StringTok{ }\NormalTok{pi }\OperatorTok{*}\StringTok{ }\NormalTok{(period) }\OperatorTok{+}\StringTok{ }\DecValTok{1}\NormalTok{)}
\NormalTok{  period_effect <-}\StringTok{ }\NormalTok{period_effect }\OperatorTok{-}\StringTok{ }\KeywordTok{mean}\NormalTok{(period_effect)}

\NormalTok{  ########################################}
\NormalTok{  ### Log hazard}
\NormalTok{  ########################################}
\NormalTok{  hazard <-}\StringTok{ }\KeywordTok{matrix}\NormalTok{(}\OtherTok{NA}\NormalTok{, n, nT_age)}
  \ControlFlowTok{for}\NormalTok{ (i }\ControlFlowTok{in} \DecValTok{1}\OperatorTok{:}\NormalTok{n) \{}
\NormalTok{    hazard[i,] <-}\StringTok{ }\KeywordTok{c}\NormalTok{(}\KeywordTok{rep}\NormalTok{(}\DecValTok{0}\NormalTok{,left_age[i] }\OperatorTok{-}\StringTok{ }\DecValTok{1}\NormalTok{), beta0}\OperatorTok{+}
\StringTok{                      }\NormalTok{age_effect[left_age[i]}\OperatorTok{:}\NormalTok{(left_age[i] }\OperatorTok{+}\StringTok{ }\NormalTok{maxtimes[i] }\OperatorTok{-}\StringTok{ }\DecValTok{1}\NormalTok{)] }\OperatorTok{+}
\StringTok{                      }\NormalTok{period_effect[left_period[i]}\OperatorTok{:}\NormalTok{(left_period[i] }\OperatorTok{+}\StringTok{ }\NormalTok{maxtimes[i] }\OperatorTok{-}\StringTok{ }\DecValTok{1}\NormalTok{)],}
                      \KeywordTok{rep}\NormalTok{(}\DecValTok{0}\NormalTok{, nT_age }\OperatorTok{-}\StringTok{ }\NormalTok{(left_age[i] }\OperatorTok{-}\StringTok{ }\DecValTok{1} \OperatorTok{+}\StringTok{ }\NormalTok{maxtimes[i])))}
\NormalTok{  \}}

\NormalTok{  ########################################}
\NormalTok{  ### Probability of mortality}
\NormalTok{  ########################################}

\NormalTok{  test_stat <-}\StringTok{ }\KeywordTok{matrix}\NormalTok{(}\DecValTok{0}\NormalTok{, n, nT_age)}
  \ControlFlowTok{for}\NormalTok{ (i }\ControlFlowTok{in} \DecValTok{1}\OperatorTok{:}\NormalTok{n) \{}
    \ControlFlowTok{for}\NormalTok{ (j }\ControlFlowTok{in}\NormalTok{ left_age[i]}\OperatorTok{:}\NormalTok{(left_age[i] }\OperatorTok{+}\StringTok{ }\NormalTok{maxtimes[i] }\OperatorTok{-}\StringTok{ }\DecValTok{1}\NormalTok{)) \{}
      \ControlFlowTok{if}\NormalTok{ (j}\OperatorTok{==}\NormalTok{left_age[i]) \{}
\NormalTok{        test_stat[i, j] <-}\StringTok{ }\NormalTok{(}\DecValTok{1} \OperatorTok{-}\StringTok{ }\KeywordTok{exp}\NormalTok{(}\OperatorTok{-}\KeywordTok{exp}\NormalTok{(hazard[i, j])))}
\NormalTok{      \} }\ControlFlowTok{else}\NormalTok{ \{}
\NormalTok{        test_stat[i, j] <-}
\StringTok{          }\NormalTok{(}\DecValTok{1} \OperatorTok{-}\StringTok{ }\KeywordTok{exp}\NormalTok{(}\OperatorTok{-}\KeywordTok{exp}\NormalTok{(hazard[i, j])))}\OperatorTok{*}\KeywordTok{exp}\NormalTok{(}\OperatorTok{-}\KeywordTok{sum}\NormalTok{(}\KeywordTok{exp}\NormalTok{(hazard[i, left_age[i]}\OperatorTok{:}\NormalTok{(j }\OperatorTok{-}\StringTok{ }\DecValTok{1}\NormalTok{)])))}
\NormalTok{      \}}
\NormalTok{    \}}
\NormalTok{    test_stat[i, left_age[i] }\OperatorTok{+}\StringTok{ }\NormalTok{maxtimes[i]] <-}
\StringTok{          }\KeywordTok{exp}\NormalTok{(}\OperatorTok{-}\KeywordTok{sum}\NormalTok{(}\KeywordTok{exp}\NormalTok{(hazard[i, left_age[i]}\OperatorTok{:}\NormalTok{(left_age[i] }\OperatorTok{+}\StringTok{ }\NormalTok{maxtimes[i] }\OperatorTok{-}\StringTok{ }\DecValTok{1}\NormalTok{)])))}
\NormalTok{  \}}

\NormalTok{  ########################################}
\NormalTok{  ### Calculating right censoring}
\NormalTok{  ########################################}
\NormalTok{  fail_int <-}\StringTok{ }\KeywordTok{rep}\NormalTok{(}\DecValTok{0}\NormalTok{, n)}
\NormalTok{  right_age <-}\StringTok{ }\KeywordTok{rep}\NormalTok{(}\DecValTok{0}\NormalTok{, n)}
\NormalTok{  rt_censor <-}\StringTok{ }\KeywordTok{rep}\NormalTok{(}\DecValTok{0}\NormalTok{, n)}
  \ControlFlowTok{for}\NormalTok{(i }\ControlFlowTok{in} \DecValTok{1}\OperatorTok{:}\NormalTok{n) \{}
\NormalTok{    fail_int[i] <-}\StringTok{ }\KeywordTok{which}\NormalTok{(}\KeywordTok{rmultinom}\NormalTok{(}\DecValTok{1}\NormalTok{, }\DecValTok{1}\NormalTok{, test_stat[i,}\DecValTok{1}\OperatorTok{:}\NormalTok{nT_age]) }\OperatorTok{==}\StringTok{ }\DecValTok{1}\NormalTok{)}
\NormalTok{    right_age[i] <-}\StringTok{ }\KeywordTok{ifelse}\NormalTok{(fail_int[i] }\OperatorTok{>=}\StringTok{ }\NormalTok{left_age[i] }\OperatorTok{+}\StringTok{ }\NormalTok{maxtimes[i],}
\NormalTok{                           left_age[i] }\OperatorTok{+}\StringTok{ }\NormalTok{maxtimes[i], fail_int[i] }\OperatorTok{+}\StringTok{ }\DecValTok{1}\NormalTok{)}
\NormalTok{    rt_censor[i] <-}\StringTok{ }\KeywordTok{ifelse}\NormalTok{(fail_int[i] }\OperatorTok{<}\StringTok{ }\NormalTok{(left_age[i] }\OperatorTok{+}\StringTok{ }\NormalTok{maxtimes[i]), }\DecValTok{0}\NormalTok{, }\DecValTok{1}\NormalTok{)}
\NormalTok{  \}}

\NormalTok{  right_period <-}\StringTok{ }\NormalTok{right_age }\OperatorTok{-}\StringTok{ }\NormalTok{left_age }\OperatorTok{+}\StringTok{ }\NormalTok{left_period}

\NormalTok{  ########################################}
\NormalTok{  ### Return values}
\NormalTok{  ########################################}

  \KeywordTok{return}\NormalTok{(}\KeywordTok{list}\NormalTok{(}\DataTypeTok{n =}\NormalTok{ n,}
              \DataTypeTok{nT_age =}\NormalTok{ nT_age,}
              \DataTypeTok{nT_period =}\NormalTok{ nT_period,}
              \DataTypeTok{beta0 =}\NormalTok{ beta0,}
              \DataTypeTok{age_effect =}\NormalTok{ age_effect,}
              \DataTypeTok{period_effect =}\NormalTok{ period_effect,}
              \DataTypeTok{hazard =}\NormalTok{ hazard,}
              \DataTypeTok{test_stat =}\NormalTok{ test_stat,}
              \DataTypeTok{right_age =}\NormalTok{ right_age,}
              \DataTypeTok{left_age =}\NormalTok{ left_age,}
              \DataTypeTok{right_period =}\NormalTok{ right_period,}
              \DataTypeTok{left_period =}\NormalTok{ left_period,}
              \DataTypeTok{rt_censor =}\NormalTok{ rt_censor,}
              \DataTypeTok{prop_right_cens =} \KeywordTok{sum}\NormalTok{(right_age }\OperatorTok{==}\StringTok{ }\NormalTok{nT_age) }\OperatorTok{/}\StringTok{ }\KeywordTok{length}\NormalTok{(right_age)}
\NormalTok{              )}
\NormalTok{         )}
\NormalTok{\}}
\end{Highlighting}
\end{Shaded}

\subsection{Rselect Data Generation}\label{rselect-data-generation}

First we have to set constants, including the intercept for the log
hazard, i.e.~the baseline hazard rate, the number of individuals that
are ``collared'', and the maximum number of age intervals and period
intervals.

\begin{Shaded}
\begin{Highlighting}[]
  \CommentTok{# sample size of individuals}
\NormalTok{  n <-}\StringTok{ }\DecValTok{1000}

  \CommentTok{# intercept of the log hazard}
\NormalTok{  beta0 <-}\StringTok{ }\OperatorTok{-}\DecValTok{4}

\NormalTok{  ### maximum age interval}
\NormalTok{  nT_age <-}\StringTok{ }\DecValTok{130}

\NormalTok{  ### maximum period interval}
\NormalTok{  nT_period <-}\StringTok{ }\DecValTok{130}
\end{Highlighting}
\end{Shaded}

Then we must generate the ages of individuals upon entry (left\_age) and
we must generate the period of entry (left\_period). The maximum age
that an individual in the study can live to is the difference between
the maximum period and the entry period (maxtimes). An individual that
is of age 1 that enters into the population can be alive and a maximum
age within the study of (nT\_period - left\_period). Here we specified
all individuals entering the population at age 1, with staggered entry
based on equal probability over the first 50 period intervals.

\begin{Shaded}
\begin{Highlighting}[]
\NormalTok{  ### For the rselect simlation, all individuals enter at age 1}
\NormalTok{  left_age <-}\StringTok{ }\KeywordTok{rep}\NormalTok{(}\DecValTok{1}\NormalTok{, n)}

\NormalTok{  ### Staggered entry during the first 50 intervals}
\NormalTok{  interval_entry <-}\StringTok{ }\DecValTok{50}
\NormalTok{  left_period <-}\StringTok{ }\NormalTok{nimble}\OperatorTok{::}\KeywordTok{rcat}\NormalTok{(n, }\DataTypeTok{prob =} \KeywordTok{rep}\NormalTok{( }\DecValTok{1} \OperatorTok{/}\StringTok{ }\NormalTok{interval_entry, interval_entry))}

\NormalTok{  ### The maximum age that an individual can be alive and in the study}
\NormalTok{  ### must be calculated to restrict the right age during generation}
\NormalTok{  maxtimes <-}\StringTok{ }\NormalTok{nT_period}\OperatorTok{-}\NormalTok{left_period}
\end{Highlighting}
\end{Shaded}

Then we specify the hazard functions. These could be any functional
(mathematical) form. We have tried to use reasonable variants that would
reflect hazards for our study systems. For the age effect for the
R-selected species, we used a Weibull shape curve with extra wiggles
that are introduced by the additive functional shape (anthro). We
allowed the additive effect to gradually start and end over the first
half of the ages using the decay sequence. Then we specified the period
effects to be a periodic curve.

\begin{Shaded}
\begin{Highlighting}[]
  \CommentTok{# Age effects for the log hazard}
  \CommentTok{# Baseline Weibull hazard of Age Effects}
\NormalTok{  lam <-.}\DecValTok{95}
\NormalTok{  age <-}\StringTok{ }\KeywordTok{seq}\NormalTok{(}\DecValTok{1}\NormalTok{, nT_age }\OperatorTok{-}\StringTok{ }\DecValTok{1}\NormalTok{, }\DataTypeTok{by =} \DecValTok{1}\NormalTok{) }
\NormalTok{  age_effect <-}\StringTok{ }\DecValTok{3} \OperatorTok{*}\StringTok{ }\NormalTok{lam }\OperatorTok{*}\StringTok{ }\NormalTok{age}\OperatorTok{^}\NormalTok{(}\OperatorTok{-}\NormalTok{(}\DecValTok{1} \OperatorTok{-}\StringTok{ }\NormalTok{lam))}

  \CommentTok{#Inducing wiggles during the early age intervals}
\NormalTok{  anthro <-}\StringTok{ }\NormalTok{.}\DecValTok{1} \OperatorTok{*}\StringTok{ }\KeywordTok{cos}\NormalTok{(}\DecValTok{1}\OperatorTok{/}\DecValTok{8} \OperatorTok{*}\StringTok{ }\NormalTok{pi }\OperatorTok{*}\StringTok{ }\NormalTok{age)}
\NormalTok{  anthro[}\DecValTok{1}\OperatorTok{:}\DecValTok{50}\NormalTok{] <-}\StringTok{ }\NormalTok{anthro[}\DecValTok{1}\OperatorTok{:}\DecValTok{50}\NormalTok{] }\OperatorTok{*}\StringTok{ }\KeywordTok{seq}\NormalTok{(}\DecValTok{0}\NormalTok{, }\DecValTok{1}\NormalTok{, }\DataTypeTok{length =} \KeywordTok{length}\NormalTok{(}\DecValTok{1}\OperatorTok{:}\DecValTok{50}\NormalTok{))}
\NormalTok{  anthro[}\DecValTok{51}\OperatorTok{:}\DecValTok{103}\NormalTok{] <-}\StringTok{ }\NormalTok{anthro[}\DecValTok{51}\OperatorTok{:}\DecValTok{103}\NormalTok{] }\OperatorTok{*}\StringTok{ }\KeywordTok{seq}\NormalTok{(}\DecValTok{1}\NormalTok{, }\DecValTok{0}\NormalTok{ , }\DataTypeTok{length =} \KeywordTok{length}\NormalTok{(}\DecValTok{51}\OperatorTok{:}\DecValTok{103}\NormalTok{))}
\NormalTok{  anthro[}\DecValTok{103}\OperatorTok{:}\NormalTok{(nT_age}\OperatorTok{-}\DecValTok{1}\NormalTok{)] <-}\StringTok{ }\DecValTok{0}
\NormalTok{  age_effect <-}\StringTok{ }\NormalTok{age_effect }\OperatorTok{+}\StringTok{ }\NormalTok{anthro}
\NormalTok{  age_effect <-}\StringTok{ }\NormalTok{age_effect }\OperatorTok{-}\StringTok{ }\KeywordTok{mean}\NormalTok{(age_effect)}
  
\NormalTok{  ########################################}
\NormalTok{  ### Period effects for the log hazard}
\NormalTok{  ########################################}
\NormalTok{  period <-}\StringTok{ }\KeywordTok{seq}\NormalTok{(}\DecValTok{1}\NormalTok{, nT_period }\OperatorTok{-}\StringTok{ }\DecValTok{1}\NormalTok{, }\DataTypeTok{by =} \DecValTok{1}\NormalTok{)}
\NormalTok{  period_effect <-}\StringTok{ }\NormalTok{.}\DecValTok{5} \OperatorTok{*}\StringTok{ }\KeywordTok{sin}\NormalTok{(}\DecValTok{5}\OperatorTok{/}\NormalTok{(}\DecValTok{120} \OperatorTok{*}\StringTok{ }\NormalTok{pi }\OperatorTok{*}\StringTok{ }\NormalTok{period) }\OperatorTok{+}\StringTok{ }\DecValTok{5}\NormalTok{)}
\NormalTok{  period_effect <-}\StringTok{ }\NormalTok{period_effect }\OperatorTok{-}\StringTok{ }\KeywordTok{mean}\NormalTok{(period_effect)}
\end{Highlighting}
\end{Shaded}

We calculate the log hazard by the assuming the age effects and period
effects are additive. Therefore, we add the age effects and period
effects to the intercept baseline hazard from the age at entry to the
maximum possible age, and across the period effects from left entry to
the maximum possible period.

\begin{Shaded}
\begin{Highlighting}[]
  \CommentTok{# Calculating the Hazard}
\NormalTok{  hazard <-}\StringTok{ }\KeywordTok{matrix}\NormalTok{(}\OtherTok{NA}\NormalTok{, n, nT_age)}
  \ControlFlowTok{for}\NormalTok{ (i }\ControlFlowTok{in} \DecValTok{1}\OperatorTok{:}\NormalTok{n) \{}
\NormalTok{    hazard[i,] <-}\StringTok{ }\KeywordTok{c}\NormalTok{(}\KeywordTok{rep}\NormalTok{(}\DecValTok{0}\NormalTok{,left_age[i]}\OperatorTok{-}\DecValTok{1}\NormalTok{), beta0 }\OperatorTok{+}
\StringTok{                      }\NormalTok{age_effect[left_age[i]}\OperatorTok{:}\NormalTok{(left_age[i] }\OperatorTok{+}\StringTok{ }\NormalTok{maxtimes[i] }\OperatorTok{-}\StringTok{ }\DecValTok{1}\NormalTok{)] }\OperatorTok{+}
\StringTok{                      }\NormalTok{period_effect[left_period[i]}\OperatorTok{:}\NormalTok{(left_period[i] }\OperatorTok{+}\StringTok{ }\NormalTok{maxtimes[i] }\OperatorTok{-}\StringTok{ }\DecValTok{1}\NormalTok{)],}
                      \KeywordTok{rep}\NormalTok{(}\DecValTok{0}\NormalTok{, nT_age }\OperatorTok{-}\StringTok{ }\NormalTok{(left_age[i] }\OperatorTok{-}\StringTok{ }\DecValTok{1} \OperatorTok{+}\StringTok{ }\NormalTok{maxtimes[i])))}
\NormalTok{  \}}
\end{Highlighting}
\end{Shaded}

Then we calculate the probability of mortality in each interval based on
the complementary log-log link function. The first period that an
individual enters the study has a probability of mortality during that
period that is a straightforward inverse of the complementary log-log
link transformation. The subsequent periods depend on the cumulative
probability of surviving the previous intervals.

\begin{Shaded}
\begin{Highlighting}[]
  \CommentTok{# Calculating the probability of mortality in each interval}
\NormalTok{  test_stat <-}\StringTok{ }\KeywordTok{matrix}\NormalTok{(}\DecValTok{0}\NormalTok{, n, nT_age)}
  \ControlFlowTok{for}\NormalTok{ (i }\ControlFlowTok{in} \DecValTok{1}\OperatorTok{:}\NormalTok{n) \{}
    \ControlFlowTok{for}\NormalTok{ (j }\ControlFlowTok{in}\NormalTok{ left_age[i]}\OperatorTok{:}\NormalTok{(left_age[i] }\OperatorTok{+}\StringTok{ }\NormalTok{maxtimes[i] }\OperatorTok{-}\StringTok{ }\DecValTok{1}\NormalTok{)) \{}
      \ControlFlowTok{if}\NormalTok{ (j }\OperatorTok{==}\StringTok{ }\NormalTok{left_age[i]) \{}
\NormalTok{        test_stat[i, j] <-}\StringTok{ }\NormalTok{(}\DecValTok{1} \OperatorTok{-}\StringTok{ }\KeywordTok{exp}\NormalTok{(}\OperatorTok{-}\KeywordTok{exp}\NormalTok{(hazard[i, j])))}
\NormalTok{      \} }\ControlFlowTok{else}\NormalTok{ \{}
\NormalTok{        test_stat[i, j] <-}
\StringTok{           }\NormalTok{(}\DecValTok{1} \OperatorTok{-}\StringTok{ }\KeywordTok{exp}\NormalTok{(}\OperatorTok{-}\KeywordTok{exp}\NormalTok{(hazard[i, j]))) }\OperatorTok{*}
\StringTok{           }\KeywordTok{exp}\NormalTok{(}\OperatorTok{-}\KeywordTok{sum}\NormalTok{(}\KeywordTok{exp}\NormalTok{(hazard[i, left_age[i]}\OperatorTok{:}\NormalTok{(j }\OperatorTok{-}\StringTok{ }\DecValTok{1}\NormalTok{)])))}
\NormalTok{      \}}
\NormalTok{    \}}
\NormalTok{    test_stat[i, left_age[i] }\OperatorTok{+}\StringTok{ }\NormalTok{maxtimes[i]] <-}
\StringTok{            }\KeywordTok{exp}\NormalTok{(}\OperatorTok{-}\KeywordTok{sum}\NormalTok{(}\KeywordTok{exp}\NormalTok{(hazard[i, left_age[i]}\OperatorTok{:}\NormalTok{(left_age[i]}
            \OperatorTok{+}\StringTok{ }\NormalTok{maxtimes[i] }\OperatorTok{-}\StringTok{ }\DecValTok{1}\NormalTok{)])))}
\NormalTok{  \}}
\end{Highlighting}
\end{Shaded}

Based on the probability of mortality during all of the possible periods
that an individual is considered part of the study, we must draw when
they die using the multinomial distribution (a categorical distribution
could also be used). The age when the failure, or mortality event occurs
can be calculated based on the age at entry. If there is no mortality
event, an individual is right censored. Lastly, based on the age of
entry, period of entry, and right age, we calculate the period of exit
from the study, either through mortality or from right censoring.

\begin{Shaded}
\begin{Highlighting}[]
\CommentTok{# calculating right censoring}
\NormalTok{  fail_int <-}\StringTok{ }\KeywordTok{rep}\NormalTok{(}\DecValTok{0}\NormalTok{, n)}
\NormalTok{  right_age <-}\StringTok{ }\KeywordTok{rep}\NormalTok{(}\DecValTok{0}\NormalTok{, n)}
\NormalTok{  rt_censor <-}\StringTok{ }\KeywordTok{rep}\NormalTok{(}\DecValTok{0}\NormalTok{, n)}
  \ControlFlowTok{for}\NormalTok{(i }\ControlFlowTok{in} \DecValTok{1}\OperatorTok{:}\NormalTok{n)\{}
\NormalTok{    fail_int[i] <-}\StringTok{ }\KeywordTok{which}\NormalTok{(}\KeywordTok{rmultinom}\NormalTok{(}\DecValTok{1}\NormalTok{, }\DecValTok{1}\NormalTok{, test_stat[i, }\DecValTok{1}\OperatorTok{:}\NormalTok{nT_age]) }\OperatorTok{==}\StringTok{ }\DecValTok{1}\NormalTok{)}
\NormalTok{    right_age[i] <-}\StringTok{ }\KeywordTok{ifelse}\NormalTok{(fail_int[i] }\OperatorTok{>=}\StringTok{ }\NormalTok{left_age[i] }\OperatorTok{+}\StringTok{ }\NormalTok{maxtimes[i],}
\NormalTok{                          left_age[i] }\OperatorTok{+}\StringTok{ }\NormalTok{maxtimes[i],}
\NormalTok{                          fail_int[i] }\OperatorTok{+}\StringTok{ }\DecValTok{1}\NormalTok{)}
\NormalTok{    rt_censor[i] <-}\StringTok{ }\KeywordTok{ifelse}\NormalTok{(fail_int[i] }\OperatorTok{<}\StringTok{ }\NormalTok{(left_age[i] }\OperatorTok{+}\StringTok{ }\NormalTok{maxtimes[i]), }\DecValTok{0}\NormalTok{, }\DecValTok{1}\NormalTok{)}
\NormalTok{  \}}
\NormalTok{  right_period <-}\StringTok{ }\NormalTok{right_age }\OperatorTok{-}\StringTok{ }\NormalTok{left_age }\OperatorTok{+}\StringTok{ }\NormalTok{left_period}
\end{Highlighting}
\end{Shaded}

Lastly, we return on values that were specified or generated.

\begin{Shaded}
\begin{Highlighting}[]
  \KeywordTok{return}\NormalTok{(}\KeywordTok{list}\NormalTok{(}\DataTypeTok{n =}\NormalTok{ n,}
              \DataTypeTok{nT_age =}\NormalTok{ nT_age,}
              \DataTypeTok{nT_period =}\NormalTok{ nT_period,}
              \DataTypeTok{beta0 =}\NormalTok{ beta0,}
              \DataTypeTok{age_effect =}\NormalTok{ age_effect,}
              \DataTypeTok{period_effect =}\NormalTok{ period_effect,}
              \DataTypeTok{hazard =}\NormalTok{ hazard,}
              \DataTypeTok{test_stat =}\NormalTok{ test_stat,}
              \DataTypeTok{right_age =}\NormalTok{ right_age,}
              \DataTypeTok{left_age =}\NormalTok{ left_age,}
              \DataTypeTok{right_period =}\NormalTok{ right_period,}
              \DataTypeTok{left_period =}\NormalTok{ left_period,}
              \DataTypeTok{rt_censor =}\NormalTok{ rt_censor,}
              \DataTypeTok{prop_right_cens =} \KeywordTok{sum}\NormalTok{(right_age }\OperatorTok{==}\StringTok{ }\NormalTok{nT_age)}\OperatorTok{/}\KeywordTok{length}\NormalTok{(right_age)}
\NormalTok{              )}
\NormalTok{  )}
\end{Highlighting}
\end{Shaded}

The following code provides the full function that can be used to
generate the age-period survival data. It is also provided in the R file
S4\_age\_period\_survival\_generate\_data\_rselect.R.

\begin{Shaded}
\begin{Highlighting}[]
\NormalTok{ageperiod_surv_sim_data_rselect <-}\StringTok{ }\ControlFlowTok{function}\NormalTok{() \{}

  \CommentTok{# sample size of individuals}
\NormalTok{  n <-}\StringTok{ }\DecValTok{1000}

  \CommentTok{# intercept of the log hazard}
\NormalTok{  beta0 <-}\StringTok{ }\OperatorTok{-}\DecValTok{4}

\NormalTok{  ### maximum age interval}
\NormalTok{  nT_age <-}\StringTok{ }\DecValTok{130}

\NormalTok{  ### maximum period interval}
\NormalTok{  nT_period <-}\StringTok{ }\DecValTok{130}

\NormalTok{  #####################################################}
\NormalTok{  ### Age at entry. np staggered entry}
\NormalTok{  #####################################################}

\NormalTok{  ### For the rselect simlation, all individuals enter at age 1}
\NormalTok{  left_age <-}\StringTok{ }\KeywordTok{rep}\NormalTok{(}\DecValTok{1}\NormalTok{, n)}

\NormalTok{  ###########################################}
\NormalTok{  ### Staggered entry period effects}
\NormalTok{  ###########################################}

\NormalTok{  ### Staggered entry during the first 50 intervals}
\NormalTok{  interval_entry <-}\StringTok{ }\DecValTok{50}
\NormalTok{  left_period <-}\StringTok{ }\NormalTok{nimble}\OperatorTok{::}\KeywordTok{rcat}\NormalTok{(n, }\DataTypeTok{prob =} \KeywordTok{rep}\NormalTok{( }\DecValTok{1} \OperatorTok{/}\StringTok{ }\NormalTok{interval_entry, interval_entry))}

\NormalTok{  ### The maximum age that an individual can be alive and in the study}
\NormalTok{  ### must be calculated to restrict the right age during generation}
\NormalTok{  maxtimes <-}\StringTok{ }\NormalTok{nT_period}\OperatorTok{-}\NormalTok{left_period}

\NormalTok{  ########################################}
\NormalTok{  ### Age effects for the log hazard}
\NormalTok{  ### Baseline Weibull hazard of Age Effects}
\NormalTok{  ########################################}

\NormalTok{  lam <-.}\DecValTok{95}
\NormalTok{  age <-}\StringTok{ }\KeywordTok{seq}\NormalTok{(}\DecValTok{1}\NormalTok{, nT_age }\OperatorTok{-}\StringTok{ }\DecValTok{1}\NormalTok{, }\DataTypeTok{by =} \DecValTok{1}\NormalTok{) }
\NormalTok{  age_effect <-}\StringTok{ }\DecValTok{3} \OperatorTok{*}\StringTok{ }\NormalTok{lam }\OperatorTok{*}\StringTok{ }\NormalTok{age}\OperatorTok{^}\NormalTok{(}\OperatorTok{-}\NormalTok{(}\DecValTok{1} \OperatorTok{-}\StringTok{ }\NormalTok{lam))}

  \CommentTok{#Inducing wiggles during the early age intervals}
\NormalTok{  anthro <-}\StringTok{ }\NormalTok{.}\DecValTok{1} \OperatorTok{*}\StringTok{ }\KeywordTok{cos}\NormalTok{(}\DecValTok{1}\OperatorTok{/}\DecValTok{8} \OperatorTok{*}\StringTok{ }\NormalTok{pi }\OperatorTok{*}\StringTok{ }\NormalTok{age)}
\NormalTok{  anthro[}\DecValTok{1}\OperatorTok{:}\DecValTok{50}\NormalTok{] <-}\StringTok{ }\NormalTok{anthro[}\DecValTok{1}\OperatorTok{:}\DecValTok{50}\NormalTok{] }\OperatorTok{*}\StringTok{ }\KeywordTok{seq}\NormalTok{(}\DecValTok{0}\NormalTok{,}\DecValTok{1}\NormalTok{,}\DataTypeTok{length =} \KeywordTok{length}\NormalTok{(}\DecValTok{1}\OperatorTok{:}\DecValTok{50}\NormalTok{))}
\NormalTok{  anthro[}\DecValTok{51}\OperatorTok{:}\DecValTok{103}\NormalTok{] <-}\StringTok{ }\NormalTok{anthro[}\DecValTok{51}\OperatorTok{:}\DecValTok{103}\NormalTok{] }\OperatorTok{*}\StringTok{ }\KeywordTok{seq}\NormalTok{(}\DecValTok{1}\NormalTok{,}\DecValTok{0}\NormalTok{,}\DataTypeTok{length =} \KeywordTok{length}\NormalTok{(}\DecValTok{51}\OperatorTok{:}\DecValTok{103}\NormalTok{))}
\NormalTok{  anthro[}\DecValTok{103}\OperatorTok{:}\NormalTok{(nT_age}\OperatorTok{-}\DecValTok{1}\NormalTok{)] <-}\StringTok{ }\DecValTok{0}
\NormalTok{  age_effect <-}\StringTok{ }\NormalTok{age_effect }\OperatorTok{+}\StringTok{ }\NormalTok{anthro}
\NormalTok{  age_effect <-}\StringTok{ }\NormalTok{age_effect }\OperatorTok{-}\StringTok{ }\KeywordTok{mean}\NormalTok{(age_effect)}


\NormalTok{  ########################################}
\NormalTok{  ### Period effects for the log hazard}
\NormalTok{  ########################################}

\NormalTok{  period <-}\StringTok{ }\KeywordTok{seq}\NormalTok{(}\DecValTok{1}\NormalTok{, nT_period }\OperatorTok{-}\StringTok{ }\DecValTok{1}\NormalTok{, }\DataTypeTok{by =} \DecValTok{1}\NormalTok{)}
\NormalTok{  period_effect <-}\StringTok{ }\NormalTok{.}\DecValTok{5} \OperatorTok{*}\StringTok{ }\KeywordTok{sin}\NormalTok{(}\DecValTok{5}\OperatorTok{/}\NormalTok{(}\DecValTok{120} \OperatorTok{*}\StringTok{ }\NormalTok{pi }\OperatorTok{*}\StringTok{ }\NormalTok{period) }\OperatorTok{+}\StringTok{ }\DecValTok{5}\NormalTok{)}
\NormalTok{  period_effect <-}\StringTok{ }\NormalTok{period_effect }\OperatorTok{-}\StringTok{ }\KeywordTok{mean}\NormalTok{(period_effect)}


\NormalTok{  ########################################}
\NormalTok{  ### Log hazard}
\NormalTok{  ########################################}

\NormalTok{  hazard <-}\StringTok{ }\KeywordTok{matrix}\NormalTok{(}\OtherTok{NA}\NormalTok{, n, nT_age)}

  \ControlFlowTok{for}\NormalTok{ (i }\ControlFlowTok{in} \DecValTok{1}\OperatorTok{:}\NormalTok{n) \{}
\NormalTok{    hazard[i,] <-}\StringTok{ }\KeywordTok{c}\NormalTok{(}\KeywordTok{rep}\NormalTok{(}\DecValTok{0}\NormalTok{,left_age[i]}\OperatorTok{-}\DecValTok{1}\NormalTok{), beta0 }\OperatorTok{+}
\StringTok{                      }\NormalTok{age_effect[left_age[i]}\OperatorTok{:}\NormalTok{(left_age[i] }\OperatorTok{+}\StringTok{ }\NormalTok{maxtimes[i] }\OperatorTok{-}\StringTok{ }\DecValTok{1}\NormalTok{)] }\OperatorTok{+}
\StringTok{                      }\NormalTok{period_effect[left_period[i]}\OperatorTok{:}\NormalTok{(left_period[i] }\OperatorTok{+}\StringTok{ }\NormalTok{maxtimes[i] }\OperatorTok{-}\StringTok{ }\DecValTok{1}\NormalTok{)],}
                      \KeywordTok{rep}\NormalTok{(}\DecValTok{0}\NormalTok{, nT_age }\OperatorTok{-}\StringTok{ }\NormalTok{(left_age[i] }\OperatorTok{-}\StringTok{ }\DecValTok{1} \OperatorTok{+}\StringTok{ }\NormalTok{maxtimes[i])))}
\NormalTok{  \}}

\NormalTok{  ########################################}
\NormalTok{  ### Probability of mortality}
\NormalTok{  ########################################}

\NormalTok{  test_stat <-}\StringTok{ }\KeywordTok{matrix}\NormalTok{(}\DecValTok{0}\NormalTok{, n, nT_age)}
  \ControlFlowTok{for}\NormalTok{ (i }\ControlFlowTok{in} \DecValTok{1}\OperatorTok{:}\NormalTok{n) \{}
    \ControlFlowTok{for}\NormalTok{ (j }\ControlFlowTok{in}\NormalTok{ left_age[i]}\OperatorTok{:}\NormalTok{(left_age[i] }\OperatorTok{+}\StringTok{ }\NormalTok{maxtimes[i] }\OperatorTok{-}\StringTok{ }\DecValTok{1}\NormalTok{)) \{}
      \ControlFlowTok{if}\NormalTok{ (j }\OperatorTok{==}\StringTok{ }\NormalTok{left_age[i]) \{}
\NormalTok{        test_stat[i, j] <-}\StringTok{ }\NormalTok{(}\DecValTok{1} \OperatorTok{-}\StringTok{ }\KeywordTok{exp}\NormalTok{(}\OperatorTok{-}\KeywordTok{exp}\NormalTok{(hazard[i, j])))}
\NormalTok{      \} }\ControlFlowTok{else}\NormalTok{ \{}
\NormalTok{        test_stat[i, j] <-}
\StringTok{           }\NormalTok{(}\DecValTok{1} \OperatorTok{-}\StringTok{ }\KeywordTok{exp}\NormalTok{(}\OperatorTok{-}\KeywordTok{exp}\NormalTok{(hazard[i, j]))) }\OperatorTok{*}
\StringTok{           }\KeywordTok{exp}\NormalTok{(}\OperatorTok{-}\KeywordTok{sum}\NormalTok{(}\KeywordTok{exp}\NormalTok{(hazard[i, left_age[i]}\OperatorTok{:}\NormalTok{(j }\OperatorTok{-}\StringTok{ }\DecValTok{1}\NormalTok{)])))}
\NormalTok{      \}}
\NormalTok{    \}}
\NormalTok{    test_stat[i, left_age[i] }\OperatorTok{+}\StringTok{ }\NormalTok{maxtimes[i]] <-}
\StringTok{            }\KeywordTok{exp}\NormalTok{(}\OperatorTok{-}\KeywordTok{sum}\NormalTok{(}\KeywordTok{exp}\NormalTok{(hazard[i, left_age[i]}\OperatorTok{:}\NormalTok{(left_age[i]}
            \OperatorTok{+}\StringTok{ }\NormalTok{maxtimes[i] }\OperatorTok{-}\StringTok{ }\DecValTok{1}\NormalTok{)])))}
\NormalTok{  \}}

\NormalTok{  ########################################}
\NormalTok{  ### Calculating right censoring}
\NormalTok{  ########################################}
\NormalTok{  fail_int <-}\StringTok{ }\KeywordTok{rep}\NormalTok{(}\DecValTok{0}\NormalTok{, n)}
\NormalTok{  right_age <-}\StringTok{ }\KeywordTok{rep}\NormalTok{(}\DecValTok{0}\NormalTok{, n)}
\NormalTok{  rt_censor <-}\StringTok{ }\KeywordTok{rep}\NormalTok{(}\DecValTok{0}\NormalTok{, n)}
  \ControlFlowTok{for}\NormalTok{(i }\ControlFlowTok{in} \DecValTok{1}\OperatorTok{:}\NormalTok{n)\{}
\NormalTok{    fail_int[i] <-}\StringTok{ }\KeywordTok{which}\NormalTok{(}\KeywordTok{rmultinom}\NormalTok{(}\DecValTok{1}\NormalTok{, }\DecValTok{1}\NormalTok{, test_stat[i, }\DecValTok{1}\OperatorTok{:}\NormalTok{nT_age]) }\OperatorTok{==}\StringTok{ }\DecValTok{1}\NormalTok{)}
\NormalTok{    right_age[i] <-}\StringTok{ }\KeywordTok{ifelse}\NormalTok{(fail_int[i] }\OperatorTok{>=}\StringTok{ }\NormalTok{left_age[i] }\OperatorTok{+}\StringTok{ }\NormalTok{maxtimes[i],}
\NormalTok{                          left_age[i] }\OperatorTok{+}\StringTok{ }\NormalTok{maxtimes[i],}
\NormalTok{                          fail_int[i] }\OperatorTok{+}\StringTok{ }\DecValTok{1}\NormalTok{)}
\NormalTok{    rt_censor[i] <-}\StringTok{ }\KeywordTok{ifelse}\NormalTok{(fail_int[i] }\OperatorTok{<}\StringTok{ }\NormalTok{(left_age[i] }\OperatorTok{+}\StringTok{ }\NormalTok{maxtimes[i]), }\DecValTok{0}\NormalTok{, }\DecValTok{1}\NormalTok{)}
\NormalTok{  \}}

\NormalTok{  right_period <-}\StringTok{ }\NormalTok{right_age }\OperatorTok{-}\StringTok{ }\NormalTok{left_age }\OperatorTok{+}\StringTok{ }\NormalTok{left_period}

\NormalTok{  ########################################}
\NormalTok{  ### Return values}
\NormalTok{  ########################################}
  
  \KeywordTok{return}\NormalTok{(}\KeywordTok{list}\NormalTok{(}\DataTypeTok{n =}\NormalTok{ n,}
              \DataTypeTok{nT_age =}\NormalTok{ nT_age,}
              \DataTypeTok{nT_period =}\NormalTok{ nT_period,}
              \DataTypeTok{beta0 =}\NormalTok{ beta0,}
              \DataTypeTok{age_effect =}\NormalTok{ age_effect,}
              \DataTypeTok{period_effect =}\NormalTok{ period_effect,}
              \DataTypeTok{hazard =}\NormalTok{ hazard,}
              \DataTypeTok{test_stat =}\NormalTok{ test_stat,}
              \DataTypeTok{right_age =}\NormalTok{ right_age,}
              \DataTypeTok{left_age =}\NormalTok{ left_age,}
              \DataTypeTok{right_period =}\NormalTok{ right_period,}
              \DataTypeTok{left_period =}\NormalTok{ left_period,}
              \DataTypeTok{rt_censor =}\NormalTok{ rt_censor,}
              \DataTypeTok{prop_right_cens =} \KeywordTok{sum}\NormalTok{(right_age }\OperatorTok{==}\StringTok{ }\NormalTok{nT_age)}\OperatorTok{/}\KeywordTok{length}\NormalTok{(right_age)}
\NormalTok{              )}
\NormalTok{  )}
\NormalTok{\}}
\end{Highlighting}
\end{Shaded}

\subsection{Format data}\label{format-data}

We set a seed so that results can be recovered (set.seed(10000)). When
these simulations were run on the CHTC, we used seed from (10000,
\ldots{}., 10099). We generated the data by sourcing the previous data
generating functions. Then we set the true values of all parameters, and
derived survival probabilities and hazards based on those true values
returned from the data generating function.

\begin{Shaded}
\begin{Highlighting}[]
\KeywordTok{set.seed}\NormalTok{(}\DecValTok{10000}\NormalTok{)}

\NormalTok{###simulating age period data}
\KeywordTok{source}\NormalTok{(}\StringTok{"S4_01_age_period_survival_generate_data_kselect.R"}\NormalTok{)}
\NormalTok{ageperiod_out<-}\KeywordTok{ageperiod_surv_sim_data_kselect}\NormalTok{()}

\NormalTok{### Setting constants}
\NormalTok{nT_age <-}\StringTok{ }\KeywordTok{max}\NormalTok{(ageperiod_out}\OperatorTok{$}\NormalTok{right_age)}\OperatorTok{-}\DecValTok{1}  \CommentTok{#ageperiod_out$nT_age-1}
\NormalTok{nT_period <-}\StringTok{ }\KeywordTok{max}\NormalTok{(ageperiod_out}\OperatorTok{$}\NormalTok{right_period)}\OperatorTok{-}\DecValTok{1}  

\NormalTok{### Setting true values for log hazard}
\NormalTok{beta0_true <-}\StringTok{ }\NormalTok{ageperiod_out}\OperatorTok{$}\NormalTok{beta0}
\NormalTok{age_effect_true <-}\StringTok{ }\NormalTok{ageperiod_out}\OperatorTok{$}\NormalTok{age_effect}
\NormalTok{period_effect_true <-}\StringTok{ }\NormalTok{ageperiod_out}\OperatorTok{$}\NormalTok{period_effect}

\CommentTok{#Initialize vectors for sub-calculations of Survival}
\NormalTok{llambda_age_true <-}\StringTok{ }\KeywordTok{rep}\NormalTok{(}\OtherTok{NA}\NormalTok{, nT_age)}
\NormalTok{UCH0_age_true <-}\StringTok{ }\KeywordTok{rep}\NormalTok{(}\OtherTok{NA}\NormalTok{, nT_age)}
\NormalTok{S0_age_true <-}\StringTok{ }\KeywordTok{rep}\NormalTok{(}\OtherTok{NA}\NormalTok{, nT_age)}

\ControlFlowTok{for}\NormalTok{ (t }\ControlFlowTok{in} \DecValTok{1}\OperatorTok{:}\NormalTok{nT_age) \{}
\NormalTok{  llambda_age_true[t] <-}\StringTok{ }\NormalTok{beta0_true }\OperatorTok{+}\StringTok{ }\NormalTok{age_effect_true[t]}
\NormalTok{  UCH0_age_true[t] <-}\StringTok{ }\KeywordTok{exp}\NormalTok{(llambda_age_true[t])}
\NormalTok{  S0_age_true[t] <-}\StringTok{ }\KeywordTok{exp}\NormalTok{(}\OperatorTok{-}\KeywordTok{sum}\NormalTok{(UCH0_age_true[}\DecValTok{1}\OperatorTok{:}\NormalTok{t]))}
\NormalTok{\}}

\CommentTok{#Initialize vectors for sub-calculations of Survival}
\NormalTok{llambda_period_true <-}\StringTok{ }\KeywordTok{rep}\NormalTok{(}\OtherTok{NA}\NormalTok{, nT_period)}
\NormalTok{UCH0_period_true <-}\StringTok{ }\KeywordTok{rep}\NormalTok{(}\OtherTok{NA}\NormalTok{, nT_period)}
\NormalTok{S0_period_true <-}\StringTok{ }\KeywordTok{rep}\NormalTok{(}\OtherTok{NA}\NormalTok{, nT_period)}

\ControlFlowTok{for}\NormalTok{ (t }\ControlFlowTok{in} \DecValTok{1}\OperatorTok{:}\NormalTok{nT_period)\{}
\NormalTok{  llambda_period_true[t] <-}\StringTok{ }\NormalTok{beta0_true }\OperatorTok{+}\StringTok{ }\NormalTok{period_effect_true[t] }\CommentTok{#female}
\NormalTok{  UCH0_period_true[t] <-}\StringTok{ }\KeywordTok{exp}\NormalTok{(llambda_period_true[t])}
\NormalTok{  S0_period_true[t] <-}\StringTok{ }\KeywordTok{exp}\NormalTok{(}\OperatorTok{-}\KeywordTok{sum}\NormalTok{(UCH0_period_true[}\DecValTok{1}\OperatorTok{:}\NormalTok{t]))}
\NormalTok{\}}

\NormalTok{### Set data generated from the data generating function}
\NormalTok{left_age <-}\StringTok{ }\NormalTok{ageperiod_out}\OperatorTok{$}\NormalTok{left_age}
\NormalTok{right_age <-}\StringTok{ }\NormalTok{ageperiod_out}\OperatorTok{$}\NormalTok{right_age}
\NormalTok{left_period <-}\StringTok{ }\NormalTok{ageperiod_out}\OperatorTok{$}\NormalTok{left_period}
\NormalTok{right_period <-}\StringTok{ }\NormalTok{ageperiod_out}\OperatorTok{$}\NormalTok{right_period}
\NormalTok{rt_censor <-}\StringTok{ }\NormalTok{ageperiod_out}\OperatorTok{$}\NormalTok{rt_censor}
\NormalTok{n <-}\StringTok{ }\NormalTok{ageperiod_out}\OperatorTok{$}\NormalTok{n}
\end{Highlighting}
\end{Shaded}

We structure the survival data such that each individual that dies
during the study has 2 rows within the data frame, and individuals that
are right censored or interval censored have 1 row in the data, where
entry (left) and exit (right) for both age intervals and period
intervals are specified. When individuals are alive and in the study,
the censored variable is 1. Individuals that change state, through
mortality, have censored equal to 0 for the second row of their entry.
To align the age and period effects within the unit cumulative hazard
calculation, i.e.~to align the indexes in the for loops for calculating
the hazard, a vector aligning the ages to the dates must be calculated
(age2date).

\begin{Shaded}
\begin{Highlighting}[]
\NormalTok{### Formatting data for entering into the model}
\NormalTok{base <-}\StringTok{ }\KeywordTok{rep}\NormalTok{(}\DecValTok{0}\NormalTok{, }\DecValTok{5}\NormalTok{)}
\ControlFlowTok{for}\NormalTok{ (i }\ControlFlowTok{in} \DecValTok{1}\OperatorTok{:}\NormalTok{n) \{}
  \ControlFlowTok{if}\NormalTok{ (rt_censor[i] }\OperatorTok{==}\StringTok{ }\DecValTok{0}\NormalTok{) \{}
\NormalTok{    temp1 <-}\StringTok{ }\KeywordTok{c}\NormalTok{(left_age[i],}
\NormalTok{               right_age[i] }\OperatorTok{-}\StringTok{ }\DecValTok{1}\NormalTok{,}
               \DecValTok{1}\NormalTok{,}
\NormalTok{               left_period[i],}
\NormalTok{               right_period[i] }\OperatorTok{-}\StringTok{ }\DecValTok{1}\NormalTok{)}
\NormalTok{    temp2 <-}\StringTok{ }\KeywordTok{c}\NormalTok{(right_age[i] }\OperatorTok{-}\StringTok{ }\DecValTok{1}\NormalTok{,}
\NormalTok{               right_age[i],}
               \DecValTok{0}\NormalTok{,}
\NormalTok{               right_period[i] }\OperatorTok{-}\StringTok{ }\DecValTok{1}\NormalTok{,}
\NormalTok{               right_period[i])}
    \ControlFlowTok{if}\NormalTok{ (left_age[i] }\OperatorTok{==}\StringTok{ }\NormalTok{(right_age[i] }\OperatorTok{-}\StringTok{ }\DecValTok{1}\NormalTok{)) \{}
\NormalTok{      base <-}\StringTok{ }\KeywordTok{rbind}\NormalTok{(base, temp2)}
\NormalTok{    \} }\ControlFlowTok{else}\NormalTok{ \{}
\NormalTok{      base <-}\StringTok{ }\KeywordTok{rbind}\NormalTok{(base, temp1, temp2)}
\NormalTok{    \}}
\NormalTok{  \} }\ControlFlowTok{else}\NormalTok{ \{}
\NormalTok{    base <-}\StringTok{ }\KeywordTok{rbind}\NormalTok{(base,}\KeywordTok{c}\NormalTok{(left_age[i],}
\NormalTok{                         right_age[i],}
                         \DecValTok{1}\NormalTok{,}
\NormalTok{                         left_period[i],}
\NormalTok{                         right_period[i])) }
\NormalTok{  \}}
\NormalTok{\}}
\NormalTok{base <-}\StringTok{ }\NormalTok{base[}\OperatorTok{-}\DecValTok{1}\NormalTok{,]}
\KeywordTok{rownames}\NormalTok{(base) <-}\StringTok{ }\OtherTok{NULL}
\KeywordTok{colnames}\NormalTok{(base) <-}\StringTok{ }\KeywordTok{c}\NormalTok{(}\StringTok{"left_age"}\NormalTok{,}
                    \StringTok{"right_age"}\NormalTok{,}
                    \StringTok{"censored"}\NormalTok{,}
                    \StringTok{"left_period"}\NormalTok{,}
                    \StringTok{"right_period"}\NormalTok{)}
\NormalTok{df_fit <-}\StringTok{ }\KeywordTok{as.data.frame}\NormalTok{(base)}
\NormalTok{n_fit <-}\StringTok{ }\KeywordTok{dim}\NormalTok{(df_fit)[}\DecValTok{1}\NormalTok{]}

\NormalTok{### Create age to date conversion vector}
\NormalTok{### This aligns the age/period intervals when }
\NormalTok{### looping over the age and period hazards}
\NormalTok{age2date <-}\StringTok{ }\NormalTok{df_fit}\OperatorTok{$}\NormalTok{left_age }\OperatorTok{-}\StringTok{ }\NormalTok{df_fit}\OperatorTok{$}\NormalTok{left_period}
\end{Highlighting}
\end{Shaded}

The data formatting for both the R-select and K-select species is
identical, except that the function that is sourced and executed for
generating data should be swapped out. Consequently, we have only
described the data formatting here once. We have provided full code for
formatting both simulations in S4\_02\_format\_data\_kselect.R and
S4\_02\_format\_data\_rselect.R. These files are sourced in the calls to
the model fitting R scripts for the simulations.

\subsection{Calculating model preliminaries and
constants}\label{calculating-model-preliminaries-and-constants}

Once the data are properly formatted, we must calculate the basis
functions and other constants needed to run each model. However, the
basis function calculation and calculation of constants is model
dependent. Here we will describe extensively the process for calculating
the basis function for the model that uses a kernel basis function for
period effects, along with the additive combination of constrained
generalized additive models with natural splines for the kselect
simulation (CSL-K). The code for each of the model variants is provided
based on the same model identifier as in Table 1 in the manuscript. The
model specific basis functions are denoted in the file names.

\subsubsection{K-select simulation model fitting
(CSL-K)}\label{k-select-simulation-model-fitting-csl-k}

First we source the data generating function and the data formatting R
code. Note that the uncommented R scripts are for the R-select
simulation data generating
(S4\_01\_age\_period\_survival\_generate\_data\_kselect.R) and data
formatting (S4\_02\_format\_data\_kselect.R).

\begin{Shaded}
\begin{Highlighting}[]
\KeywordTok{source}\NormalTok{(}\StringTok{"../S4_01_age_period_survival_generate_data_kselect.R"}\NormalTok{)}
\CommentTok{# source("../S4_01_age_period_survival_generate_data_rselect.R")}

\NormalTok{########################################################}
\NormalTok{### Generate the data and format to run in models}
\NormalTok{########################################################}

\KeywordTok{source}\NormalTok{(}\StringTok{"../S4_02_format_data_kselect.R"}\NormalTok{)}
\CommentTok{# source("../S4_02_format_data_rselect.R")}
\end{Highlighting}
\end{Shaded}

Then we provide a nimble function that can be used to calculate the
kernel convolution process. We constrained the kernel convolutions to
sum to 1, so the kernel is calculated by dividing by it's sum. The
parameters of this function include the distance matrix (Z), the square
root of the smoothing parameter, or precision, for the kernel (stauk), a
constant (nconst) that is simply the inverse of the square root of
\(2 \times \pi\), the precision of the kernel (tauk), the number of
knots (nknots), and the white noise process (alphau).

\begin{Shaded}
\begin{Highlighting}[]
\NormalTok{kernel_conv <-}\StringTok{ }\KeywordTok{nimbleFunction}\NormalTok{(}
  \DataTypeTok{run =} \ControlFlowTok{function}\NormalTok{(}\DataTypeTok{nT =} \KeywordTok{double}\NormalTok{(}\DecValTok{0}\NormalTok{),}
                 \DataTypeTok{Z =} \KeywordTok{double}\NormalTok{(}\DecValTok{2}\NormalTok{),}
                 \DataTypeTok{stauk =} \KeywordTok{double}\NormalTok{(}\DecValTok{0}\NormalTok{),}
                 \DataTypeTok{nconst =} \KeywordTok{double}\NormalTok{(}\DecValTok{0}\NormalTok{),}
                 \DataTypeTok{tauk =} \KeywordTok{double}\NormalTok{(}\DecValTok{0}\NormalTok{),}
                 \DataTypeTok{nknots =} \KeywordTok{double}\NormalTok{(}\DecValTok{0}\NormalTok{),}
                 \DataTypeTok{alphau =} \KeywordTok{double}\NormalTok{(}\DecValTok{1}\NormalTok{)}
\NormalTok{  )\{}
\NormalTok{    temp <-}\StringTok{ }\KeywordTok{nimMatrix}\NormalTok{(}\DataTypeTok{value =} \DecValTok{0}\NormalTok{,}\DataTypeTok{nrow =}\NormalTok{ nT, }\DataTypeTok{ncol =}\NormalTok{ nknots)}
\NormalTok{    temp1 <-}\StringTok{ }\KeywordTok{nimMatrix}\NormalTok{(}\DataTypeTok{value =} \DecValTok{0}\NormalTok{,}\DataTypeTok{nrow =}\NormalTok{ nT, }\DataTypeTok{ncol =}\NormalTok{ nknots)}
\NormalTok{    temp2 <-}\StringTok{ }\KeywordTok{nimNumeric}\NormalTok{(nknots)}
\NormalTok{    KA <-}\StringTok{ }\KeywordTok{nimNumeric}\NormalTok{(nT)}

    \ControlFlowTok{for}\NormalTok{ (i }\ControlFlowTok{in} \DecValTok{1}\OperatorTok{:}\NormalTok{nT) \{}
      \ControlFlowTok{for}\NormalTok{ (j }\ControlFlowTok{in} \DecValTok{1}\OperatorTok{:}\NormalTok{nknots) \{}
\NormalTok{        temp1[i, j] <-}\StringTok{ }\NormalTok{stauk }\OperatorTok{*}\StringTok{ }\NormalTok{nconst }\OperatorTok{*}\StringTok{ }\KeywordTok{exp}\NormalTok{(}\OperatorTok{-}\FloatTok{0.5} \OperatorTok{*}\StringTok{ }\NormalTok{Z[i, j]}\OperatorTok{^}\DecValTok{2} \OperatorTok{*}\StringTok{ }\NormalTok{tauk)}
\NormalTok{      \}}
\NormalTok{    \}}
    \ControlFlowTok{for}\NormalTok{ (j }\ControlFlowTok{in} \DecValTok{1}\OperatorTok{:}\NormalTok{nknots) \{}
\NormalTok{      temp2[j] <-}\StringTok{ }\KeywordTok{sum}\NormalTok{(temp1[}\DecValTok{1}\OperatorTok{:}\NormalTok{nT, j])}
\NormalTok{    \}}
    \ControlFlowTok{for}\NormalTok{ (i }\ControlFlowTok{in} \DecValTok{1}\OperatorTok{:}\NormalTok{nT) \{}
      \ControlFlowTok{for}\NormalTok{ (j }\ControlFlowTok{in} \DecValTok{1}\OperatorTok{:}\NormalTok{nknots) \{}
\NormalTok{        temp[i,j] <-}\StringTok{ }\NormalTok{(temp1[i, j] }\OperatorTok{/}\StringTok{ }\NormalTok{temp2[j]) }\OperatorTok{*}\StringTok{ }\NormalTok{alphau[j]}
\NormalTok{      \}}
\NormalTok{      KA[i] <-}\StringTok{ }\KeywordTok{sum}\NormalTok{(temp[i, }\DecValTok{1}\OperatorTok{:}\NormalTok{nknots])}
\NormalTok{    \}}

\NormalTok{    muKA <-}\StringTok{ }\KeywordTok{mean}\NormalTok{(KA[}\DecValTok{1}\OperatorTok{:}\NormalTok{nT])}
\NormalTok{    KA[}\DecValTok{1}\OperatorTok{:}\NormalTok{nT] <-}\StringTok{ }\NormalTok{KA[}\DecValTok{1}\OperatorTok{:}\NormalTok{nT] }\OperatorTok{-}\StringTok{ }\NormalTok{muKA}

    \KeywordTok{returnType}\NormalTok{(}\KeywordTok{double}\NormalTok{(}\DecValTok{1}\NormalTok{))}
    \KeywordTok{return}\NormalTok{(KA[}\DecValTok{1}\OperatorTok{:}\NormalTok{nT])}
\NormalTok{  \})}

\NormalTok{ckernel_conv <-}\StringTok{ }\KeywordTok{compileNimble}\NormalTok{(kernel_conv)}
\end{Highlighting}
\end{Shaded}

Next, we provide the function that is used to calculate the basis
function for the constrained generalized additive model (cgam) that is
taken from Meyer et al (2008) and the bcgam R package.

\begin{Shaded}
\begin{Highlighting}[]
\NormalTok{convex <-}\StringTok{ }\ControlFlowTok{function}\NormalTok{(x, t, }\DataTypeTok{pred.new=}\OtherTok{TRUE}\NormalTok{) \{}
\NormalTok{  n =}\StringTok{ }\KeywordTok{length}\NormalTok{(x)}
\NormalTok{  k =}\StringTok{ }\KeywordTok{length}\NormalTok{(t)}\OperatorTok{-}\DecValTok{2}
\NormalTok{  m =}\StringTok{ }\NormalTok{k }\OperatorTok{+}\StringTok{ }\DecValTok{2}
\NormalTok{  sigma =}\StringTok{ }\KeywordTok{matrix}\NormalTok{(}\DecValTok{1}\OperatorTok{:}\NormalTok{m}\OperatorTok{*}\NormalTok{n, }\DataTypeTok{nrow =}\NormalTok{ m, }\DataTypeTok{ncol =}\NormalTok{ n)}
  \ControlFlowTok{for}\NormalTok{(j }\ControlFlowTok{in} \DecValTok{1}\OperatorTok{:}\NormalTok{(k}\OperatorTok{-}\DecValTok{1}\NormalTok{))\{}
\NormalTok{    i1=x}\OperatorTok{<=}\NormalTok{t[j]}
\NormalTok{    sigma[j,i1] =}\StringTok{ }\DecValTok{0}
\NormalTok{    i2=x}\OperatorTok{>}\NormalTok{t[j]}\OperatorTok{&}\NormalTok{x}\OperatorTok{<=}\NormalTok{t[j}\OperatorTok{+}\DecValTok{1}\NormalTok{]}
\NormalTok{    sigma[j,i2] =}\StringTok{ }\NormalTok{(x[i2]}\OperatorTok{-}\NormalTok{t[j])}\OperatorTok{^}\DecValTok{3} \OperatorTok{/}\StringTok{ }\NormalTok{(t[j}\OperatorTok{+}\DecValTok{2}\NormalTok{]}\OperatorTok{-}\NormalTok{t[j]) }\OperatorTok{/}\StringTok{ }\NormalTok{(t[j}\OperatorTok{+}\DecValTok{1}\NormalTok{]}\OperatorTok{-}\NormalTok{t[j])}\OperatorTok{/}\DecValTok{3}
\NormalTok{    i3=x}\OperatorTok{>}\NormalTok{t[j}\OperatorTok{+}\DecValTok{1}\NormalTok{]}\OperatorTok{&}\NormalTok{x}\OperatorTok{<=}\NormalTok{t[j}\OperatorTok{+}\DecValTok{2}\NormalTok{]}
\NormalTok{    sigma[j,i3] =}\StringTok{ }\NormalTok{x[i3]}\OperatorTok{-}\NormalTok{t[j}\OperatorTok{+}\DecValTok{1}\NormalTok{]}\OperatorTok{-}\NormalTok{(x[i3]}\OperatorTok{-}\NormalTok{t[j}\OperatorTok{+}\DecValTok{2}\NormalTok{])}\OperatorTok{^}\DecValTok{3}\OperatorTok{/}\NormalTok{(t[j}\OperatorTok{+}\DecValTok{2}\NormalTok{]}\OperatorTok{-}\NormalTok{t[j])}\OperatorTok{/}\NormalTok{(t[j}\OperatorTok{+}\DecValTok{2}\NormalTok{]}\OperatorTok{-}\NormalTok{t[j}\OperatorTok{+}\DecValTok{1}\NormalTok{])}\OperatorTok{/}\DecValTok{3}\OperatorTok{+}\NormalTok{(t[j}\OperatorTok{+}\DecValTok{1}\NormalTok{]}\OperatorTok{-}\NormalTok{t[j])}\OperatorTok{^}\DecValTok{2}\OperatorTok{/}\DecValTok{3}\OperatorTok{/}\NormalTok{(t[j}\OperatorTok{+}\DecValTok{2}\NormalTok{]}\OperatorTok{-}\NormalTok{t[j])}\OperatorTok{-}\NormalTok{(t[j}\OperatorTok{+}\DecValTok{2}\NormalTok{]}\OperatorTok{-}\NormalTok{t[j}\OperatorTok{+}\DecValTok{1}\NormalTok{])}\OperatorTok{^}\DecValTok{2}\OperatorTok{/}\DecValTok{3}\OperatorTok{/}\NormalTok{(t[j}\OperatorTok{+}\DecValTok{2}\NormalTok{]}\OperatorTok{-}\NormalTok{t[j])}
\NormalTok{    i4=x}\OperatorTok{>}\NormalTok{t[j}\OperatorTok{+}\DecValTok{2}\NormalTok{]}
\NormalTok{    sigma[j,i4]=(x[i4]}\OperatorTok{-}\NormalTok{t[j}\OperatorTok{+}\DecValTok{1}\NormalTok{])}\OperatorTok{+}\NormalTok{(t[j}\OperatorTok{+}\DecValTok{1}\NormalTok{]}\OperatorTok{-}\NormalTok{t[j])}\OperatorTok{^}\DecValTok{2}\OperatorTok{/}\DecValTok{3}\OperatorTok{/}\NormalTok{(t[j}\OperatorTok{+}\DecValTok{2}\NormalTok{]}\OperatorTok{-}\NormalTok{t[j])}\OperatorTok{-}\NormalTok{(t[j}\OperatorTok{+}\DecValTok{2}\NormalTok{]}\OperatorTok{-}\NormalTok{t[j}\OperatorTok{+}\DecValTok{1}\NormalTok{])}\OperatorTok{^}\DecValTok{2}\OperatorTok{/}\DecValTok{3}\OperatorTok{/}\NormalTok{(t[j}\OperatorTok{+}\DecValTok{2}\NormalTok{]}\OperatorTok{-}\NormalTok{t[j])}
\NormalTok{  \}}
\NormalTok{  i1=x}\OperatorTok{<=}\NormalTok{t[k]}
\NormalTok{  sigma[k,i1] =}\StringTok{ }\DecValTok{0}
\NormalTok{  i2=x}\OperatorTok{>}\NormalTok{t[k]}\OperatorTok{&}\NormalTok{x}\OperatorTok{<=}\NormalTok{t[k}\OperatorTok{+}\DecValTok{1}\NormalTok{]}
\NormalTok{  sigma[k,i2] =}\StringTok{ }\NormalTok{(x[i2]}\OperatorTok{-}\NormalTok{t[k])}\OperatorTok{^}\DecValTok{3} \OperatorTok{/}\StringTok{ }\NormalTok{(t[k}\OperatorTok{+}\DecValTok{2}\NormalTok{]}\OperatorTok{-}\NormalTok{t[k]) }\OperatorTok{/}\StringTok{ }\NormalTok{(t[k}\OperatorTok{+}\DecValTok{1}\NormalTok{]}\OperatorTok{-}\NormalTok{t[k])}\OperatorTok{/}\DecValTok{3}
\NormalTok{  i3=x}\OperatorTok{>}\NormalTok{t[k}\OperatorTok{+}\DecValTok{1}\NormalTok{]}
\NormalTok{  sigma[k,i3] =}\StringTok{ }\NormalTok{x[i3]}\OperatorTok{-}\NormalTok{t[k}\OperatorTok{+}\DecValTok{1}\NormalTok{]}\OperatorTok{-}\NormalTok{(x[i3]}\OperatorTok{-}\NormalTok{t[k}\OperatorTok{+}\DecValTok{2}\NormalTok{])}\OperatorTok{^}\DecValTok{3}\OperatorTok{/}\NormalTok{(t[k}\OperatorTok{+}\DecValTok{2}\NormalTok{]}\OperatorTok{-}\NormalTok{t[k])}\OperatorTok{/}\NormalTok{(t[k}\OperatorTok{+}\DecValTok{2}\NormalTok{]}\OperatorTok{-}\NormalTok{t[k}\OperatorTok{+}\DecValTok{1}\NormalTok{])}\OperatorTok{/}\DecValTok{3}\OperatorTok{+}\NormalTok{(t[k}\OperatorTok{+}\DecValTok{1}\NormalTok{]}\OperatorTok{-}\NormalTok{t[k])}\OperatorTok{^}\DecValTok{2}\OperatorTok{/}\DecValTok{3}\OperatorTok{/}\NormalTok{(t[k}\OperatorTok{+}\DecValTok{2}\NormalTok{]}\OperatorTok{-}\NormalTok{t[k])}\OperatorTok{-}\NormalTok{(t[k}\OperatorTok{+}\DecValTok{2}\NormalTok{]}\OperatorTok{-}\NormalTok{t[k}\OperatorTok{+}\DecValTok{1}\NormalTok{])}\OperatorTok{^}\DecValTok{2}\OperatorTok{/}\DecValTok{3}\OperatorTok{/}\NormalTok{(t[k}\OperatorTok{+}\DecValTok{2}\NormalTok{]}\OperatorTok{-}\NormalTok{t[k])}
\NormalTok{  i1=x}\OperatorTok{<=}\NormalTok{t[}\DecValTok{2}\NormalTok{]}
\NormalTok{  sigma[k}\OperatorTok{+}\DecValTok{1}\NormalTok{,i1]=x[i1]}\OperatorTok{-}\NormalTok{t[}\DecValTok{1}\NormalTok{]}\OperatorTok{+}\NormalTok{(t[}\DecValTok{2}\NormalTok{]}\OperatorTok{-}\NormalTok{x[i1])}\OperatorTok{^}\DecValTok{3}\OperatorTok{/}\NormalTok{(t[}\DecValTok{2}\NormalTok{]}\OperatorTok{-}\NormalTok{t[}\DecValTok{1}\NormalTok{])}\OperatorTok{^}\DecValTok{2}\OperatorTok{/}\DecValTok{3}
\NormalTok{  i2=x}\OperatorTok{>}\NormalTok{t[}\DecValTok{2}\NormalTok{]}
\NormalTok{  sigma[k}\OperatorTok{+}\DecValTok{1}\NormalTok{,i2]=x[i2]}\OperatorTok{-}\NormalTok{t[}\DecValTok{1}\NormalTok{]}
\NormalTok{  i1=x}\OperatorTok{<=}\NormalTok{t[k}\OperatorTok{+}\DecValTok{1}\NormalTok{]}
\NormalTok{  sigma[k}\OperatorTok{+}\DecValTok{2}\NormalTok{,i1]=}\DecValTok{0}
\NormalTok{  i2=x}\OperatorTok{>}\NormalTok{t[k}\OperatorTok{+}\DecValTok{1}\NormalTok{]}
\NormalTok{  sigma[k}\OperatorTok{+}\DecValTok{2}\NormalTok{,i2]=(x[i2]}\OperatorTok{-}\NormalTok{t[k}\OperatorTok{+}\DecValTok{1}\NormalTok{])}\OperatorTok{^}\DecValTok{3}\OperatorTok{/}\NormalTok{(t[k}\OperatorTok{+}\DecValTok{2}\NormalTok{]}\OperatorTok{-}\NormalTok{t[k}\OperatorTok{+}\DecValTok{1}\NormalTok{])}\OperatorTok{^}\DecValTok{2}\OperatorTok{/}\DecValTok{3}
  
\NormalTok{  v1=}\DecValTok{1}\OperatorTok{:}\NormalTok{n}\OperatorTok{*}\DecValTok{0}\OperatorTok{+}\DecValTok{1}
\NormalTok{  v2=x}
\NormalTok{  x.mat=}\KeywordTok{cbind}\NormalTok{(v1,v2)}
  
  \ControlFlowTok{if}\NormalTok{(pred.new}\OperatorTok{==}\OtherTok{TRUE}\NormalTok{)\{}
    \KeywordTok{list}\NormalTok{(}\DataTypeTok{sigma=}\NormalTok{sigma,}\DataTypeTok{x.mat=}\NormalTok{x.mat)\}}
  
  \ControlFlowTok{else}\NormalTok{\{}
    \ControlFlowTok{if}\NormalTok{(pred.new}\OperatorTok{==}\OtherTok{FALSE}\NormalTok{)\{}
\NormalTok{      coef=}\KeywordTok{solve}\NormalTok{(}\KeywordTok{t}\NormalTok{(x.mat)}\OperatorTok\NormalTok{x.mat)}\OperatorTok\KeywordTok{t}\NormalTok{(x.mat)}\OperatorTok\KeywordTok{t}\NormalTok{(sigma)}
      \KeywordTok{list}\NormalTok{(}\DataTypeTok{sigma=}\NormalTok{sigma, }\DataTypeTok{x.mat=}\NormalTok{x.mat, }\DataTypeTok{center.vector=}\NormalTok{coef)\}}
\NormalTok{  \}}
  
\NormalTok{\}}
\end{Highlighting}
\end{Shaded}

For the cgam basis, we specify 6 knots using quantiles of the right\_age
vector. Using the above convex function, we calculate the basis function
for the age effects so that they are neither decreasing nor increasing.
We re-scaled the basis function by dividing the function values by the
maximum value to aid in convergence.

\begin{Shaded}
\begin{Highlighting}[]
\NormalTok{quant_age <-}\StringTok{ }\NormalTok{.}\DecValTok{25}

\NormalTok{knots_age <-}\StringTok{ }\KeywordTok{c}\NormalTok{(}\DecValTok{1}\NormalTok{,}
              \KeywordTok{round}\NormalTok{(}\KeywordTok{quantile}\NormalTok{(df.fit}\OperatorTok{$}\NormalTok{right_age }\OperatorTok{-}\StringTok{ }\DecValTok{1}\NormalTok{,}
                             \KeywordTok{c}\NormalTok{(}\KeywordTok{seq}\NormalTok{(quant_age, .}\DecValTok{99}\NormalTok{, }\DataTypeTok{by =}\NormalTok{ quant_age), .}\DecValTok{99}\NormalTok{))))}
\NormalTok{knots_age <-}\StringTok{ }\KeywordTok{unique}\NormalTok{(knots_age)}

\NormalTok{delta_i <-}\StringTok{ }\KeywordTok{convex}\NormalTok{(}\DecValTok{1}\OperatorTok{:}\NormalTok{nT_age,knots_age, }\DataTypeTok{pred.new=}\OtherTok{FALSE}\NormalTok{)}
\NormalTok{delta <-}\StringTok{ }\KeywordTok{t}\NormalTok{(}\KeywordTok{rbind}\NormalTok{(delta_i}\OperatorTok{$}\NormalTok{sigma }\OperatorTok{-}\StringTok{ }\KeywordTok{t}\NormalTok{(delta_i}\OperatorTok{$}\NormalTok{x.mat }\OperatorTok\StringTok{ }\NormalTok{delta_i}\OperatorTok{$}\NormalTok{center.vector)))}
\NormalTok{delta <-}\StringTok{ }\NormalTok{delta }\OperatorTok{/}\StringTok{ }\KeywordTok{max}\NormalTok{(delta)}

\NormalTok{Z_age_cgam <-}\StringTok{ }\NormalTok{delta}
\NormalTok{nknots_age_cgam <-}\StringTok{ }\KeywordTok{dim}\NormalTok{(Z_age_cgam)[}\DecValTok{2}\NormalTok{]}
\end{Highlighting}
\end{Shaded}

Next, we calculate the basis function for the natural spline for the age
effects. First we set knots based on the quantiles of the right\_age
vector such that there are at least 20 knots. Then we calculate the
natural spline basis from the splines R package function. Then we use QR
factorization to obtain the null space of the spline basis, so that we
can impose the sum to zero constraint for the age effects.

\begin{Shaded}
\begin{Highlighting}[]
\NormalTok{### Knots}
\NormalTok{quant_age <-}\StringTok{ }\NormalTok{.}\DecValTok{05}
\NormalTok{knots_age_spline <-}\StringTok{ }\KeywordTok{c}\NormalTok{(}\DecValTok{1}\NormalTok{,}
                     \KeywordTok{round}\NormalTok{(}\KeywordTok{quantile}\NormalTok{(df.fit}\OperatorTok{$}\NormalTok{right_age}\OperatorTok{-}\DecValTok{1}\NormalTok{, }
                                   \KeywordTok{c}\NormalTok{(}\KeywordTok{seq}\NormalTok{(quant_age,.}\DecValTok{99}\NormalTok{, }\DataTypeTok{by=}\NormalTok{quant_age),}
\NormalTok{                                   .}\DecValTok{99}\NormalTok{))))}
\NormalTok{knots_age_spline <-}\StringTok{ }\KeywordTok{unique}\NormalTok{(knots_age_spline)}
\NormalTok{nknots_age_spline <-}\StringTok{ }\KeywordTok{length}\NormalTok{(knots_age_spline)}

\NormalTok{### Basis for age hazard}
\NormalTok{splinebasis <-}\StringTok{ }\KeywordTok{ns}\NormalTok{(}\DecValTok{1}\OperatorTok{:}\NormalTok{nT_age, }\DataTypeTok{knots =}\NormalTok{ knots_age_spline)}

\NormalTok{### A constraint matrix so period_effects == 0}
\NormalTok{constr_sumzero <-}\StringTok{ }\KeywordTok{matrix}\NormalTok{(}\DecValTok{1}\NormalTok{, }\DecValTok{1}\NormalTok{, }\KeywordTok{nrow}\NormalTok{(splinebasis)) }\OperatorTok\StringTok{ }\NormalTok{splinebasis}

\NormalTok{### QR factorization for null space of constraint}
\NormalTok{qrc <-}\StringTok{ }\KeywordTok{qr}\NormalTok{(}\KeywordTok{t}\NormalTok{(constr_sumzero))}
\NormalTok{Z <-}\StringTok{ }\KeywordTok{qr.Q}\NormalTok{(qrc,}
          \DataTypeTok{complete =} \OtherTok{TRUE}\NormalTok{)[, (}\KeywordTok{nrow}\NormalTok{(constr_sumzero) }\OperatorTok{+}\StringTok{ }\DecValTok{1}\NormalTok{)}\OperatorTok{:}\KeywordTok{ncol}\NormalTok{(constr_sumzero)]}
\NormalTok{Z_age_spline <-}\StringTok{ }\NormalTok{splinebasis }\OperatorTok\StringTok{ }\NormalTok{Z}
\NormalTok{nknots_age_spline <-}\StringTok{ }\KeywordTok{dim}\NormalTok{(Z_age_spline)[}\DecValTok{2}\NormalTok{]}
\end{Highlighting}
\end{Shaded}

We plot the basis functions.

\begin{Shaded}
\begin{Highlighting}[]
\KeywordTok{pdf}\NormalTok{(}\StringTok{"figures/basis_function_age.pdf"}\NormalTok{)}
\KeywordTok{plot}\NormalTok{(}\DecValTok{1}\OperatorTok{:}\NormalTok{nT_age,}
\NormalTok{     Z_age_spline[, }\DecValTok{1}\NormalTok{],}
     \DataTypeTok{ylim =} \KeywordTok{c}\NormalTok{(}\OperatorTok{-}\DecValTok{1}\NormalTok{, }\DecValTok{1}\NormalTok{),}
     \DataTypeTok{type =} \StringTok{"l"}\NormalTok{,}
     \DataTypeTok{main =} \StringTok{"Basis Function Age Effect Spline"}\NormalTok{)}
\ControlFlowTok{for}\NormalTok{ (i }\ControlFlowTok{in} \DecValTok{2}\OperatorTok{:}\NormalTok{nknots_age_spline) \{}
  \KeywordTok{lines}\NormalTok{(}\DecValTok{1}\OperatorTok{:}\NormalTok{nT_age, Z_age_spline[, i])}
\NormalTok{\}}
\KeywordTok{plot}\NormalTok{(}\DecValTok{1}\OperatorTok{:}\NormalTok{nT_age,}
\NormalTok{     Z_age_cgam[, }\DecValTok{1}\NormalTok{],}
     \DataTypeTok{ylim =} \KeywordTok{c}\NormalTok{(}\OperatorTok{-}\DecValTok{1}\NormalTok{, }\DecValTok{1}\NormalTok{),}
     \DataTypeTok{type =} \StringTok{"l"}\NormalTok{,}
     \DataTypeTok{main =} \StringTok{"Basis Function Age Effect CGAM"}\NormalTok{)}
\ControlFlowTok{for}\NormalTok{ (i }\ControlFlowTok{in} \DecValTok{2}\OperatorTok{:}\NormalTok{nknots_age_cgam) \{}
  \KeywordTok{lines}\NormalTok{(}\DecValTok{1}\OperatorTok{:}\NormalTok{nT_age, Z_age_cgam[, i])}
\NormalTok{\}}
\KeywordTok{dev.off}\NormalTok{()}
\end{Highlighting}
\end{Shaded}

The we calculate the basis expansion for the period effects for the
kernel convolution process. First we set the knots, where we used dense
equal space knots on each period interval. Then the basis function is
just the absolute value of the distance between each period and the
neighboring periods.

\begin{Shaded}
\begin{Highlighting}[]
\NormalTok{intvl_period <-}\StringTok{ }\DecValTok{1}
\NormalTok{knots_period <-}\StringTok{ }\KeywordTok{seq}\NormalTok{(}\DecValTok{1}\NormalTok{, nT_period, }\DataTypeTok{by =}\NormalTok{ intvl_period)}
\NormalTok{knots_period <-}\StringTok{ }\KeywordTok{unique}\NormalTok{(knots_period)}
\NormalTok{nknots_period <-}\StringTok{ }\KeywordTok{length}\NormalTok{(knots_period)}

\NormalTok{Z_period <-}\StringTok{ }\KeywordTok{matrix}\NormalTok{(}\DecValTok{0}\NormalTok{, nT_period, nknots_period)}
\ControlFlowTok{for}\NormalTok{ (i }\ControlFlowTok{in} \DecValTok{1}\OperatorTok{:}\KeywordTok{nrow}\NormalTok{(Z_period)) \{}
  \ControlFlowTok{for}\NormalTok{ (j }\ControlFlowTok{in} \DecValTok{1}\OperatorTok{:}\NormalTok{nknots_period) \{}
\NormalTok{    Z_period[i, j] <-}\StringTok{ }\KeywordTok{abs}\NormalTok{(i }\OperatorTok{-}\StringTok{ }\NormalTok{knots_period[j])}
\NormalTok{  \}}
\NormalTok{\}}
\end{Highlighting}
\end{Shaded}

Lastly, we specify the number of Markov chain Monte-Carlo iterations
(reps), the burn-in (bin), the number of MCMC chains (n\_chains), and
the thinning interval (n\_thin).

\begin{Shaded}
\begin{Highlighting}[]
\NormalTok{reps <-}\StringTok{ }\DecValTok{50000}
\NormalTok{bin <-}\StringTok{ }\NormalTok{reps }\OperatorTok{*}\StringTok{ }\NormalTok{.}\DecValTok{5}
\NormalTok{n_chains <-}\StringTok{ }\DecValTok{3}
\NormalTok{n_thin <-}\StringTok{ }\DecValTok{1}
\end{Highlighting}
\end{Shaded}

The full preliminary function to run all of these basis function
calculations is provided in
S4\_03\_prelim\_constants\_kselect\_CSL\_K.R.

\subsection{Run the model using the NIMBLE
package}\label{run-the-model-using-the-nimble-package}

We provide a function that can calculate the probability of a mortality
for each individual across the intervals (age and period) that the
individual is collared and alive in the study. The function is based on
indexing over the age effects (age\_effect), and uses the age2date
vector to calculate the proper index for the period intervals
(period\_effect). The intercept (beta0) is added to the log hazard. The
summation from left to right provides the approximation of the
integration of the log hazard for obtaining the cumulative probability
of survival during each age interval and each period interval.

\begin{Shaded}
\begin{Highlighting}[]
\NormalTok{state_transition <-}\StringTok{ }\KeywordTok{nimbleFunction}\NormalTok{(}
  \DataTypeTok{run =} \ControlFlowTok{function}\NormalTok{(}\DataTypeTok{records =} \KeywordTok{double}\NormalTok{(}\DecValTok{0}\NormalTok{),}
                 \DataTypeTok{left =} \KeywordTok{double}\NormalTok{(}\DecValTok{1}\NormalTok{),}
                 \DataTypeTok{right =} \KeywordTok{double}\NormalTok{(}\DecValTok{1}\NormalTok{),}
                 \DataTypeTok{beta0 =} \KeywordTok{double}\NormalTok{(}\DecValTok{0}\NormalTok{)}
                 \DataTypeTok{age_effect =} \KeywordTok{double}\NormalTok{(}\DecValTok{1}\NormalTok{),}
                 \DataTypeTok{period_effect =} \KeywordTok{double}\NormalTok{(}\DecValTok{1}\NormalTok{),}
                 \DataTypeTok{age2date =} \KeywordTok{double}\NormalTok{(}\DecValTok{1}\NormalTok{),}
                 \DataTypeTok{nT_age =} \KeywordTok{double}\NormalTok{(}\DecValTok{0}\NormalTok{)}
\NormalTok{  )\{}

\NormalTok{    SLR <-}\StringTok{ }\KeywordTok{nimNumeric}\NormalTok{(records)}
\NormalTok{    UCH <-}\KeywordTok{nimMatrix}\NormalTok{(}\DataTypeTok{value =} \DecValTok{0}\NormalTok{,}\DataTypeTok{nrow =}\NormalTok{ records, }\DataTypeTok{ncol =}\NormalTok{ nT_age)}
    \ControlFlowTok{for}\NormalTok{ (j }\ControlFlowTok{in} \DecValTok{1}\OperatorTok{:}\NormalTok{records) \{}
      \ControlFlowTok{for}\NormalTok{ (k }\ControlFlowTok{in}\NormalTok{ left[j]}\OperatorTok{:}\NormalTok{(right[j] }\OperatorTok{-}\StringTok{ }\DecValTok{1}\NormalTok{)) \{}
\NormalTok{        UCH[j, k] <-}\StringTok{ }\KeywordTok{exp}\NormalTok{(beta0 }\OperatorTok{+}
\StringTok{                         }\NormalTok{age_effect[k] }\OperatorTok{+}
\StringTok{                         }\NormalTok{period_effect[k }\OperatorTok{-}\StringTok{ }\NormalTok{age2date[j]])}
\NormalTok{      \}}
\NormalTok{      SLR[j] <-}\StringTok{ }\KeywordTok{exp}\NormalTok{(}\OperatorTok{-}\KeywordTok{sum}\NormalTok{(UCH[j, left[j]}\OperatorTok{:}\NormalTok{(right[j] }\OperatorTok{-}\StringTok{ }\DecValTok{1}\NormalTok{)]))}
\NormalTok{    \}}
    \KeywordTok{returnType}\NormalTok{(}\KeywordTok{double}\NormalTok{(}\DecValTok{1}\NormalTok{))}
    \KeywordTok{return}\NormalTok{(SLR[}\DecValTok{1}\OperatorTok{:}\NormalTok{records])}
\NormalTok{  \})}

\NormalTok{cstate_transition <-}\StringTok{ }\KeywordTok{compileNimble}\NormalTok{(state_transition)}
\end{Highlighting}
\end{Shaded}

Then we specify the model statement using the NIMBLE syntax, which is
based on declarative BUGS code. We specify the prior distribution for
the intercept (beta0) using parameter expansion.

\begin{Shaded}
\begin{Highlighting}[]
\NormalTok{  beta0_temp }\OperatorTok{~}\StringTok{ }\KeywordTok{dnorm}\NormalTok{(}\DecValTok{0}\NormalTok{, .}\DecValTok{01}\NormalTok{)}
\NormalTok{  mix }\OperatorTok{~}\StringTok{ }\KeywordTok{dunif}\NormalTok{(}\OperatorTok{-}\DecValTok{1}\NormalTok{, }\DecValTok{1}\NormalTok{)}
\NormalTok{  beta0 <-}\StringTok{ }\NormalTok{beta0_temp }\OperatorTok{*}\StringTok{ }\NormalTok{mix}
\end{Highlighting}
\end{Shaded}

Then we specify the prior distribution for the effects for the cgam
model, and calculate the age effects. We centered the age effects for
the cgam model by subtracting the mean of the age effects and then used
the sum-to-zero age effects. We could not use QR factorization to obtain
the null space of the basis function when calculating the basis
function, because this mapped the basis functions that prevented the
non-negative constraint for the coefficients to ensure the concave
shape.

\begin{Shaded}
\begin{Highlighting}[]
  \ControlFlowTok{for}\NormalTok{ (k }\ControlFlowTok{in} \DecValTok{1}\OperatorTok{:}\NormalTok{nknots_age_cgam) \{}
\NormalTok{    ln_b_age_cgam[k] }\OperatorTok{~}\StringTok{ }\KeywordTok{dnorm}\NormalTok{(}\DecValTok{0}\NormalTok{, tau_age_cgam)}
\NormalTok{    b_age_cgam[k] <-}\StringTok{ }\KeywordTok{exp}\NormalTok{(ln_b_age_cgam[k])}
\NormalTok{  \}}
\NormalTok{  tau_age_cgam }\OperatorTok{~}\StringTok{ }\KeywordTok{dgamma}\NormalTok{(}\DecValTok{1}\NormalTok{, }\DecValTok{1}\NormalTok{)}
  \ControlFlowTok{for}\NormalTok{ (t }\ControlFlowTok{in} \DecValTok{1}\OperatorTok{:}\NormalTok{nT_age) \{}
\NormalTok{    age_effect_temp[t] <-}\StringTok{ }\KeywordTok{inprod}\NormalTok{(b_age_cgam[}\DecValTok{1}\OperatorTok{:}\NormalTok{nknots_age_cgam],}
\NormalTok{                                 Z_age_cgam[t, }\DecValTok{1}\OperatorTok{:}\NormalTok{nknots_age_cgam])}
\NormalTok{    age_effect_cgam[t] <-}\StringTok{ }\NormalTok{age_effect_temp[t] }\OperatorTok{-}\StringTok{ }\NormalTok{mu_age_cgam}
\NormalTok{  \}}
\NormalTok{  mu_age_cgam <-}\StringTok{ }\KeywordTok{mean}\NormalTok{(age_effect_temp[}\DecValTok{1}\OperatorTok{:}\NormalTok{nT_age])}
\end{Highlighting}
\end{Shaded}

The prior for spline model was specified using the double exponential,
or Laplace prior rather than a normal prior distribution. The QR
factorization that was implemented on the basis function ensures these
effects are summing to zero. We combined the cgam and spline models by
adding them.

\begin{Shaded}
\begin{Highlighting}[]
  \ControlFlowTok{for}\NormalTok{ (k }\ControlFlowTok{in} \DecValTok{1}\OperatorTok{:}\NormalTok{nknots_age_spline) \{}
\NormalTok{    b_age_spline[k] }\OperatorTok{~}\StringTok{ }\KeywordTok{ddexp}\NormalTok{(}\DecValTok{0}\NormalTok{, tau_age_spline)}
\NormalTok{  \}}
\NormalTok{  tau_age_spline }\OperatorTok{~}\StringTok{ }\KeywordTok{dgamma}\NormalTok{(.}\DecValTok{01}\NormalTok{, .}\DecValTok{01}\NormalTok{)}
  \ControlFlowTok{for}\NormalTok{ (t }\ControlFlowTok{in} \DecValTok{1}\OperatorTok{:}\NormalTok{nT_age) \{}
\NormalTok{    age_effect_spline[t] <-}\StringTok{ }\KeywordTok{inprod}\NormalTok{(b_age_spline[}\DecValTok{1}\OperatorTok{:}\NormalTok{nknots_age_spline],}
\NormalTok{                                   Z_age_spline[t, }\DecValTok{1}\OperatorTok{:}\NormalTok{nknots_age_spline])}
\NormalTok{  \}}

\NormalTok{  ########################}
\NormalTok{  ### Age effect combine}
\NormalTok{  ########################}

\NormalTok{  age_effect[}\DecValTok{1}\OperatorTok{:}\NormalTok{nT_age] <-}\StringTok{ }\NormalTok{age_effect_cgam[}\DecValTok{1}\OperatorTok{:}\NormalTok{nT_age] }\OperatorTok{+}\StringTok{ }\NormalTok{age_effect_spline[}\DecValTok{1}\OperatorTok{:}\NormalTok{nT_age]}
\end{Highlighting}
\end{Shaded}

We calculate the period effects using the kernel convolution function
provided earlier (period\_effect). We specify the prior for the kernel
smoother on the log scale (ln\_sk\_period), along with a parameter
expansion mixing parameter to aid in mixing (mix2). We generate the
Gaussian white noise process (alphau\_period) by generating the standard
normal (alpha\_period) and then multiply this by the standard deviation
(sda\_period) that has a truncated normal prior.

\begin{Shaded}
\begin{Highlighting}[]
\NormalTok{  mix2 }\OperatorTok{~}\StringTok{ }\KeywordTok{dunif}\NormalTok{(}\OperatorTok{-}\DecValTok{1}\NormalTok{, }\DecValTok{1}\NormalTok{)}
\NormalTok{  ln_sk_period }\OperatorTok{~}\StringTok{ }\KeywordTok{dnorm}\NormalTok{(}\DecValTok{0}\NormalTok{, }\DataTypeTok{sd =} \DecValTok{1}\NormalTok{)}
\NormalTok{  sdk_period <-}\StringTok{ }\KeywordTok{exp}\NormalTok{(mix2 }\OperatorTok{*}\StringTok{ }\NormalTok{ln_sk_period)}
\NormalTok{  tauk_period <-}\StringTok{ }\DecValTok{1} \OperatorTok{/}\StringTok{ }\NormalTok{sdk_period}\OperatorTok{^}\DecValTok{2}
\NormalTok{  stauk_period <-}\StringTok{ }\KeywordTok{sqrt}\NormalTok{(tauk_period)}
\NormalTok{  sda_period }\OperatorTok{~}\StringTok{ }\KeywordTok{T}\NormalTok{(}\KeywordTok{dnorm}\NormalTok{(}\DecValTok{0}\NormalTok{, }\DataTypeTok{sd =} \DecValTok{1}\NormalTok{),}\DecValTok{0}\NormalTok{, }\OtherTok{Inf}\NormalTok{)}
\NormalTok{  taua_period <-}\StringTok{ }\DecValTok{1} \OperatorTok{/}\StringTok{ }\NormalTok{sda_period}\OperatorTok{^}\DecValTok{2}
  \ControlFlowTok{for}\NormalTok{ (i }\ControlFlowTok{in} \DecValTok{1}\OperatorTok{:}\NormalTok{(nknots_period)) \{}
\NormalTok{    alpha_period[i] }\OperatorTok{~}\StringTok{ }\KeywordTok{dnorm}\NormalTok{(}\DecValTok{0}\NormalTok{, }\DecValTok{1}\NormalTok{)}
\NormalTok{    alphau_period[i] <-}\StringTok{ }\NormalTok{sda_period }\OperatorTok{*}\StringTok{ }\NormalTok{alpha_period[i]}
\NormalTok{  \}}
\NormalTok{  ratioinf_period <-}\StringTok{ }\NormalTok{sdk_period }\OperatorTok{/}\StringTok{ }\NormalTok{sda_period }\CommentTok{#ratio of variability/smoothing}

\NormalTok{  period_effect[}\DecValTok{1}\OperatorTok{:}\NormalTok{nT_period] <-}\StringTok{ }\KeywordTok{kernel.conv}\NormalTok{(}
    \DataTypeTok{nT =}\NormalTok{ nT_period,}
    \DataTypeTok{Z =}\NormalTok{ Z_period[}\DecValTok{1}\OperatorTok{:}\NormalTok{nT_period, }\DecValTok{1}\OperatorTok{:}\NormalTok{nknots_period],}
    \DataTypeTok{stauk =}\NormalTok{ stauk_period,}
    \DataTypeTok{nconst =}\NormalTok{ nconst,}
    \DataTypeTok{tauk =}\NormalTok{ tauk_period,}
    \DataTypeTok{nknots =}\NormalTok{ nknots_period,}
    \DataTypeTok{alphau =}\NormalTok{ alphau_period[}\DecValTok{1}\OperatorTok{:}\NormalTok{nknots_period]}
\NormalTok{  )}
\end{Highlighting}
\end{Shaded}

Then we compute the cumulative probability of mortality within each
period using the state\_transition function. The likelihood is just a
Bernoulli distribution for each individual given whether they are alive
and in the study (censor=1), or whether they have died (censor=0).

\begin{Shaded}
\begin{Highlighting}[]
\NormalTok{  SLR[}\DecValTok{1}\OperatorTok{:}\NormalTok{records] <-}\StringTok{ }\KeywordTok{state_transition}\NormalTok{(}\DataTypeTok{records =}\NormalTok{ records,}
                                   \DataTypeTok{left =}\NormalTok{ left_age[}\DecValTok{1}\OperatorTok{:}\NormalTok{records],}
                                   \DataTypeTok{right =}\NormalTok{ right_age[}\DecValTok{1}\OperatorTok{:}\NormalTok{records],}
                                   \DataTypeTok{nT_age =}\NormalTok{ nT_age,}
                                   \DataTypeTok{age_effect =}\NormalTok{ age_effect[}\DecValTok{1}\OperatorTok{:}\NormalTok{nT_age],}
                                   \DataTypeTok{period_effect =}\NormalTok{ period_effect[}\DecValTok{1}\OperatorTok{:}\NormalTok{nT_period],}
                                   \DataTypeTok{age2date =}\NormalTok{ age2date[}\DecValTok{1}\OperatorTok{:}\NormalTok{records],}
                                   \DataTypeTok{beta0 =}\NormalTok{ beta0)}

\NormalTok{  ###}
\NormalTok{  ### Likelihood}
\NormalTok{  ###}
  \ControlFlowTok{for}\NormalTok{ (j }\ControlFlowTok{in} \DecValTok{1}\OperatorTok{:}\NormalTok{records) \{}
\NormalTok{    censor[j] }\OperatorTok{~}\StringTok{ }\KeywordTok{dbern}\NormalTok{(SLR[j])}
\NormalTok{  \}}
\end{Highlighting}
\end{Shaded}

We calculate derived parameters consisting of the cumulative probability
of survival over the age intervals and period intervals separately.
These could be calculated differently, where the could be combined by
adding over specific intervals, or a full survival surface could also be
calculated.

\begin{Shaded}
\begin{Highlighting}[]
  \ControlFlowTok{for}\NormalTok{ (t }\ControlFlowTok{in} \DecValTok{1}\OperatorTok{:}\NormalTok{nT_age) \{}
\NormalTok{    llambda_age[t] <-}\StringTok{ }\NormalTok{beta0 }\OperatorTok{+}\StringTok{ }\NormalTok{age_effect[t]}
\NormalTok{    UCH0_age[t] <-}\StringTok{ }\KeywordTok{exp}\NormalTok{(llambda_age[t])}
\NormalTok{    S0_age[t] <-}\StringTok{ }\KeywordTok{exp}\NormalTok{(}\OperatorTok{-}\KeywordTok{sum}\NormalTok{(UCH0_age[}\DecValTok{1}\OperatorTok{:}\NormalTok{t]))}
\NormalTok{  \}}
  \ControlFlowTok{for}\NormalTok{ (t }\ControlFlowTok{in} \DecValTok{1}\OperatorTok{:}\NormalTok{nT_period) \{}
\NormalTok{    llambda_period[t] <-}\StringTok{ }\NormalTok{beta0 }\OperatorTok{+}\StringTok{ }\NormalTok{period_effect[t]}
\NormalTok{    UCH0_period[t] <-}\StringTok{ }\KeywordTok{exp}\NormalTok{(llambda_period[t])}
\NormalTok{    S0_period[t] <-}\StringTok{ }\KeywordTok{exp}\NormalTok{(}\OperatorTok{-}\KeywordTok{sum}\NormalTok{(UCH0_period[}\DecValTok{1}\OperatorTok{:}\NormalTok{t]))}
\NormalTok{  \}}
\end{Highlighting}
\end{Shaded}

We put this altogether in the modelcode statement.

\begin{Shaded}
\begin{Highlighting}[]
\NormalTok{modelcode <-}\StringTok{ }\KeywordTok{nimbleCode}\NormalTok{(\{}
\NormalTok{  ### Prior for intercept}
\NormalTok{  ### using parameter expansion for convergence}
\NormalTok{  beta0_temp }\OperatorTok{~}\StringTok{ }\KeywordTok{dnorm}\NormalTok{(}\DecValTok{0}\NormalTok{, .}\DecValTok{01}\NormalTok{)}
\NormalTok{  mix }\OperatorTok{~}\StringTok{ }\KeywordTok{dunif}\NormalTok{(}\OperatorTok{-}\DecValTok{1}\NormalTok{, }\DecValTok{1}\NormalTok{)}
\NormalTok{  beta0 <-}\StringTok{ }\NormalTok{beta0_temp }\OperatorTok{*}\StringTok{ }\NormalTok{mix}

\NormalTok{  #####################}
\NormalTok{  ### Age effect cgam}
\NormalTok{  #####################}

  \ControlFlowTok{for}\NormalTok{ (k }\ControlFlowTok{in} \DecValTok{1}\OperatorTok{:}\NormalTok{nknots_age_cgam) \{}
\NormalTok{    ln_b_age_cgam[k] }\OperatorTok{~}\StringTok{ }\KeywordTok{dnorm}\NormalTok{(}\DecValTok{0}\NormalTok{, tau_age_cgam)}
\NormalTok{    b_age_cgam[k] <-}\StringTok{ }\KeywordTok{exp}\NormalTok{(ln_b_age_cgam[k])}
\NormalTok{  \}}
\NormalTok{  tau_age_cgam }\OperatorTok{~}\StringTok{ }\KeywordTok{dgamma}\NormalTok{(}\DecValTok{1}\NormalTok{, }\DecValTok{1}\NormalTok{)}
  \ControlFlowTok{for}\NormalTok{ (t }\ControlFlowTok{in} \DecValTok{1}\OperatorTok{:}\NormalTok{nT_age) \{}
\NormalTok{    age_effect_temp[t] <-}\StringTok{ }\KeywordTok{inprod}\NormalTok{(b_age_cgam[}\DecValTok{1}\OperatorTok{:}\NormalTok{nknots_age_cgam],}
\NormalTok{                                 Z_age_cgam[t, }\DecValTok{1}\OperatorTok{:}\NormalTok{nknots_age_cgam])}
\NormalTok{    age_effect_cgam[t] <-}\StringTok{ }\NormalTok{age_effect_temp[t] }\OperatorTok{-}\StringTok{ }\NormalTok{mu_age_cgam}
\NormalTok{  \}}
\NormalTok{  mu_age_cgam <-}\StringTok{ }\KeywordTok{mean}\NormalTok{(age_effect_temp[}\DecValTok{1}\OperatorTok{:}\NormalTok{nT_age])}

\NormalTok{  #####################}
\NormalTok{  ### Age effect spline}
\NormalTok{  #####################}

  \ControlFlowTok{for}\NormalTok{ (k }\ControlFlowTok{in} \DecValTok{1}\OperatorTok{:}\NormalTok{nknots_age_spline) \{}
\NormalTok{    b_age_spline[k] }\OperatorTok{~}\StringTok{ }\KeywordTok{ddexp}\NormalTok{(}\DecValTok{0}\NormalTok{, tau_age_spline)}
\NormalTok{  \}}
\NormalTok{  tau_age_spline }\OperatorTok{~}\StringTok{ }\KeywordTok{dgamma}\NormalTok{(.}\DecValTok{01}\NormalTok{, .}\DecValTok{01}\NormalTok{)}
  \ControlFlowTok{for}\NormalTok{ (t }\ControlFlowTok{in} \DecValTok{1}\OperatorTok{:}\NormalTok{nT_age) \{}
\NormalTok{    age_effect_spline[t] <-}\StringTok{ }\KeywordTok{inprod}\NormalTok{(b_age_spline[}\DecValTok{1}\OperatorTok{:}\NormalTok{nknots_age_spline],}
\NormalTok{                                   Z_age_spline[t, }\DecValTok{1}\OperatorTok{:}\NormalTok{nknots_age_spline])}
\NormalTok{  \}}

\NormalTok{  ########################}
\NormalTok{  ### Age effect combine}
\NormalTok{  ########################}

\NormalTok{  age_effect[}\DecValTok{1}\OperatorTok{:}\NormalTok{nT_age] <-}\StringTok{ }\NormalTok{age_effect_cgam[}\DecValTok{1}\OperatorTok{:}\NormalTok{nT_age] }\OperatorTok{+}\StringTok{ }\NormalTok{age_effect_spline[}\DecValTok{1}\OperatorTok{:}\NormalTok{nT_age]}

\NormalTok{  #####################################}
\NormalTok{  ### Period effect kernel convolution}
\NormalTok{  #####################################}

\NormalTok{  mix2 }\OperatorTok{~}\StringTok{ }\KeywordTok{dunif}\NormalTok{(}\OperatorTok{-}\DecValTok{1}\NormalTok{, }\DecValTok{1}\NormalTok{)}
\NormalTok{  ln_sk_period }\OperatorTok{~}\StringTok{ }\KeywordTok{dnorm}\NormalTok{(}\DecValTok{0}\NormalTok{, }\DataTypeTok{sd =} \DecValTok{1}\NormalTok{)}
\NormalTok{  sdk_period <-}\StringTok{ }\KeywordTok{exp}\NormalTok{(mix2 }\OperatorTok{*}\StringTok{ }\NormalTok{ln_sk_period)}
\NormalTok{  tauk_period <-}\StringTok{ }\DecValTok{1} \OperatorTok{/}\StringTok{ }\NormalTok{sdk_period}\OperatorTok{^}\DecValTok{2}
\NormalTok{  stauk_period <-}\StringTok{ }\KeywordTok{sqrt}\NormalTok{(tauk_period)}
\NormalTok{  sda_period }\OperatorTok{~}\StringTok{ }\KeywordTok{T}\NormalTok{(}\KeywordTok{dnorm}\NormalTok{(}\DecValTok{0}\NormalTok{, }\DataTypeTok{sd =} \DecValTok{1}\NormalTok{),}\DecValTok{0}\NormalTok{, }\OtherTok{Inf}\NormalTok{)}
\NormalTok{  taua_period <-}\StringTok{ }\DecValTok{1} \OperatorTok{/}\StringTok{ }\NormalTok{sda_period}\OperatorTok{^}\DecValTok{2}
  \ControlFlowTok{for}\NormalTok{ (i }\ControlFlowTok{in} \DecValTok{1}\OperatorTok{:}\NormalTok{(nknots_period)) \{}
\NormalTok{    alpha_period[i] }\OperatorTok{~}\StringTok{ }\KeywordTok{dnorm}\NormalTok{(}\DecValTok{0}\NormalTok{, }\DecValTok{1}\NormalTok{)}
\NormalTok{    alphau_period[i] <-}\StringTok{ }\NormalTok{sda_period }\OperatorTok{*}\StringTok{ }\NormalTok{alpha_period[i]}
\NormalTok{  \}}
\NormalTok{  ratioinf_period <-}\StringTok{ }\NormalTok{sdk_period }\OperatorTok{/}\StringTok{ }\NormalTok{sda_period }\CommentTok{#ratio of variability/smoothing}

\NormalTok{  period_effect[}\DecValTok{1}\OperatorTok{:}\NormalTok{nT_period] <-}\StringTok{ }\KeywordTok{kernel.conv}\NormalTok{(}
    \DataTypeTok{nT =}\NormalTok{ nT_period,}
    \DataTypeTok{Z =}\NormalTok{ Z_period[}\DecValTok{1}\OperatorTok{:}\NormalTok{nT_period, }\DecValTok{1}\OperatorTok{:}\NormalTok{nknots_period],}
    \DataTypeTok{stauk =}\NormalTok{ stauk_period,}
    \DataTypeTok{nconst =}\NormalTok{ nconst,}
    \DataTypeTok{tauk =}\NormalTok{ tauk_period,}
    \DataTypeTok{nknots =}\NormalTok{ nknots_period,}
    \DataTypeTok{alphau =}\NormalTok{ alphau_period[}\DecValTok{1}\OperatorTok{:}\NormalTok{nknots_period]}
\NormalTok{  )}

\NormalTok{  ###}
\NormalTok{  ### Computing state transition probability}
\NormalTok{  ###}

\NormalTok{  SLR[}\DecValTok{1}\OperatorTok{:}\NormalTok{records] <-}\StringTok{ }\KeywordTok{state_transition}\NormalTok{(}\DataTypeTok{records =}\NormalTok{ records,}
                                   \DataTypeTok{left =}\NormalTok{ left_age[}\DecValTok{1}\OperatorTok{:}\NormalTok{records],}
                                   \DataTypeTok{right =}\NormalTok{ right_age[}\DecValTok{1}\OperatorTok{:}\NormalTok{records],}
                                   \DataTypeTok{nT_age =}\NormalTok{ nT_age,}
                                   \DataTypeTok{age_effect =}\NormalTok{ age_effect[}\DecValTok{1}\OperatorTok{:}\NormalTok{nT_age],}
                                   \DataTypeTok{period_effect =}\NormalTok{ period_effect[}\DecValTok{1}\OperatorTok{:}\NormalTok{nT_period],}
                                   \DataTypeTok{age2date =}\NormalTok{ age2date[}\DecValTok{1}\OperatorTok{:}\NormalTok{records],}
                                   \DataTypeTok{beta0 =}\NormalTok{ beta0)}

\NormalTok{  ###}
\NormalTok{  ### Likelihood}
\NormalTok{  ###}
  \ControlFlowTok{for}\NormalTok{ (j }\ControlFlowTok{in} \DecValTok{1}\OperatorTok{:}\NormalTok{records) \{}
\NormalTok{    censor[j] }\OperatorTok{~}\StringTok{ }\KeywordTok{dbern}\NormalTok{(SLR[j])}
\NormalTok{  \}}

\NormalTok{  ##########################}
\NormalTok{  ### Derived parameters}
\NormalTok{  ##########################}

  \ControlFlowTok{for}\NormalTok{ (t }\ControlFlowTok{in} \DecValTok{1}\OperatorTok{:}\NormalTok{nT_age) \{}
\NormalTok{    llambda_age[t] <-}\StringTok{ }\NormalTok{beta0 }\OperatorTok{+}\StringTok{ }\NormalTok{age_effect[t]}
\NormalTok{    UCH0_age[t] <-}\StringTok{ }\KeywordTok{exp}\NormalTok{(llambda_age[t])}
\NormalTok{    S0_age[t] <-}\StringTok{ }\KeywordTok{exp}\NormalTok{(}\OperatorTok{-}\KeywordTok{sum}\NormalTok{(UCH0_age[}\DecValTok{1}\OperatorTok{:}\NormalTok{t]))}
\NormalTok{  \}}
  \ControlFlowTok{for}\NormalTok{ (t }\ControlFlowTok{in} \DecValTok{1}\OperatorTok{:}\NormalTok{nT_period) \{}
\NormalTok{    llambda_period[t] <-}\StringTok{ }\NormalTok{beta0 }\OperatorTok{+}\StringTok{ }\NormalTok{period_effect[t]}
\NormalTok{    UCH0_period[t] <-}\StringTok{ }\KeywordTok{exp}\NormalTok{(llambda_period[t])}
\NormalTok{    S0_period[t] <-}\StringTok{ }\KeywordTok{exp}\NormalTok{(}\OperatorTok{-}\KeywordTok{sum}\NormalTok{(UCH0_period[}\DecValTok{1}\OperatorTok{:}\NormalTok{t]))}
\NormalTok{  \}}

\NormalTok{\})}\CommentTok{#end model statement}
\end{Highlighting}
\end{Shaded}

We specify data, constants and initial values for all of the parameters
in the model.

\begin{Shaded}
\begin{Highlighting}[]
\NormalTok{###Data}
\NormalTok{nimData <-}\StringTok{ }\KeywordTok{list}\NormalTok{(}\DataTypeTok{censor =}\NormalTok{ df_fit[, }\DecValTok{3}\NormalTok{],}
                \DataTypeTok{Z_period =}\NormalTok{ Z_period,}
                \DataTypeTok{Z_age_cgam =}\NormalTok{ Z_age_cgam,}
                \DataTypeTok{Z_age_spline =}\NormalTok{ Z_age_spline,}
                \DataTypeTok{left_age =}\NormalTok{ df_fit[, }\DecValTok{1}\NormalTok{],}
                \DataTypeTok{right_age =}\NormalTok{ df_fit[, }\DecValTok{2}\NormalTok{],}
                \DataTypeTok{age2date =}\NormalTok{ age2date}
\NormalTok{                )}
\NormalTok{###Constants}
\NormalTok{nimConsts <-}\StringTok{ }\KeywordTok{list}\NormalTok{(}\DataTypeTok{records =}\NormalTok{ n_fit,}
                 \DataTypeTok{nT_age =}\NormalTok{ nT_age,}
                 \DataTypeTok{nT_period =}\NormalTok{ nT_period,}
                 \DataTypeTok{nknots_age_cgam =}\NormalTok{ nknots_age_cgam,}
                 \DataTypeTok{nknots_age_spline =}\NormalTok{ nknots_age_spline,}
                 \DataTypeTok{nknots_period =}\NormalTok{ nknots_period,}
                 \DataTypeTok{nconst =} \DecValTok{1} \OperatorTok{/}\StringTok{ }\KeywordTok{sqrt}\NormalTok{(}\DecValTok{2} \OperatorTok{*}\StringTok{ }\NormalTok{pi))}

\NormalTok{### Initial values}
\NormalTok{initsFun <-}\StringTok{ }\ControlFlowTok{function}\NormalTok{()}\KeywordTok{list}\NormalTok{(}\DataTypeTok{tau_age_cgam =} \KeywordTok{runif}\NormalTok{(}\DecValTok{1}\NormalTok{, .}\DecValTok{1}\NormalTok{, }\DecValTok{1}\NormalTok{),}
                          \DataTypeTok{tau_age_spline =} \KeywordTok{runif}\NormalTok{(}\DecValTok{1}\NormalTok{, .}\DecValTok{1}\NormalTok{, }\DecValTok{1}\NormalTok{),}
                          \DataTypeTok{beta0_temp =} \KeywordTok{rnorm}\NormalTok{(}\DecValTok{1}\NormalTok{, beta0_true, .}\DecValTok{0001}\NormalTok{),}
                          \DataTypeTok{mix =} \DecValTok{1}\NormalTok{,}
                          \DataTypeTok{mix2 =} \DecValTok{1}\NormalTok{,}
                          \DataTypeTok{sda_period =} \KeywordTok{runif}\NormalTok{(}\DecValTok{1}\NormalTok{, }\DecValTok{0}\NormalTok{, }\DecValTok{5}\NormalTok{),}
                          \DataTypeTok{ln_sk_period =} \KeywordTok{rnorm}\NormalTok{(}\DecValTok{1}\NormalTok{, }\DecValTok{0}\NormalTok{, }\DecValTok{1}\NormalTok{),}
                          \DataTypeTok{alpha_period =} \KeywordTok{rep}\NormalTok{(}\DecValTok{0}\NormalTok{, nknots_period),}
                          \DataTypeTok{ln_b_age_cgam =} \KeywordTok{runif}\NormalTok{(nknots_age_cgam, }\OperatorTok{-}\DecValTok{10}\NormalTok{, }\OperatorTok{-}\DecValTok{5}\NormalTok{),}
                          \DataTypeTok{b_age_spline =} \KeywordTok{rnorm}\NormalTok{(nknots_period) }\OperatorTok{*}\StringTok{ }\DecValTok{10}\OperatorTok{^-}\DecValTok{4}
\NormalTok{                          )}
\NormalTok{nimInits <-}\StringTok{ }\KeywordTok{initsFun}\NormalTok{()}
\end{Highlighting}
\end{Shaded}

Then we build the model, MCMC object, and run the MCMC approximation. We
save the MCMC iterations.

\begin{Shaded}
\begin{Highlighting}[]
\NormalTok{Rmodel <-}\StringTok{ }\KeywordTok{nimbleModel}\NormalTok{(}\DataTypeTok{code =}\NormalTok{ modelcode,}
                      \DataTypeTok{constants =}\NormalTok{ nimConsts,}
                      \DataTypeTok{data =}\NormalTok{ nimData,}
                      \DataTypeTok{inits =} \KeywordTok{initsFun}\NormalTok{()}
\NormalTok{                      )}

\CommentTok{#identify params to monitor}
\NormalTok{parameters <-}\StringTok{ }\KeywordTok{c}\NormalTok{(}
              \StringTok{"beta0"}\NormalTok{,}
              \StringTok{"S0_age"}\NormalTok{,}
              \StringTok{"age_effect"}\NormalTok{,}
              \StringTok{"age_effect_cgam"}\NormalTok{,}
              \StringTok{"age_effect_spline"}\NormalTok{,}
              \StringTok{"b_age_spline"}\NormalTok{,}
              \StringTok{"b_age_cgam"}\NormalTok{,}
              \StringTok{"tau_age_cgam"}\NormalTok{,}
              \StringTok{"tau_age_spline"}\NormalTok{,}
              \StringTok{"mu_age_cgam"}\NormalTok{,}
              \StringTok{"period_effect"}\NormalTok{,}
              \StringTok{"S0_period"}\NormalTok{,}
              \StringTok{"llambda_period"}\NormalTok{,}
              \StringTok{"sdk_period"}\NormalTok{,}
              \StringTok{"sda_period"}\NormalTok{,}
              \StringTok{"alpha_period"}\NormalTok{,}
              \StringTok{"ratioinf_period"}
\NormalTok{              )}

\NormalTok{starttime<-}\KeywordTok{Sys.time}\NormalTok{()}
\NormalTok{confMCMC <-}\StringTok{ }\KeywordTok{configureMCMC}\NormalTok{(Rmodel,}
                          \DataTypeTok{monitors =}\NormalTok{ parameters,}
                          \DataTypeTok{thin =}\NormalTok{ n_thin,}
                          \DataTypeTok{useConjugacy =} \OtherTok{FALSE}\NormalTok{)}
\NormalTok{nimMCMC <-}\StringTok{ }\KeywordTok{buildMCMC}\NormalTok{(confMCMC, }\DataTypeTok{enableWAIC =} \OtherTok{TRUE}\NormalTok{)}
\NormalTok{Cnim <-}\StringTok{ }\KeywordTok{compileNimble}\NormalTok{(Rmodel)}
\NormalTok{CnimMCMC <-}\StringTok{ }\KeywordTok{compileNimble}\NormalTok{(nimMCMC, }\DataTypeTok{project=}\NormalTok{Rmodel)}
\NormalTok{mcmcout <-}\StringTok{ }\KeywordTok{runMCMC}\NormalTok{(CnimMCMC,}
                   \DataTypeTok{niter =}\NormalTok{ reps,}
                   \DataTypeTok{nburnin =}\NormalTok{ bin,}
                   \DataTypeTok{nchains =}\NormalTok{ n_chains,}
                   \DataTypeTok{inits =}\NormalTok{ initsFun,}
                   \DataTypeTok{samplesAsCodaMCMC =} \OtherTok{TRUE}\NormalTok{,}
                   \DataTypeTok{summary =} \OtherTok{TRUE}\NormalTok{,}
                   \DataTypeTok{WAIC =} \OtherTok{TRUE}\NormalTok{)}

\NormalTok{runtime <-}\StringTok{ }\KeywordTok{diffperiod_}\NormalTok{(}\KeywordTok{Sys.time}\NormalTok{(), starttime, }\DataTypeTok{units =} \StringTok{"min"}\NormalTok{)}

\KeywordTok{save}\NormalTok{(runtime, }\DataTypeTok{file =} \StringTok{"results/runtime.Rdata"}\NormalTok{)}
\KeywordTok{save}\NormalTok{(mcmcout, }\DataTypeTok{file =} \StringTok{"results/mcmcout.Rdata"}\NormalTok{)}
\end{Highlighting}
\end{Shaded}

All of the code in this section is aggregated and combined in the file
S4\_04\_run\_model\_kselect\_CSL\_K.R. Simulations were run on the
University of Wisconsin, Madison's center for high though-put computing
cluster. For each of the simulations, a different seed was used to run
the models. Code for plotting this results of the simulation from a
single seed, as well as for obtaining convergence diagnostics will be
provided in the file S4\_05\_post\_plots\_kselect\_CSL\_K.R.

It is not possible to extensively describe all of the models from both
the simulations and from the case studies. However, they were all fit
using similar code/methods. A key for fitting these models in in having
to calculate the basis functions. In the model described above, several
of the basis functions were used. I have aggregated all of the basis
function calculations from all of the models in the document
S4\_03\_basis\_functions.R.

Code for running the kselect simulation for a single seed has been
provided. To avoid redundant code, these same models were used for the
rselect simulation for the single seed, as well as for both case
studies. Data sets from the white-tailed deer and Columbian sharp-tailed
grouse case studies were used rather than generated data. Code for
loading these data are provided in the corresponding case study
directories.

\end{document}
