\documentclass[11pt,]{article}
\usepackage{lmodern}
\usepackage{amssymb,amsmath}
\usepackage{ifxetex,ifluatex}
\usepackage{fixltx2e} % provides \textsubscript
\ifnum 0\ifxetex 1\fi\ifluatex 1\fi=0 % if pdftex
  \usepackage[T1]{fontenc}
  \usepackage[utf8]{inputenc}
\else % if luatex or xelatex
  \ifxetex
    \usepackage{mathspec}
  \else
    \usepackage{fontspec}
  \fi
  \defaultfontfeatures{Ligatures=TeX,Scale=MatchLowercase}
\fi
% use upquote if available, for straight quotes in verbatim environments
\IfFileExists{upquote.sty}{\usepackage{upquote}}{}
% use microtype if available
\IfFileExists{microtype.sty}{%
\usepackage[]{microtype}
\UseMicrotypeSet[protrusion]{basicmath} % disable protrusion for tt fonts
}{}
\PassOptionsToPackage{hyphens}{url} % url is loaded by hyperref
\usepackage[unicode=true]{hyperref}
\PassOptionsToPackage{usenames,dvipsnames}{color} % color is loaded by hyperref
\hypersetup{
            pdftitle={Appendix S5 Columbian Sharp-tailed Grouse Age-Period Survival Analysis R Code},
            pdfauthor={Alison C. Ketz},
            colorlinks=true,
            linkcolor=Maroon,
            citecolor=Blue,
            urlcolor=Blue,
            breaklinks=true}
\urlstyle{same}  % don't use monospace font for urls
\usepackage[margin=1in]{geometry}
\usepackage{color}
\usepackage{fancyvrb}
\newcommand{\VerbBar}{|}
\newcommand{\VERB}{\Verb[commandchars=\\\{\}]}
\DefineVerbatimEnvironment{Highlighting}{Verbatim}{commandchars=\\\{\}}
% Add ',fontsize=\small' for more characters per line
\usepackage{framed}
\definecolor{shadecolor}{RGB}{248,248,248}
\newenvironment{Shaded}{\begin{snugshade}}{\end{snugshade}}
\newcommand{\KeywordTok}[1]{\textcolor[rgb]{0.13,0.29,0.53}{\textbf{#1}}}
\newcommand{\DataTypeTok}[1]{\textcolor[rgb]{0.13,0.29,0.53}{#1}}
\newcommand{\DecValTok}[1]{\textcolor[rgb]{0.00,0.00,0.81}{#1}}
\newcommand{\BaseNTok}[1]{\textcolor[rgb]{0.00,0.00,0.81}{#1}}
\newcommand{\FloatTok}[1]{\textcolor[rgb]{0.00,0.00,0.81}{#1}}
\newcommand{\ConstantTok}[1]{\textcolor[rgb]{0.00,0.00,0.00}{#1}}
\newcommand{\CharTok}[1]{\textcolor[rgb]{0.31,0.60,0.02}{#1}}
\newcommand{\SpecialCharTok}[1]{\textcolor[rgb]{0.00,0.00,0.00}{#1}}
\newcommand{\StringTok}[1]{\textcolor[rgb]{0.31,0.60,0.02}{#1}}
\newcommand{\VerbatimStringTok}[1]{\textcolor[rgb]{0.31,0.60,0.02}{#1}}
\newcommand{\SpecialStringTok}[1]{\textcolor[rgb]{0.31,0.60,0.02}{#1}}
\newcommand{\ImportTok}[1]{#1}
\newcommand{\CommentTok}[1]{\textcolor[rgb]{0.56,0.35,0.01}{\textit{#1}}}
\newcommand{\DocumentationTok}[1]{\textcolor[rgb]{0.56,0.35,0.01}{\textbf{\textit{#1}}}}
\newcommand{\AnnotationTok}[1]{\textcolor[rgb]{0.56,0.35,0.01}{\textbf{\textit{#1}}}}
\newcommand{\CommentVarTok}[1]{\textcolor[rgb]{0.56,0.35,0.01}{\textbf{\textit{#1}}}}
\newcommand{\OtherTok}[1]{\textcolor[rgb]{0.56,0.35,0.01}{#1}}
\newcommand{\FunctionTok}[1]{\textcolor[rgb]{0.00,0.00,0.00}{#1}}
\newcommand{\VariableTok}[1]{\textcolor[rgb]{0.00,0.00,0.00}{#1}}
\newcommand{\ControlFlowTok}[1]{\textcolor[rgb]{0.13,0.29,0.53}{\textbf{#1}}}
\newcommand{\OperatorTok}[1]{\textcolor[rgb]{0.81,0.36,0.00}{\textbf{#1}}}
\newcommand{\BuiltInTok}[1]{#1}
\newcommand{\ExtensionTok}[1]{#1}
\newcommand{\PreprocessorTok}[1]{\textcolor[rgb]{0.56,0.35,0.01}{\textit{#1}}}
\newcommand{\AttributeTok}[1]{\textcolor[rgb]{0.77,0.63,0.00}{#1}}
\newcommand{\RegionMarkerTok}[1]{#1}
\newcommand{\InformationTok}[1]{\textcolor[rgb]{0.56,0.35,0.01}{\textbf{\textit{#1}}}}
\newcommand{\WarningTok}[1]{\textcolor[rgb]{0.56,0.35,0.01}{\textbf{\textit{#1}}}}
\newcommand{\AlertTok}[1]{\textcolor[rgb]{0.94,0.16,0.16}{#1}}
\newcommand{\ErrorTok}[1]{\textcolor[rgb]{0.64,0.00,0.00}{\textbf{#1}}}
\newcommand{\NormalTok}[1]{#1}
\usepackage{graphicx,grffile}
\makeatletter
\def\maxwidth{\ifdim\Gin@nat@width>\linewidth\linewidth\else\Gin@nat@width\fi}
\def\maxheight{\ifdim\Gin@nat@height>\textheight\textheight\else\Gin@nat@height\fi}
\makeatother
% Scale images if necessary, so that they will not overflow the page
% margins by default, and it is still possible to overwrite the defaults
% using explicit options in \includegraphics[width, height, ...]{}
\setkeys{Gin}{width=\maxwidth,height=\maxheight,keepaspectratio}
\IfFileExists{parskip.sty}{%
\usepackage{parskip}
}{% else
\setlength{\parindent}{0pt}
\setlength{\parskip}{6pt plus 2pt minus 1pt}
}
\setlength{\emergencystretch}{3em}  % prevent overfull lines
\providecommand{\tightlist}{%
  \setlength{\itemsep}{0pt}\setlength{\parskip}{0pt}}
\setcounter{secnumdepth}{5}
% Redefines (sub)paragraphs to behave more like sections
\ifx\paragraph\undefined\else
\let\oldparagraph\paragraph
\renewcommand{\paragraph}[1]{\oldparagraph{#1}\mbox{}}
\fi
\ifx\subparagraph\undefined\else
\let\oldsubparagraph\subparagraph
\renewcommand{\subparagraph}[1]{\oldsubparagraph{#1}\mbox{}}
\fi

% set default figure placement to htbp
\makeatletter
\def\fps@figure{htbp}
\makeatother


\title{Appendix S5 Columbian Sharp-tailed Grouse Age-Period Survival Analysis R
Code}
\author{Alison C. Ketz}
\date{11/10/2021}

\begin{document}
\maketitle

\section{Analysis Approach}\label{analysis-approach}

This appendix provides R code for implementing the analysis of the
age-period survival models using data from the Columbian sharp-tailed
grouse case study.The model fitting procedure for a single model
(CSL\_K, an additive constrained generalized additive model with natural
splines specified with the Bayesian LASSO for age effects and the kernel
convolution model for the period effects) is described in detail. The
other models have similarly conducted analysis structure so the
step-by-step break down of running each model will be analogous despite
the somewhat differing model structure, and consequently, somewhat
different R code. We have provided code for each of the 10 models that
were fit using NIMBLE, where each model has five R scripts including: 1)
A script to execute and run all the additional Rscripts as well as for
providing necessary libraries. 2) A script to load the data and compute
constants needed to run the analysis. 3) A script that computes the
necessary basis expansion, and calculates any other constants needed for
model fitting. 4) A script for running the models in nimble with
additional nimble functions used for intermediary calculations. 5) An R
script for doing model checking, calculations of posterior distribution
summary statistics, and plotting of results. These five R scripts for
each model are provided for all 10 models.

\subsection{Mother script to run all
scripts}\label{mother-script-to-run-all-scripts}

The first script in each analysis is used to run all of the other
scripts. It is named S5\_01\_grouse\_CSL\_K.R, where each model
signifier will change according to the initials described in the main
body of the manuscript. Thus, each mother scripts will be specified
using S5\_01\_grouse\_X\_Y.R, where X represents the age effects
specification and Y represents period effects specification. First,
clear the memory of R and set the working directory. Then, load the
libraries needed in the analysis. Sourcing each of the subsequent R
scripts will run the complete analysis.

\begin{Shaded}
\begin{Highlighting}[]
\CommentTok{#clear memory}
\KeywordTok{rm}\NormalTok{(}\DataTypeTok{list =} \KeywordTok{ls}\NormalTok{())}

\CommentTok{#setwd}
\KeywordTok{setwd}\NormalTok{(}\StringTok{".../grouse_CSL_K"}\NormalTok{)}

\CommentTok{#load libraries}
\KeywordTok{library}\NormalTok{(nimble)}
\KeywordTok{library}\NormalTok{(Matrix)}
\KeywordTok{library}\NormalTok{(coda)}
\KeywordTok{library}\NormalTok{(lattice)}
\KeywordTok{library}\NormalTok{(splines)}
\KeywordTok{library}\NormalTok{(Hmisc)}
\KeywordTok{library}\NormalTok{(lubridate)}
\KeywordTok{library}\NormalTok{(ggplot2)}
\KeywordTok{library}\NormalTok{(gridExtra)}
\KeywordTok{library}\NormalTok{(xtable)}
\KeywordTok{library}\NormalTok{(RColorBrewer)}

\NormalTok{####################################}
\NormalTok{### Load data}
\NormalTok{####################################}

\KeywordTok{source}\NormalTok{(}\StringTok{"S5_02_load_format_data.R"}\NormalTok{)}

\NormalTok{##################################}
\NormalTok{### Setting constants}
\NormalTok{##################################}

\KeywordTok{source}\NormalTok{(}\StringTok{"S5_03_prelim_CSL_K.R"}\NormalTok{)}

\NormalTok{##################################}
\NormalTok{### Run model}
\NormalTok{##################################}

\KeywordTok{source}\NormalTok{(}\StringTok{"S5_04_run_model_CSL_K.R"}\NormalTok{)}

\NormalTok{######################################}
\NormalTok{### Post processing: summaries, plots}
\NormalTok{######################################}

\KeywordTok{source}\NormalTok{(}\StringTok{"S5_05_results_plots_sum_CSL_K.R"}\NormalTok{)}
\end{Highlighting}
\end{Shaded}

\subsection{Loading and Formatting
Data}\label{loading-and-formatting-data}

The R script S5\_02\_load\_format\_data.R is used to load the data and
calculate constants. First, we load the data, and set the number of
records in the data frame. The data are already formatted for model
fitting. Then we calculate the vector age2date, which is used to offset
index differences between the period effects and age effects. The number
of age effects are set as a constant. The number of period effects for
each year are calculated from the data, and the total possible number of
period effects are also calculated.

\begin{Shaded}
\begin{Highlighting}[]
\NormalTok{d_fit <-}\StringTok{ }\KeywordTok{read.csv}\NormalTok{(}\StringTok{"../grouse_data.csv"}\NormalTok{)}
\NormalTok{n_fit <-}\StringTok{ }\KeywordTok{dim}\NormalTok{(d_fit)[}\DecValTok{1}\NormalTok{]}

\CommentTok{#vector to calibrate indexes}
\NormalTok{age2date <-}\StringTok{ }\NormalTok{d_fit}\OperatorTok{$}\NormalTok{left.time }\OperatorTok{-}\StringTok{ }\NormalTok{d_fit}\OperatorTok{$}\NormalTok{left.age}

\NormalTok{### Number of Age effects }
\NormalTok{nT_age <-}\StringTok{ }\KeywordTok{max}\NormalTok{(d_fit}\OperatorTok{$}\NormalTok{right.age) }\OperatorTok{-}\StringTok{ }\DecValTok{1}

\CommentTok{#Number of Period effects for each }
\CommentTok{#year seperately and combined}
\NormalTok{nT_yr1 <-}\StringTok{ }\KeywordTok{length}\NormalTok{(}\KeywordTok{as.numeric}\NormalTok{(}\KeywordTok{names}\NormalTok{(}\KeywordTok{table}\NormalTok{(d_fit[d_fit[,}\DecValTok{5}\NormalTok{]}\OperatorTok{==}\DecValTok{1}\NormalTok{,}\DecValTok{1}\NormalTok{]))[}\DecValTok{1}\NormalTok{])}\OperatorTok{:}\KeywordTok{as.numeric}\NormalTok{(}\KeywordTok{names}\NormalTok{(}\KeywordTok{table}\NormalTok{(d_fit[d_fit[,}\DecValTok{5}\NormalTok{]}\OperatorTok{==}\DecValTok{1}\NormalTok{,}\DecValTok{2}\NormalTok{]))[}\KeywordTok{length}\NormalTok{(}\KeywordTok{table}\NormalTok{(d_fit[d_fit[,}\DecValTok{5}\NormalTok{]}\OperatorTok{==}\DecValTok{1}\NormalTok{,}\DecValTok{2}\NormalTok{]))]))}
\NormalTok{nT_yr2 <-}\StringTok{ }\KeywordTok{length}\NormalTok{(}\KeywordTok{as.numeric}\NormalTok{(}\KeywordTok{names}\NormalTok{(}\KeywordTok{table}\NormalTok{(d_fit[d_fit[,}\DecValTok{6}\NormalTok{]}\OperatorTok{==}\DecValTok{1}\NormalTok{,}\DecValTok{1}\NormalTok{]))[}\DecValTok{1}\NormalTok{])}\OperatorTok{:}\KeywordTok{as.numeric}\NormalTok{(}\KeywordTok{names}\NormalTok{(}\KeywordTok{table}\NormalTok{(d_fit[d_fit[,}\DecValTok{6}\NormalTok{]}\OperatorTok{==}\DecValTok{1}\NormalTok{,}\DecValTok{2}\NormalTok{]))[}\KeywordTok{length}\NormalTok{(}\KeywordTok{table}\NormalTok{(d_fit[d_fit[,}\DecValTok{6}\NormalTok{]}\OperatorTok{==}\DecValTok{1}\NormalTok{,}\DecValTok{2}\NormalTok{]))]))}
\NormalTok{nT_yr3 <-}\StringTok{ }\KeywordTok{length}\NormalTok{(}\KeywordTok{as.numeric}\NormalTok{(}\KeywordTok{names}\NormalTok{(}\KeywordTok{table}\NormalTok{(d_fit[d_fit[,}\DecValTok{7}\NormalTok{]}\OperatorTok{==}\DecValTok{1}\NormalTok{,}\DecValTok{1}\NormalTok{]))[}\DecValTok{1}\NormalTok{])}\OperatorTok{:}\KeywordTok{as.numeric}\NormalTok{(}\KeywordTok{names}\NormalTok{(}\KeywordTok{table}\NormalTok{(d_fit[d_fit[,}\DecValTok{7}\NormalTok{]}\OperatorTok{==}\DecValTok{1}\NormalTok{,}\DecValTok{2}\NormalTok{]))[}\KeywordTok{length}\NormalTok{(}\KeywordTok{table}\NormalTok{(d_fit[d_fit[,}\DecValTok{7}\NormalTok{]}\OperatorTok{==}\DecValTok{1}\NormalTok{,}\DecValTok{2}\NormalTok{]))]))}
\NormalTok{nT_period_total <-}\StringTok{ }\NormalTok{nT_yr1}\OperatorTok{+}\NormalTok{nT_yr2}\OperatorTok{+}\NormalTok{nT_yr3}
\NormalTok{nT_period_max <-}\StringTok{ }\KeywordTok{max}\NormalTok{(nT_yr1, nT_yr2, nT_yr3)}
\NormalTok{nT_period <-}\StringTok{ }\KeywordTok{max}\NormalTok{(d_fit}\OperatorTok{$}\NormalTok{right.time)}
\end{Highlighting}
\end{Shaded}

\subsection{Calculating Basis
Expansion}\label{calculating-basis-expansion}

The R script S5\_03\_prelim\_CSL\_K.R is used to calculate the basis
expansion used in the analysis. First, we specify and compile a Nimble
function to calculate the kernel convolution. This function takes seven
arguments. The maximum number of effects (nT), the basis function, or
distance matrix (Z), the square root of the precision parameter of the
kernel convolution (stauk), a constant (nconst = \(1 / sqrt(2 * \pi)\)),
the precision of the kernel (tauk), the number of knots which also
equals the number of columns in the distance matrix (nknots), and the
white noise random effects used to convolve the kernel density (alphau).
The normal kernel density is calculated for each effect and knot in
temp1. These are normalized by summing the temp1 values and dividing by
the sum of the temp1 values for each effect. Then the normalized kernel
density is convolved with the white noise process in temp. These are
mean centered to ultimately calculate the kernel convolution process
(KA), which is then returned.

\begin{Shaded}
\begin{Highlighting}[]
\NormalTok{kernel_conv <-}\StringTok{ }\KeywordTok{nimbleFunction}\NormalTok{(}
  \DataTypeTok{run =} \ControlFlowTok{function}\NormalTok{(}\DataTypeTok{nT =} \KeywordTok{double}\NormalTok{(}\DecValTok{0}\NormalTok{),}
                 \DataTypeTok{Z =} \KeywordTok{double}\NormalTok{(}\DecValTok{2}\NormalTok{),}
                 \DataTypeTok{stauk =} \KeywordTok{double}\NormalTok{(}\DecValTok{0}\NormalTok{),}
                 \DataTypeTok{nconst =} \KeywordTok{double}\NormalTok{(}\DecValTok{0}\NormalTok{),}
                 \DataTypeTok{tauk =} \KeywordTok{double}\NormalTok{(}\DecValTok{0}\NormalTok{),}
                 \DataTypeTok{nknots =} \KeywordTok{double}\NormalTok{(}\DecValTok{0}\NormalTok{),}
                 \DataTypeTok{alphau =} \KeywordTok{double}\NormalTok{(}\DecValTok{1}\NormalTok{)}
\NormalTok{  )\{}
\NormalTok{    temp <-}\StringTok{ }\KeywordTok{nimMatrix}\NormalTok{(}\DataTypeTok{value =} \DecValTok{0}\NormalTok{, }\DataTypeTok{nrow =}\NormalTok{ nT, }\DataTypeTok{ncol =}\NormalTok{ nknots)}
\NormalTok{    temp1 <-}\StringTok{ }\KeywordTok{nimMatrix}\NormalTok{(}\DataTypeTok{value =} \DecValTok{0}\NormalTok{, }\DataTypeTok{nrow =}\NormalTok{ nT, }\DataTypeTok{ncol =}\NormalTok{ nknots)}
\NormalTok{    temp2 <-}\StringTok{ }\KeywordTok{nimNumeric}\NormalTok{(nknots)}
\NormalTok{    KA <-}\StringTok{ }\KeywordTok{nimNumeric}\NormalTok{(nT)}

    \ControlFlowTok{for}\NormalTok{ (i }\ControlFlowTok{in} \DecValTok{1}\OperatorTok{:}\NormalTok{nT) \{}
      \ControlFlowTok{for}\NormalTok{ (j }\ControlFlowTok{in} \DecValTok{1}\OperatorTok{:}\NormalTok{nknots) \{}
\NormalTok{        temp1[i, j] <-}\StringTok{ }\NormalTok{stauk }\OperatorTok{*}\StringTok{ }\NormalTok{nconst }\OperatorTok{*}\StringTok{ }\KeywordTok{exp}\NormalTok{(}\OperatorTok{-}\FloatTok{0.5} \OperatorTok{*}\StringTok{ }\NormalTok{Z[i, j]}\OperatorTok{^}\DecValTok{2} \OperatorTok{*}\StringTok{ }\NormalTok{tauk)}
\NormalTok{      \}}
\NormalTok{    \}}

    \ControlFlowTok{for}\NormalTok{ (j }\ControlFlowTok{in} \DecValTok{1}\OperatorTok{:}\NormalTok{nknots) \{}
\NormalTok{      temp2[j] <-}\StringTok{ }\KeywordTok{sum}\NormalTok{(temp1[}\DecValTok{1}\OperatorTok{:}\NormalTok{nT, j])}
\NormalTok{    \}}

    \ControlFlowTok{for}\NormalTok{ (i }\ControlFlowTok{in} \DecValTok{1}\OperatorTok{:}\NormalTok{nT) \{}
      \ControlFlowTok{for}\NormalTok{ (j }\ControlFlowTok{in} \DecValTok{1}\OperatorTok{:}\NormalTok{nknots) \{}
\NormalTok{        temp[i, j] <-}\StringTok{ }\NormalTok{(temp1[i, j] }\OperatorTok{/}\StringTok{ }\NormalTok{temp2[j]) }\OperatorTok{*}\StringTok{ }\NormalTok{alphau[j]}
\NormalTok{      \}}
\NormalTok{      KA[i] <-}\StringTok{ }\KeywordTok{sum}\NormalTok{(temp[i, }\DecValTok{1}\OperatorTok{:}\NormalTok{nknots])}
\NormalTok{    \}}
\NormalTok{    muKA <-}\StringTok{ }\KeywordTok{mean}\NormalTok{(KA[}\DecValTok{1}\OperatorTok{:}\NormalTok{nT])}
\NormalTok{    KA[}\DecValTok{1}\OperatorTok{:}\NormalTok{nT] <-}\StringTok{ }\NormalTok{KA[}\DecValTok{1}\OperatorTok{:}\NormalTok{nT] }\OperatorTok{-}\StringTok{ }\NormalTok{muKA}

    \KeywordTok{returnType}\NormalTok{(}\KeywordTok{double}\NormalTok{(}\DecValTok{1}\NormalTok{))}
    \KeywordTok{return}\NormalTok{(KA[}\DecValTok{1}\OperatorTok{:}\NormalTok{nT])}
\NormalTok{  \})}

\NormalTok{Ckernel_conv <-}\StringTok{ }\KeywordTok{compileNimble}\NormalTok{(kernel_conv)}
\end{Highlighting}
\end{Shaded}

Next, we specify the function used for the constrained generalized
additive model basis calculation. For the grouse data, we use a function
that caculates a basis that assumes a shape-constrained function to be
descending an convex (decconvex). See Meyer et al (2008) for details.

\begin{Shaded}
\begin{Highlighting}[]
\NormalTok{decconvex=}\ControlFlowTok{function}\NormalTok{(x,t)\{}
\NormalTok{  n=}\KeywordTok{length}\NormalTok{(x)}
\NormalTok{  k=}\KeywordTok{length}\NormalTok{(t)}\OperatorTok{-}\DecValTok{2}
\NormalTok{  m=k}\OperatorTok{+}\DecValTok{3}
\NormalTok{  sigma=}\KeywordTok{matrix}\NormalTok{(}\DecValTok{1}\OperatorTok{:}\NormalTok{(m}\OperatorTok{*}\NormalTok{n)}\OperatorTok{*}\DecValTok{0}\NormalTok{,}\DataTypeTok{nrow=}\NormalTok{m,}\DataTypeTok{ncol=}\NormalTok{n)}
  \ControlFlowTok{for}\NormalTok{(j }\ControlFlowTok{in} \DecValTok{1}\OperatorTok{:}\NormalTok{k)\{}
\NormalTok{    i1=x}\OperatorTok{<=}\NormalTok{t[j]}
\NormalTok{    sigma[j,i1] =}\StringTok{ }\NormalTok{x[i1]}\OperatorTok{-}\NormalTok{t[}\DecValTok{1}\NormalTok{]}
\NormalTok{    i2=x}\OperatorTok{>}\NormalTok{t[j]}\OperatorTok{&}\NormalTok{x}\OperatorTok{<=}\NormalTok{t[j}\OperatorTok{+}\DecValTok{1}\NormalTok{]}
\NormalTok{    sigma[j,i2] =}\StringTok{ }\NormalTok{t[j]}\OperatorTok{-}\NormalTok{t[}\DecValTok{1}\NormalTok{]}\OperatorTok{+}\NormalTok{((t[j}\OperatorTok{+}\DecValTok{1}\NormalTok{]}\OperatorTok{-}\NormalTok{t[j])}\OperatorTok{^}\DecValTok{3}\OperatorTok{-}\NormalTok{(t[j}\OperatorTok{+}\DecValTok{1}\NormalTok{]}\OperatorTok{-}\NormalTok{x[i2])}\OperatorTok{^}\DecValTok{3}\NormalTok{)}\OperatorTok{/}\DecValTok{3}\OperatorTok{/}\NormalTok{(t[j}\OperatorTok{+}\DecValTok{1}\NormalTok{]}\OperatorTok{-}\NormalTok{t[j])}\OperatorTok{/}\NormalTok{(t[j}\OperatorTok{+}\DecValTok{2}\NormalTok{]}\OperatorTok{-}\NormalTok{t[j]) }\OperatorTok{+}\NormalTok{(x[i2]}\OperatorTok{-}\NormalTok{t[j])}\OperatorTok{*}\NormalTok{(t[j}\OperatorTok{+}\DecValTok{2}\NormalTok{]}\OperatorTok{-}\NormalTok{t[j}\OperatorTok{+}\DecValTok{1}\NormalTok{])}\OperatorTok{/}\NormalTok{(t[j}\OperatorTok{+}\DecValTok{2}\NormalTok{]}\OperatorTok{-}\NormalTok{t[j])}
\NormalTok{    i3=x}\OperatorTok{>}\NormalTok{t[j}\OperatorTok{+}\DecValTok{1}\NormalTok{]}\OperatorTok{&}\NormalTok{x}\OperatorTok{<=}\NormalTok{t[j}\OperatorTok{+}\DecValTok{2}\NormalTok{]}
\NormalTok{    sigma[j,i3] =}\StringTok{ }\NormalTok{t[j]}\OperatorTok{-}\NormalTok{t[}\DecValTok{1}\NormalTok{] }\OperatorTok{+}\StringTok{ }\NormalTok{(t[j}\OperatorTok{+}\DecValTok{1}\NormalTok{]}\OperatorTok{-}\NormalTok{t[j])}\OperatorTok{^}\DecValTok{2}\OperatorTok{/}\DecValTok{3}\OperatorTok{/}\NormalTok{(t[j}\OperatorTok{+}\DecValTok{2}\NormalTok{]}\OperatorTok{-}\NormalTok{t[j]) }\OperatorTok{+}\StringTok{ }\NormalTok{(t[j}\OperatorTok{+}\DecValTok{2}\NormalTok{]}\OperatorTok{-}\NormalTok{t[j}\OperatorTok{+}\DecValTok{1}\NormalTok{])}\OperatorTok{*}\NormalTok{(t[j}\OperatorTok{+}\DecValTok{1}\NormalTok{]}\OperatorTok{-}\NormalTok{t[j])}\OperatorTok{/}\NormalTok{(t[j}\OperatorTok{+}\DecValTok{2}\NormalTok{]}\OperatorTok{-}\NormalTok{t[j]) }\OperatorTok{+}\NormalTok{((t[j}\OperatorTok{+}\DecValTok{2}\NormalTok{]}\OperatorTok{-}\NormalTok{t[j}\OperatorTok{+}\DecValTok{1}\NormalTok{])}\OperatorTok{^}\DecValTok{3}\OperatorTok{-}\NormalTok{(t[j}\OperatorTok{+}\DecValTok{2}\NormalTok{]}\OperatorTok{-}\NormalTok{x[i3])}\OperatorTok{^}\DecValTok{3}\NormalTok{)}\OperatorTok{/}\DecValTok{3}\OperatorTok{/}\NormalTok{(t[j}\OperatorTok{+}\DecValTok{2}\NormalTok{]}\OperatorTok{-}\NormalTok{t[j}\OperatorTok{+}\DecValTok{1}\NormalTok{])}\OperatorTok{/}\NormalTok{(t[j}\OperatorTok{+}\DecValTok{2}\NormalTok{]}\OperatorTok{-}\NormalTok{t[j])}
\NormalTok{    i4=x}\OperatorTok{>=}\NormalTok{t[j}\OperatorTok{+}\DecValTok{2}\NormalTok{]}
\NormalTok{    sigma[j,i4] =}\StringTok{ }\NormalTok{t[j]}\OperatorTok{-}\NormalTok{t[}\DecValTok{1}\NormalTok{] }\OperatorTok{+}\StringTok{ }\NormalTok{(t[j}\OperatorTok{+}\DecValTok{1}\NormalTok{]}\OperatorTok{-}\NormalTok{t[j])}\OperatorTok{^}\DecValTok{2}\OperatorTok{/}\DecValTok{3}\OperatorTok{/}\NormalTok{(t[j}\OperatorTok{+}\DecValTok{2}\NormalTok{]}\OperatorTok{-}\NormalTok{t[j]) }\OperatorTok{+}\StringTok{ }\NormalTok{(t[j}\OperatorTok{+}\DecValTok{2}\NormalTok{]}\OperatorTok{-}\NormalTok{t[j}\OperatorTok{+}\DecValTok{1}\NormalTok{])}\OperatorTok{*}\NormalTok{(t[j}\OperatorTok{+}\DecValTok{1}\NormalTok{]}\OperatorTok{-}\NormalTok{t[j])}\OperatorTok{/}\NormalTok{(t[j}\OperatorTok{+}\DecValTok{2}\NormalTok{]}\OperatorTok{-}\NormalTok{t[j]) }\OperatorTok{+}\NormalTok{(t[j}\OperatorTok{+}\DecValTok{2}\NormalTok{]}\OperatorTok{-}\NormalTok{t[j}\OperatorTok{+}\DecValTok{1}\NormalTok{])}\OperatorTok{^}\DecValTok{2}\OperatorTok{/}\DecValTok{3}\OperatorTok{/}\NormalTok{(t[j}\OperatorTok{+}\DecValTok{2}\NormalTok{]}\OperatorTok{-}\NormalTok{t[j])}
\NormalTok{  \}}
\NormalTok{  i1=x}\OperatorTok{<=}\NormalTok{t[}\DecValTok{2}\NormalTok{]}
\NormalTok{  sigma[k}\OperatorTok{+}\DecValTok{1}\NormalTok{,i1]=}\OperatorTok{-}\NormalTok{(t[}\DecValTok{2}\NormalTok{]}\OperatorTok{-}\NormalTok{x[i1])}\OperatorTok{^}\DecValTok{3}\OperatorTok{/}\DecValTok{3}\OperatorTok{/}\NormalTok{(t[}\DecValTok{2}\NormalTok{]}\OperatorTok{-}\NormalTok{t[}\DecValTok{1}\NormalTok{])}\OperatorTok{^}\DecValTok{2}
\NormalTok{  i2=x}\OperatorTok{>}\NormalTok{t[}\DecValTok{2}\NormalTok{]}
\NormalTok{  sigma[k}\OperatorTok{+}\DecValTok{1}\NormalTok{,i2]=}\DecValTok{0}
\NormalTok{  i1=x}\OperatorTok{<=}\NormalTok{t[k}\OperatorTok{+}\DecValTok{1}\NormalTok{]}
\NormalTok{  sigma[k}\OperatorTok{+}\DecValTok{2}\NormalTok{,i1]=x[i1]}\OperatorTok{-}\NormalTok{t[}\DecValTok{1}\NormalTok{]}
\NormalTok{  i2=x}\OperatorTok{>}\NormalTok{t[k}\OperatorTok{+}\DecValTok{1}\NormalTok{]}\OperatorTok{&}\NormalTok{x}\OperatorTok{<=}\NormalTok{t[k}\OperatorTok{+}\DecValTok{2}\NormalTok{]}
\NormalTok{  sigma[k}\OperatorTok{+}\DecValTok{2}\NormalTok{,i2]=t[k}\OperatorTok{+}\DecValTok{1}\NormalTok{]}\OperatorTok{-}\NormalTok{t[}\DecValTok{1}\NormalTok{]}\OperatorTok{+}\NormalTok{((t[k}\OperatorTok{+}\DecValTok{2}\NormalTok{]}\OperatorTok{-}\NormalTok{t[k}\OperatorTok{+}\DecValTok{1}\NormalTok{])}\OperatorTok{^}\DecValTok{2}\OperatorTok{*}\NormalTok{(x[i2]}\OperatorTok{-}\NormalTok{t[k}\OperatorTok{+}\DecValTok{1}\NormalTok{])}\OperatorTok{-}\NormalTok{(x[i2]}\OperatorTok{-}\NormalTok{t[k}\OperatorTok{+}\DecValTok{1}\NormalTok{])}\OperatorTok{^}\DecValTok{3}\OperatorTok{/}\DecValTok{3}\NormalTok{)}\OperatorTok{/}\NormalTok{(t[k}\OperatorTok{+}\DecValTok{2}\NormalTok{]}\OperatorTok{-}\NormalTok{t[k}\OperatorTok{+}\DecValTok{1}\NormalTok{])}\OperatorTok{^}\DecValTok{2}
\NormalTok{  i3=x}\OperatorTok{>}\NormalTok{t[k}\OperatorTok{+}\DecValTok{2}\NormalTok{]}
\NormalTok{  sigma[k}\OperatorTok{+}\DecValTok{2}\NormalTok{,i3]=t[k}\OperatorTok{+}\DecValTok{1}\NormalTok{]}\OperatorTok{-}\NormalTok{t[}\DecValTok{1}\NormalTok{]}\OperatorTok{+}\NormalTok{((t[k}\OperatorTok{+}\DecValTok{2}\NormalTok{]}\OperatorTok{-}\NormalTok{t[k}\OperatorTok{+}\DecValTok{1}\NormalTok{])}\OperatorTok{^}\DecValTok{2}\OperatorTok{*}\NormalTok{(t[k}\OperatorTok{+}\DecValTok{2}\NormalTok{]}\OperatorTok{-}\NormalTok{t[k}\OperatorTok{+}\DecValTok{1}\NormalTok{])}\OperatorTok{-}\NormalTok{(t[k}\OperatorTok{+}\DecValTok{2}\NormalTok{]}\OperatorTok{-}\NormalTok{t[k}\OperatorTok{+}\DecValTok{1}\NormalTok{])}\OperatorTok{^}\DecValTok{3}\OperatorTok{/}\DecValTok{3}\NormalTok{)}\OperatorTok{/}\NormalTok{(t[k}\OperatorTok{+}\DecValTok{2}\NormalTok{]}\OperatorTok{-}\NormalTok{t[k}\OperatorTok{+}\DecValTok{1}\NormalTok{])}\OperatorTok{^}\DecValTok{2}
\NormalTok{  sigma[k}\OperatorTok{+}\DecValTok{3}\NormalTok{,]=x}
  
\NormalTok{  center.vector=}\KeywordTok{apply}\NormalTok{(sigma,}\DecValTok{1}\NormalTok{,mean)}
  
  \KeywordTok{list}\NormalTok{(}\DataTypeTok{sigma=}\OperatorTok{-}\NormalTok{sigma, }\DataTypeTok{center.vector=}\OperatorTok{-}\NormalTok{center.vector)}
\NormalTok{\}}
\end{Highlighting}
\end{Shaded}

Then we calculate the basis function using the above decconvex function
specified above. We specify the CGAM basis for the age effects. We set
knots based on quantiles of the failure or censoring intervals, with a
goal of 6 knots total. We calculate the basis (delta\_i), then mean
center it (delta). Then we rescale this basis by dividing by the maximum
value of the basis (delta) to aid convergence. We set a variable with
the number of knots (nknots\_age\_cgam).

\begin{Shaded}
\begin{Highlighting}[]
\NormalTok{quant_age <-}\StringTok{ }\NormalTok{.}\DecValTok{2}
\NormalTok{knots_age <-}\StringTok{ }\KeywordTok{c}\NormalTok{(}\DecValTok{1}\NormalTok{, }\KeywordTok{round}\NormalTok{(}\KeywordTok{quantile}\NormalTok{(d_fit}\OperatorTok{$}\NormalTok{right.age }\OperatorTok{-}\StringTok{ }\DecValTok{1}\NormalTok{,}
                                \KeywordTok{c}\NormalTok{(}\KeywordTok{seq}\NormalTok{(quant_age,}
\NormalTok{                                      .}\DecValTok{99}\NormalTok{,}
                                      \DataTypeTok{by =}\NormalTok{ quant_age),}
\NormalTok{                                  .}\DecValTok{99}\NormalTok{))))}
\NormalTok{knots_age_cgam <-}\StringTok{ }\KeywordTok{unique}\NormalTok{(knots_age)}
\NormalTok{delta_i <-}\StringTok{ }\KeywordTok{decconvex}\NormalTok{(}\DecValTok{1}\OperatorTok{:}\NormalTok{nT_age, knots_age_cgam)}
\NormalTok{delta <-}\StringTok{ }\KeywordTok{t}\NormalTok{(delta_i}\OperatorTok{$}\NormalTok{sigma }\OperatorTok{-}\StringTok{ }\NormalTok{delta_i}\OperatorTok{$}\NormalTok{center.vector)}
\NormalTok{Z_age_cgam <-}\StringTok{ }\NormalTok{delta }\OperatorTok{/}\StringTok{ }\KeywordTok{max}\NormalTok{(delta)}
\NormalTok{nknots_age_cgam <-}\StringTok{ }\KeywordTok{dim}\NormalTok{(Z_age_cgam)[}\DecValTok{2}\NormalTok{]}
\end{Highlighting}
\end{Shaded}

Then we calculate the basis function for the age effects using natural
splines, using the spline package function (ns). Again, we use quantile
knots based on the mortality and censoring interval frequencies, and we
use an equal quantile interval that allows for the number of knots to be
close to 20. We calculate a constraint matrix (constr\_sumzero) so that
the age effects will sum to zero. We use QR factorization (qrc) to find
the null space of the constraint matrix (set to Z). Then we calculate
the basis function provided this constraint matrix (Z\_age\_spline). We
calculate the number of knots (nknots\_age\_spline).

\begin{Shaded}
\begin{Highlighting}[]
\NormalTok{quant_age <-}\StringTok{ }\NormalTok{.}\DecValTok{04}
\NormalTok{knots_age <-}\StringTok{ }\KeywordTok{c}\NormalTok{(}\DecValTok{1}\NormalTok{,}
               \KeywordTok{round}\NormalTok{(}\KeywordTok{quantile}\NormalTok{(d_fit}\OperatorTok{$}\NormalTok{right.age }\OperatorTok{-}\StringTok{ }\DecValTok{1}\NormalTok{,}
                              \KeywordTok{c}\NormalTok{(}\KeywordTok{seq}\NormalTok{(quant_age,}
\NormalTok{                                    .}\DecValTok{99}\NormalTok{,}
                                    \DataTypeTok{by =}\NormalTok{ quant_age),}
\NormalTok{                                .}\DecValTok{99}\NormalTok{))))}
\NormalTok{knots_age_spline <-}\StringTok{ }\KeywordTok{unique}\NormalTok{(knots_age)}

\NormalTok{splinebasis <-}\StringTok{ }\KeywordTok{ns}\NormalTok{(}\DecValTok{1}\OperatorTok{:}\NormalTok{nT_age, }\DataTypeTok{knots =}\NormalTok{ knots_age_spline)}

\NormalTok{##A constraint matrix so age.effects = 0}
\NormalTok{constr_sumzero <-}\StringTok{ }\KeywordTok{matrix}\NormalTok{(}\DecValTok{1}\NormalTok{, }\DecValTok{1}\NormalTok{, }\KeywordTok{nrow}\NormalTok{(splinebasis)) }\OperatorTok\StringTok{ }\NormalTok{splinebasis}

\NormalTok{##Get a basis for null space of constraint}
\NormalTok{qrc <-}\StringTok{ }\KeywordTok{qr}\NormalTok{(}\KeywordTok{t}\NormalTok{(constr_sumzero))}
\NormalTok{Z <-}\StringTok{ }\KeywordTok{qr.Q}\NormalTok{(qrc,}
          \DataTypeTok{complete =} \OtherTok{TRUE}\NormalTok{)[, (}\KeywordTok{nrow}\NormalTok{(constr_sumzero) }\OperatorTok{+}\StringTok{ }\DecValTok{1}\NormalTok{)}\OperatorTok{:}\KeywordTok{ncol}\NormalTok{(constr_sumzero)]}
\NormalTok{Z_age_spline <-}\StringTok{ }\NormalTok{splinebasis }\OperatorTok\StringTok{ }\NormalTok{Z}
\NormalTok{nknots_age_spline <-}\StringTok{ }\KeywordTok{dim}\NormalTok{(Z_age_spline)[}\DecValTok{2}\NormalTok{]}
\end{Highlighting}
\end{Shaded}

We plot the basis functions.

\begin{Shaded}
\begin{Highlighting}[]
\KeywordTok{pdf}\NormalTok{(}\StringTok{"figures/basis_function_age.pdf"}\NormalTok{)}
\KeywordTok{plot}\NormalTok{(}\DecValTok{1}\OperatorTok{:}\NormalTok{nT_age,}
\NormalTok{     Z_age_spline[, }\DecValTok{1}\NormalTok{],}
     \DataTypeTok{ylim =} \KeywordTok{c}\NormalTok{(}\OperatorTok{-}\DecValTok{1}\NormalTok{, }\DecValTok{1}\NormalTok{),}
     \DataTypeTok{type =} \StringTok{"l"}\NormalTok{,}
     \DataTypeTok{main =} \StringTok{"Basis Function Age Effect Spline"}\NormalTok{)}
\ControlFlowTok{for}\NormalTok{ (i }\ControlFlowTok{in} \DecValTok{2}\OperatorTok{:}\NormalTok{nknots_age_spline) \{}
  \KeywordTok{lines}\NormalTok{(}\DecValTok{1}\OperatorTok{:}\NormalTok{nT_age, Z_age_spline[, i])}
\NormalTok{\}}
\KeywordTok{plot}\NormalTok{(}\DecValTok{1}\OperatorTok{:}\NormalTok{nT_age,}
\NormalTok{     Z_age_cgam[, }\DecValTok{1}\NormalTok{],}
     \DataTypeTok{ylim =} \KeywordTok{c}\NormalTok{(}\OperatorTok{-}\DecValTok{1}\NormalTok{, }\DecValTok{1}\NormalTok{),}
     \DataTypeTok{type =} \StringTok{"l"}\NormalTok{,}
     \DataTypeTok{main =} \StringTok{"Basis Function Age Effect CGAM"}\NormalTok{)}
\ControlFlowTok{for}\NormalTok{ (i }\ControlFlowTok{in} \DecValTok{2}\OperatorTok{:}\NormalTok{nknots_age_cgam) \{}
  \KeywordTok{lines}\NormalTok{(}\DecValTok{1}\OperatorTok{:}\NormalTok{nT_age, Z_age_cgam[, i])}
\NormalTok{\}}
\KeywordTok{dev.off}\NormalTok{()}
\end{Highlighting}
\end{Shaded}

Then, we calculate the distance matrix, or basis expansions, for the
kernel convolution smoother for the period effects for each year
seperately. We used the same interval for each year with a knot at each
period (day). We calculate the absolute distance between each period for
each knot.

\begin{Shaded}
\begin{Highlighting}[]
\NormalTok{######################}
\NormalTok{###}
\NormalTok{### Year 1 = 2015}
\NormalTok{###}
\NormalTok{######################}

\NormalTok{intvl_period_yr1 <-}\StringTok{ }\NormalTok{intvl_period }\CommentTok{# interval for knots}
\NormalTok{knots_period_yr1 <-}\StringTok{ }\KeywordTok{unique}\NormalTok{(}\KeywordTok{c}\NormalTok{(}\KeywordTok{seq}\NormalTok{(}\DecValTok{1}\NormalTok{,}
\NormalTok{                                 nT_yr1,}
\NormalTok{                                 intvl_period_yr1),}
\NormalTok{                             nT_yr1)) }\CommentTok{#knots}
\NormalTok{nknots_period_yr1 <-}\StringTok{ }\KeywordTok{length}\NormalTok{(knots_period_yr1)}

\NormalTok{Z_period_yr1 <-}\StringTok{ }\KeywordTok{matrix}\NormalTok{(}\DecValTok{0}\NormalTok{,}
\NormalTok{                     nT_yr1,}
\NormalTok{                     nknots_period_yr1)}
\ControlFlowTok{for}\NormalTok{ (i }\ControlFlowTok{in} \DecValTok{1}\OperatorTok{:}\KeywordTok{nrow}\NormalTok{(Z_period_yr1)) \{}
  \ControlFlowTok{for}\NormalTok{ (j }\ControlFlowTok{in} \DecValTok{1}\OperatorTok{:}\NormalTok{nknots_period_yr1) \{}
\NormalTok{    Z_period_yr1[i, j] <-}\StringTok{ }\KeywordTok{abs}\NormalTok{(i }\OperatorTok{-}\StringTok{ }\NormalTok{knots_period_yr1[j])}
\NormalTok{  \}}
\NormalTok{\}}

\NormalTok{######################}
\NormalTok{###}
\NormalTok{### Year 2 = 2016}
\NormalTok{###}
\NormalTok{######################}

\NormalTok{intvl_period_yr2 <-}\StringTok{ }\NormalTok{intvl_period }\CommentTok{# interval for knots}

\NormalTok{knots_period_yr2 <-}\StringTok{ }\KeywordTok{unique}\NormalTok{(}\KeywordTok{c}\NormalTok{(}\KeywordTok{seq}\NormalTok{(}\DecValTok{1}\NormalTok{,}
\NormalTok{                               nT_yr2,}
\NormalTok{                               intvl_period_yr2),}
\NormalTok{                            nT_yr2)) }\CommentTok{#knots}
\NormalTok{nknots_period_yr2 <-}\StringTok{ }\KeywordTok{length}\NormalTok{(knots_period_yr2)}
\NormalTok{Z_period_yr2 <-}\StringTok{ }\KeywordTok{matrix}\NormalTok{(}\DecValTok{0}\NormalTok{,}
\NormalTok{                       nT_yr2,}
\NormalTok{                       nknots_period_yr2)}
\ControlFlowTok{for}\NormalTok{ (i }\ControlFlowTok{in} \DecValTok{1}\OperatorTok{:}\KeywordTok{nrow}\NormalTok{(Z_period_yr2)) \{}
  \ControlFlowTok{for}\NormalTok{ (j }\ControlFlowTok{in} \DecValTok{1}\OperatorTok{:}\NormalTok{nknots_period_yr2) \{}
\NormalTok{    Z_period_yr2[i, j] <-}\StringTok{ }\KeywordTok{abs}\NormalTok{(i }\OperatorTok{-}\StringTok{ }\NormalTok{knots_period_yr2[j]) }\CommentTok{#absolute value}
\NormalTok{  \}}
\NormalTok{\}}
\NormalTok{######################}
\NormalTok{###}
\NormalTok{### Year 3 = 2017}
\NormalTok{###}
\NormalTok{######################}

\NormalTok{intvl_period_yr3 <-}\StringTok{ }\NormalTok{intvl_period }\CommentTok{# interval for knots}
\NormalTok{knots_period_yr3 <-}\StringTok{ }\KeywordTok{unique}\NormalTok{(}\KeywordTok{c}\NormalTok{(}\KeywordTok{seq}\NormalTok{(}\DecValTok{1}\NormalTok{,}
\NormalTok{                                 nT_yr3,}
\NormalTok{                                 intvl_period_yr3),}
\NormalTok{                             nT_yr3)) }\CommentTok{#knots}
\NormalTok{nknots_period_yr3 <-}\StringTok{ }\KeywordTok{length}\NormalTok{(knots_period_yr3)}

\NormalTok{Z_period_yr3 <-}\StringTok{ }\KeywordTok{matrix}\NormalTok{(}\DecValTok{0}\NormalTok{, nT_yr3, nknots_period_yr3)}
\ControlFlowTok{for}\NormalTok{ (i }\ControlFlowTok{in} \DecValTok{1}\OperatorTok{:}\KeywordTok{nrow}\NormalTok{(Z_period_yr3)) \{}
  \ControlFlowTok{for}\NormalTok{ (j }\ControlFlowTok{in} \DecValTok{1}\OperatorTok{:}\NormalTok{nknots_period_yr3) \{}
\NormalTok{    Z_period_yr3[i, j] <-}\StringTok{ }\KeywordTok{abs}\NormalTok{(i }\OperatorTok{-}\StringTok{ }\NormalTok{knots_period_yr3[j])}
\NormalTok{  \}}
\NormalTok{\}}
\end{Highlighting}
\end{Shaded}

We set the number of MCMC iterations (reps), the burn-in period (bin),
the number of MCMC chains (n\_chains), and the thinning interval
(n\_thin).

\begin{Shaded}
\begin{Highlighting}[]
\NormalTok{reps <-}\StringTok{ }\DecValTok{50000}
\NormalTok{bin <-}\StringTok{ }\NormalTok{reps }\OperatorTok{*}\StringTok{ }\NormalTok{.}\DecValTok{5}
\NormalTok{n_chains <-}\StringTok{ }\DecValTok{3}
\NormalTok{n_thin <-}\StringTok{ }\DecValTok{1}
\end{Highlighting}
\end{Shaded}

\subsection{Run the model using the NIMBLE
package}\label{run-the-model-using-the-nimble-package}

The following R script S5\_04\_run\_model\_CSL\_K.R is used to run the
model using the NIMBLE package. First, we provide a function that can
calculate the probability of a mortality for each individual across the
intervals (age and period) that the individual is collared and alive in
the study. The function is based on indexing over the age effects
(age\_effect), and uses the age2date vector to calculate the proper
index for the period intervals (period\_effect). The intercept (beta0)
is added to the log hazard. The summation from left to right provides
the approximation of the integration of the log hazard for obtaining the
cumulative probability of survival during each age interval and each
period interval. The number of rows in the data (records), describes
where individuals that die during the study have 2 rows and individuals
that are right censored have a single row. Additional arguments include
the age of entry (left), and age of exiting the study (right), and the
total number of age effects (nT\_age).

\begin{Shaded}
\begin{Highlighting}[]
\NormalTok{state_transition <-}\StringTok{ }\KeywordTok{nimbleFunction}\NormalTok{(}
  \DataTypeTok{run =} \ControlFlowTok{function}\NormalTok{(}\DataTypeTok{records =} \KeywordTok{double}\NormalTok{(}\DecValTok{0}\NormalTok{),}
                 \DataTypeTok{left =} \KeywordTok{double}\NormalTok{(}\DecValTok{1}\NormalTok{),}
                 \DataTypeTok{right =} \KeywordTok{double}\NormalTok{(}\DecValTok{1}\NormalTok{),}
                 \DataTypeTok{beta0 =} \KeywordTok{double}\NormalTok{(}\DecValTok{0}\NormalTok{)}
                 \DataTypeTok{age_effect =} \KeywordTok{double}\NormalTok{(}\DecValTok{1}\NormalTok{),}
                 \DataTypeTok{period_effect =} \KeywordTok{double}\NormalTok{(}\DecValTok{1}\NormalTok{),}
                 \DataTypeTok{age2date =} \KeywordTok{double}\NormalTok{(}\DecValTok{1}\NormalTok{),}
                 \DataTypeTok{nT_age =} \KeywordTok{double}\NormalTok{(}\DecValTok{0}\NormalTok{)}
\NormalTok{  )\{}

\NormalTok{    SLR <-}\StringTok{ }\KeywordTok{nimNumeric}\NormalTok{(records)}
\NormalTok{    UCH <-}\KeywordTok{nimMatrix}\NormalTok{(}\DataTypeTok{value =} \DecValTok{0}\NormalTok{,}\DataTypeTok{nrow =}\NormalTok{ records, }\DataTypeTok{ncol =}\NormalTok{ nT_age)}
    \ControlFlowTok{for}\NormalTok{ (j }\ControlFlowTok{in} \DecValTok{1}\OperatorTok{:}\NormalTok{records) \{}
      \ControlFlowTok{for}\NormalTok{ (k }\ControlFlowTok{in}\NormalTok{ left[j]}\OperatorTok{:}\NormalTok{(right[j] }\OperatorTok{-}\StringTok{ }\DecValTok{1}\NormalTok{)) \{}
\NormalTok{        UCH[j, k] <-}\StringTok{ }\KeywordTok{exp}\NormalTok{(beta0 }\OperatorTok{+}
\StringTok{                         }\NormalTok{age_effect[k] }\OperatorTok{+}
\StringTok{                         }\NormalTok{period_effect[k }\OperatorTok{-}\StringTok{ }\NormalTok{age2date[j]])}
\NormalTok{      \}}
\NormalTok{      SLR[j] <-}\StringTok{ }\KeywordTok{exp}\NormalTok{(}\OperatorTok{-}\KeywordTok{sum}\NormalTok{(UCH[j, left[j]}\OperatorTok{:}\NormalTok{(right[j] }\OperatorTok{-}\StringTok{ }\DecValTok{1}\NormalTok{)]))}
\NormalTok{    \}}
    \KeywordTok{returnType}\NormalTok{(}\KeywordTok{double}\NormalTok{(}\DecValTok{1}\NormalTok{))}
    \KeywordTok{return}\NormalTok{(SLR[}\DecValTok{1}\OperatorTok{:}\NormalTok{records])}
\NormalTok{  \})}

\NormalTok{cstate_transition <-}\StringTok{ }\KeywordTok{compileNimble}\NormalTok{(state_transition)}
\end{Highlighting}
\end{Shaded}

Then we specify the model statement using the NIMBLE syntax, which is
based on declarative BUGS language. We specify the prior distribution
for the intercept (beta0) using parameter expansion.

\begin{Shaded}
\begin{Highlighting}[]
\NormalTok{  beta0_temp }\OperatorTok{~}\StringTok{ }\KeywordTok{dnorm}\NormalTok{(}\DecValTok{0}\NormalTok{, .}\DecValTok{01}\NormalTok{)}
\NormalTok{  mix }\OperatorTok{~}\StringTok{ }\KeywordTok{dunif}\NormalTok{(}\OperatorTok{-}\DecValTok{1}\NormalTok{, }\DecValTok{1}\NormalTok{)}
\NormalTok{  beta0 <-}\StringTok{ }\NormalTok{beta0_temp }\OperatorTok{*}\StringTok{ }\NormalTok{mix}
\end{Highlighting}
\end{Shaded}

Then we specify the prior distribution for the effects for the CGAM
model, and calculate the age effects. We centered the age effects for
the cgam model by subtracting the mean of the age effects and then used
the sum-to-zero age effects. We could not use QR factorization to obtain
the null space of the basis function when calculating the basis
function, because this mapping of the basis expansion prevented the
non-negative constraint for the coefficients to ensure the convex shape.

\begin{Shaded}
\begin{Highlighting}[]
  \ControlFlowTok{for}\NormalTok{ (k }\ControlFlowTok{in} \DecValTok{1}\OperatorTok{:}\NormalTok{nknots_age_cgam) \{}
\NormalTok{    ln_b_age_cgam[k] }\OperatorTok{~}\StringTok{ }\KeywordTok{dnorm}\NormalTok{(}\DecValTok{0}\NormalTok{, tau_age_cgam)}
\NormalTok{    b_age_cgam[k] <-}\StringTok{ }\KeywordTok{exp}\NormalTok{(ln_b_age_cgam[k])}
\NormalTok{  \}}
\NormalTok{  tau_age_cgam }\OperatorTok{~}\StringTok{ }\KeywordTok{dgamma}\NormalTok{(}\DecValTok{1}\NormalTok{, }\DecValTok{1}\NormalTok{)}

  \ControlFlowTok{for}\NormalTok{ (t }\ControlFlowTok{in} \DecValTok{1}\OperatorTok{:}\NormalTok{nT_age) \{}
\NormalTok{    age_effect_temp[t] <-}\StringTok{ }\KeywordTok{inprod}\NormalTok{(b_age_cgam[}\DecValTok{1}\OperatorTok{:}\NormalTok{nknots_age_cgam],}
\NormalTok{                                 Z_age_cgam[t,}\DecValTok{1}\OperatorTok{:}\NormalTok{nknots_age_cgam])}
\NormalTok{    age_effect_cgam[t] <-}\StringTok{ }\NormalTok{age_effect_temp[t] }\OperatorTok{-}\StringTok{ }\NormalTok{mu_age_cgam}
\NormalTok{  \}}
\NormalTok{  mu_age_cgam <-}\StringTok{ }\KeywordTok{mean}\NormalTok{(age_effect_temp[}\DecValTok{1}\OperatorTok{:}\NormalTok{nT_age])}
\end{Highlighting}
\end{Shaded}

The prior for the spline model was specified using the double
exponential, or Laplace prior rather than a normal prior distribution.
The QR factorization that was implemented on the basis function ensures
these effects are summing to zero. We combined the cgam and spline
models by adding them.

\begin{Shaded}
\begin{Highlighting}[]
  \ControlFlowTok{for}\NormalTok{ (k }\ControlFlowTok{in} \DecValTok{1}\OperatorTok{:}\NormalTok{nknots_age_spline) \{}
\NormalTok{    b_age_spline[k] }\OperatorTok{~}\StringTok{ }\KeywordTok{ddexp}\NormalTok{(}\DecValTok{0}\NormalTok{, tau_age_spline)}
\NormalTok{    \}}
\NormalTok{  tau_age_spline }\OperatorTok{~}\StringTok{ }\KeywordTok{dgamma}\NormalTok{(.}\DecValTok{1}\NormalTok{, .}\DecValTok{1}\NormalTok{)}
  \ControlFlowTok{for}\NormalTok{ (t }\ControlFlowTok{in} \DecValTok{1}\OperatorTok{:}\NormalTok{nT_age) \{}
\NormalTok{    age_effect_spline[t] <-}\StringTok{ }\KeywordTok{inprod}\NormalTok{(b_age_spline[}\DecValTok{1}\OperatorTok{:}\NormalTok{nknots_age_spline],}
\NormalTok{                                   Z_age_spline[t, }\DecValTok{1}\OperatorTok{:}\NormalTok{nknots_age_spline])}
\NormalTok{  \}}
\NormalTok{  ########################}
\NormalTok{  ### Age effect combine}
\NormalTok{  ########################}
\NormalTok{  age_effect[}\DecValTok{1}\OperatorTok{:}\NormalTok{nT_age] <-}\StringTok{ }\NormalTok{age_effect_cgam[}\DecValTok{1}\OperatorTok{:}\NormalTok{nT_age] }\OperatorTok{+}
\StringTok{                          }\NormalTok{age_effect_spline[}\DecValTok{1}\OperatorTok{:}\NormalTok{nT_age]}
\end{Highlighting}
\end{Shaded}

We calculate the period effects using the kernel convolution function
provided earlier (period\_effect). We used a shared specification for
the kernel smoother for the three years. We specify the prior for the
kernel smoother on the log scale (ln\_sk\_period), along with a
parameter expansion mixing parameter to aid in mixing (mix2). We
generate the Gaussian white noise process (alphau\_period) by generating
the standard normal (alpha\_period) and then multiply this by the
standard deviation (sda\_period) that has a truncated normal prior.

\begin{Shaded}
\begin{Highlighting}[]
\NormalTok{  mix2 }\OperatorTok{~}\StringTok{ }\KeywordTok{dunif}\NormalTok{(}\OperatorTok{-}\DecValTok{1}\NormalTok{, }\DecValTok{1}\NormalTok{)}
\NormalTok{  ln_sk_period }\OperatorTok{~}\StringTok{ }\KeywordTok{dnorm}\NormalTok{(}\DecValTok{0}\NormalTok{, }\DataTypeTok{sd =} \DecValTok{1}\NormalTok{)}
\NormalTok{  sdk_period <-}\StringTok{ }\KeywordTok{exp}\NormalTok{(mix2 }\OperatorTok{*}\StringTok{ }\NormalTok{ln_sk_period)}
\NormalTok{  tauk_period <-}\StringTok{ }\DecValTok{1} \OperatorTok{/}\StringTok{ }\NormalTok{sdk_period}\OperatorTok{^}\DecValTok{2}
\NormalTok{  stauk_period <-}\StringTok{ }\KeywordTok{sqrt}\NormalTok{(tauk_period)}
\NormalTok{  sda_period }\OperatorTok{~}\StringTok{ }\KeywordTok{T}\NormalTok{(}\KeywordTok{dnorm}\NormalTok{(}\DecValTok{0}\NormalTok{, }\DataTypeTok{sd =} \DecValTok{1}\NormalTok{), }\DecValTok{0}\NormalTok{, }\OtherTok{Inf}\NormalTok{)}
\NormalTok{  taua_period <-}\StringTok{ }\DecValTok{1} \OperatorTok{/}\StringTok{ }\NormalTok{sda_period}\OperatorTok{^}\DecValTok{2}
\NormalTok{  ratioinf_period <-}\StringTok{ }\NormalTok{sdk_period }\OperatorTok{/}\StringTok{ }\NormalTok{sda_period}

\NormalTok{  ###}
\NormalTok{  ### Period effects for year 1}
\NormalTok{  ###}

  \ControlFlowTok{for}\NormalTok{ (i }\ControlFlowTok{in} \DecValTok{1}\OperatorTok{:}\NormalTok{(nknots_period_yr1)) \{}
\NormalTok{    alpha_period_yr1[i] }\OperatorTok{~}\StringTok{ }\KeywordTok{dnorm}\NormalTok{(}\DecValTok{0}\NormalTok{, }\DecValTok{1}\NormalTok{)}
\NormalTok{    alphau_period_yr1[i] <-}\StringTok{ }\NormalTok{sda_period }\OperatorTok{*}\StringTok{ }\NormalTok{alpha_period_yr1[i]}
\NormalTok{  \}}
\NormalTok{  period_effect_yr1[}\DecValTok{1}\OperatorTok{:}\NormalTok{nT_yr1] <-}\StringTok{ }\KeywordTok{kernel_conv}\NormalTok{(}
    \DataTypeTok{nT =}\NormalTok{ nT_yr1,}
    \DataTypeTok{Z =}\NormalTok{ Z_period_yr1[}\DecValTok{1}\OperatorTok{:}\NormalTok{nT_yr1, }\DecValTok{1}\OperatorTok{:}\NormalTok{nknots_period_yr1],}
    \DataTypeTok{stauk =}\NormalTok{ stauk_period,}
    \DataTypeTok{nconst =}\NormalTok{ nconst,}
    \DataTypeTok{tauk =}\NormalTok{ tauk_period,}
    \DataTypeTok{nknots =}\NormalTok{ nknots_period_yr1,}
    \DataTypeTok{alphau =}\NormalTok{ alphau_period_yr1[}\DecValTok{1}\OperatorTok{:}\NormalTok{nknots_period_yr1]}
\NormalTok{  )}

\NormalTok{  ###}
\NormalTok{  ### Period effects for year 2}
\NormalTok{  ###}

  \ControlFlowTok{for}\NormalTok{ (i }\ControlFlowTok{in} \DecValTok{1}\OperatorTok{:}\NormalTok{(nknots_period_yr2)) \{}
\NormalTok{    alpha_period_yr2[i] }\OperatorTok{~}\StringTok{ }\KeywordTok{dnorm}\NormalTok{(}\DecValTok{0}\NormalTok{, }\DecValTok{1}\NormalTok{)}
\NormalTok{    alphau_period_yr2[i] <-}\StringTok{ }\NormalTok{sda_period }\OperatorTok{*}\StringTok{ }\NormalTok{alpha_period_yr2[i]}
\NormalTok{  \}}

\NormalTok{  period_effect_yr2[}\DecValTok{1}\OperatorTok{:}\NormalTok{nT_yr2] <-}\StringTok{ }\KeywordTok{kernel_conv}\NormalTok{(}
    \DataTypeTok{nT =}\NormalTok{ nT_yr2,}
    \DataTypeTok{Z =}\NormalTok{ Z_period_yr2[}\DecValTok{1}\OperatorTok{:}\NormalTok{nT_yr2, }\DecValTok{1}\OperatorTok{:}\NormalTok{nknots_period_yr2],}
    \DataTypeTok{stauk =}\NormalTok{ stauk_period,}
    \DataTypeTok{nconst =}\NormalTok{ nconst,}
    \DataTypeTok{tauk =}\NormalTok{ tauk_period,}
    \DataTypeTok{nknots =}\NormalTok{ nknots_period_yr2,}
    \DataTypeTok{alphau =}\NormalTok{ alphau_period_yr2[}\DecValTok{1}\OperatorTok{:}\NormalTok{nknots_period_yr2]}
\NormalTok{  )}

\NormalTok{  ###}
\NormalTok{  ### Period effects for year 3}
\NormalTok{  ###}

  \ControlFlowTok{for}\NormalTok{ (i }\ControlFlowTok{in} \DecValTok{1}\OperatorTok{:}\NormalTok{(nknots_period_yr3)) \{}
\NormalTok{    alpha_period_yr3[i] }\OperatorTok{~}\StringTok{ }\KeywordTok{dnorm}\NormalTok{(}\DecValTok{0}\NormalTok{, }\DecValTok{1}\NormalTok{)}
\NormalTok{    alphau_period_yr3[i] <-}\StringTok{ }\NormalTok{sda_period }\OperatorTok{*}\StringTok{ }\NormalTok{alpha_period_yr3[i]}
\NormalTok{  \}}

\NormalTok{  period_effect_yr3[}\DecValTok{1}\OperatorTok{:}\NormalTok{nT_yr3] <-}\StringTok{ }\KeywordTok{kernel_conv}\NormalTok{(}
    \DataTypeTok{nT =}\NormalTok{ nT_yr3,}
    \DataTypeTok{Z =}\NormalTok{ Z_period_yr3[}\DecValTok{1}\OperatorTok{:}\NormalTok{nT_yr3, }\DecValTok{1}\OperatorTok{:}\NormalTok{nknots_period_yr3],}
    \DataTypeTok{stauk =}\NormalTok{ stauk_period,}
    \DataTypeTok{nconst =}\NormalTok{ nconst,}
    \DataTypeTok{tauk =}\NormalTok{ tauk_period,}
    \DataTypeTok{nknots =}\NormalTok{ nknots_period_yr3,}
    \DataTypeTok{alphau =}\NormalTok{ alphau_period_yr3[}\DecValTok{1}\OperatorTok{:}\NormalTok{nknots_period_yr3]}
\NormalTok{  )}
  
\NormalTok{  ###}
\NormalTok{  ### Combine period effects into a single vector}
\NormalTok{  ###}

\NormalTok{  period_effect[}\DecValTok{1}\OperatorTok{:}\NormalTok{nT_yr1] <-}\StringTok{ }\NormalTok{period_effect_yr1[}\DecValTok{1}\OperatorTok{:}\NormalTok{nT_yr1]}
\NormalTok{  period_effect[(nT_yr1 }\OperatorTok{+}\StringTok{ }\DecValTok{1}\NormalTok{)}\OperatorTok{:}\NormalTok{(nT_yr1 }\OperatorTok{+}\StringTok{ }\NormalTok{nT_yr2)] <-}\StringTok{ }\NormalTok{period_effect_yr2[}\DecValTok{1}\OperatorTok{:}\NormalTok{nT_yr2]}
\NormalTok{  period_effect[(nT_yr1 }\OperatorTok{+}\StringTok{ }\NormalTok{nT_yr2 }\OperatorTok{+}\StringTok{ }\DecValTok{1}\NormalTok{)}\OperatorTok{:}\NormalTok{(nT_period_total)] <-}\StringTok{ }\NormalTok{period_effect_yr3[}\DecValTok{1}\OperatorTok{:}\NormalTok{nT_yr3]}
\end{Highlighting}
\end{Shaded}

Then we compute the cumulative probability of mortality within each
period using the state\_transition function. The likelihood is a
Bernoulli distribution for each individual given whether they are alive
and in the study (censor=1), or whether they have died (censor = 0).

\begin{Shaded}
\begin{Highlighting}[]
\NormalTok{  SLR[}\DecValTok{1}\OperatorTok{:}\NormalTok{records] <-}\StringTok{ }\KeywordTok{state_transition}\NormalTok{(}\DataTypeTok{records =}\NormalTok{ records,}
                                   \DataTypeTok{left =}\NormalTok{ left_age[}\DecValTok{1}\OperatorTok{:}\NormalTok{records],}
                                   \DataTypeTok{right =}\NormalTok{ right_age[}\DecValTok{1}\OperatorTok{:}\NormalTok{records],}
                                   \DataTypeTok{nT_age =}\NormalTok{ nT_age,}
                                   \DataTypeTok{age_effect =}\NormalTok{ age_effect[}\DecValTok{1}\OperatorTok{:}\NormalTok{nT_age],}
                                   \DataTypeTok{period_effect =}\NormalTok{ period_effect[}\DecValTok{1}\OperatorTok{:}\NormalTok{nT_period_total],}
                                   \DataTypeTok{age2date =}\NormalTok{ age2date[}\DecValTok{1}\OperatorTok{:}\NormalTok{records],}
                                   \DataTypeTok{beta0 =}\NormalTok{ beta0)}

\NormalTok{  ####################}
\NormalTok{  ### Likelihood}
\NormalTok{  ####################}
  \ControlFlowTok{for}\NormalTok{ (j }\ControlFlowTok{in} \DecValTok{1}\OperatorTok{:}\NormalTok{records) \{}
\NormalTok{    censor[j] }\OperatorTok{~}\StringTok{ }\KeywordTok{dbern}\NormalTok{(SLR[j])}
\NormalTok{  \}}
\end{Highlighting}
\end{Shaded}

We calculate derived parameters consisting of the cumulative probability
of survival over the age intervals and period intervals separately. We
also calculated combined age+period survival by adding the age effects
with the period effects over specific intervals. A full survival surface
could also be calculated.

\begin{Shaded}
\begin{Highlighting}[]
\NormalTok{  ##########################}
\NormalTok{  ### Derived parameters}
\NormalTok{  ##########################}

  \ControlFlowTok{for}\NormalTok{ (t }\ControlFlowTok{in} \DecValTok{1}\OperatorTok{:}\NormalTok{nT_age) \{}
\NormalTok{    llambda_age[t] <-}\StringTok{ }\NormalTok{beta0 }\OperatorTok{+}\StringTok{ }\NormalTok{age_effect[t]}
\NormalTok{    UCH0_age[t] <-}\StringTok{ }\KeywordTok{exp}\NormalTok{(llambda_age[t])}
\NormalTok{    S0_age[t] <-}\StringTok{ }\KeywordTok{exp}\NormalTok{(}\OperatorTok{-}\KeywordTok{sum}\NormalTok{(UCH0_age[}\DecValTok{1}\OperatorTok{:}\NormalTok{t]))}
\NormalTok{  \}}

  \ControlFlowTok{for}\NormalTok{ (t }\ControlFlowTok{in} \DecValTok{1}\OperatorTok{:}\NormalTok{nT_yr1) \{}
\NormalTok{    llambda_period[t, }\DecValTok{1}\NormalTok{] <-}\StringTok{ }\NormalTok{beta0 }\OperatorTok{+}\StringTok{ }\NormalTok{period_effect_yr1[t]}
\NormalTok{    UCH0_period[t, }\DecValTok{1}\NormalTok{] <-}\StringTok{ }\KeywordTok{exp}\NormalTok{(llambda_period[t, }\DecValTok{1}\NormalTok{])}
\NormalTok{    S0_period[t, }\DecValTok{1}\NormalTok{] <-}\StringTok{ }\KeywordTok{exp}\NormalTok{(}\OperatorTok{-}\KeywordTok{sum}\NormalTok{(UCH0_period[}\DecValTok{1}\OperatorTok{:}\NormalTok{t, }\DecValTok{1}\NormalTok{]))}
\NormalTok{  \}}
  \ControlFlowTok{for}\NormalTok{ (t }\ControlFlowTok{in} \DecValTok{1}\OperatorTok{:}\NormalTok{(nT_yr2)) \{}
\NormalTok{    llambda_period[t, }\DecValTok{2}\NormalTok{] <-}\StringTok{ }\NormalTok{beta0}\OperatorTok{+}\NormalTok{period_effect_yr2[t]}
\NormalTok{    UCH0_period[t, }\DecValTok{2}\NormalTok{] <-}\StringTok{ }\KeywordTok{exp}\NormalTok{(llambda_period[t, }\DecValTok{2}\NormalTok{])}
\NormalTok{    S0_period[t, }\DecValTok{2}\NormalTok{] <-}\StringTok{ }\KeywordTok{exp}\NormalTok{(}\OperatorTok{-}\KeywordTok{sum}\NormalTok{(UCH0_period[}\DecValTok{1}\OperatorTok{:}\NormalTok{t, }\DecValTok{2}\NormalTok{]))}
\NormalTok{  \}}
  \ControlFlowTok{for}\NormalTok{ (t }\ControlFlowTok{in} \DecValTok{1}\OperatorTok{:}\NormalTok{(nT_yr3)) \{}
\NormalTok{    llambda_period[t, }\DecValTok{3}\NormalTok{] <-}\StringTok{ }\NormalTok{beta0 }\OperatorTok{+}\StringTok{ }\NormalTok{period_effect_yr3[t]}
\NormalTok{    UCH0_period[t, }\DecValTok{3}\NormalTok{] <-}\StringTok{ }\KeywordTok{exp}\NormalTok{(llambda_period[t, }\DecValTok{3}\NormalTok{])}
\NormalTok{    S0_period[t, }\DecValTok{3}\NormalTok{] <-}\StringTok{ }\KeywordTok{exp}\NormalTok{(}\OperatorTok{-}\KeywordTok{sum}\NormalTok{(UCH0_period[}\DecValTok{1}\OperatorTok{:}\NormalTok{t, }\DecValTok{3}\NormalTok{]))}
\NormalTok{  \}}
  \CommentTok{#combining age+period}
  \ControlFlowTok{for}\NormalTok{ (t }\ControlFlowTok{in} \DecValTok{1}\OperatorTok{:}\NormalTok{nT_age) \{}\CommentTok{#nT_age<nT_yr1}
\NormalTok{    llambda_ageperiod[t, }\DecValTok{1}\NormalTok{] <-}\StringTok{ }\NormalTok{beta0 }\OperatorTok{+}\StringTok{ }\NormalTok{period_effect_yr1[t] }\OperatorTok{+}\StringTok{ }\NormalTok{age_effect[t]}
\NormalTok{    UCH0_ageperiod[t, }\DecValTok{1}\NormalTok{] <-}\StringTok{ }\KeywordTok{exp}\NormalTok{(llambda_ageperiod[t, }\DecValTok{1}\NormalTok{])}
\NormalTok{    S0_ageperiod[t, }\DecValTok{1}\NormalTok{] <-}\StringTok{ }\KeywordTok{exp}\NormalTok{(}\OperatorTok{-}\KeywordTok{sum}\NormalTok{(UCH0_ageperiod[}\DecValTok{1}\OperatorTok{:}\NormalTok{t, }\DecValTok{1}\NormalTok{]))}
\NormalTok{  \}}
  \ControlFlowTok{for}\NormalTok{ (t }\ControlFlowTok{in} \DecValTok{1}\OperatorTok{:}\NormalTok{(nT_yr2)) \{}
\NormalTok{    llambda_ageperiod[t, }\DecValTok{2}\NormalTok{] <-}\StringTok{ }\NormalTok{beta0 }\OperatorTok{+}\StringTok{ }\NormalTok{period_effect_yr2[t] }\OperatorTok{+}\StringTok{ }\NormalTok{age_effect[t]}
\NormalTok{    UCH0_ageperiod[t, }\DecValTok{2}\NormalTok{] <-}\StringTok{ }\KeywordTok{exp}\NormalTok{(llambda_ageperiod[t, }\DecValTok{2}\NormalTok{])}
\NormalTok{    S0_ageperiod[t, }\DecValTok{2}\NormalTok{] <-}\StringTok{ }\KeywordTok{exp}\NormalTok{(}\OperatorTok{-}\KeywordTok{sum}\NormalTok{(UCH0_ageperiod[}\DecValTok{1}\OperatorTok{:}\NormalTok{t, }\DecValTok{2}\NormalTok{]))}
\NormalTok{  \}}
  \ControlFlowTok{for}\NormalTok{ (t }\ControlFlowTok{in} \DecValTok{1}\OperatorTok{:}\NormalTok{(nT_yr3)) \{}
\NormalTok{    llambda_ageperiod[t, }\DecValTok{3}\NormalTok{] <-}\StringTok{ }\NormalTok{beta0 }\OperatorTok{+}\StringTok{ }\NormalTok{period_effect_yr3[t] }\OperatorTok{+}\StringTok{ }\NormalTok{age_effect[t]}
\NormalTok{    UCH0_ageperiod[t, }\DecValTok{3}\NormalTok{] <-}\StringTok{ }\KeywordTok{exp}\NormalTok{(llambda_ageperiod[t, }\DecValTok{3}\NormalTok{])}
\NormalTok{    S0_ageperiod[t, }\DecValTok{3}\NormalTok{] <-}\StringTok{ }\KeywordTok{exp}\NormalTok{(}\OperatorTok{-}\KeywordTok{sum}\NormalTok{(UCH0_ageperiod[}\DecValTok{1}\OperatorTok{:}\NormalTok{t, }\DecValTok{3}\NormalTok{]))}
\NormalTok{  \}}
\end{Highlighting}
\end{Shaded}

We put this altogether in the modelcode statement.

\begin{Shaded}
\begin{Highlighting}[]
\NormalTok{modelcode <-}\StringTok{ }\KeywordTok{nimbleCode}\NormalTok{(\{}

  \CommentTok{#Priors for Age and Period effects}
\NormalTok{  beta0_temp }\OperatorTok{~}\StringTok{ }\KeywordTok{dnorm}\NormalTok{(}\DecValTok{0}\NormalTok{, .}\DecValTok{01}\NormalTok{)}
\NormalTok{  mix }\OperatorTok{~}\StringTok{ }\KeywordTok{dunif}\NormalTok{(}\OperatorTok{-}\DecValTok{1}\NormalTok{, }\DecValTok{1}\NormalTok{)}
\NormalTok{  beta0 <-}\StringTok{ }\NormalTok{beta0_temp }\OperatorTok{*}\StringTok{ }\NormalTok{mix}

\NormalTok{  #####################}
\NormalTok{  ### Age effect cgam}
\NormalTok{  #####################}

  \ControlFlowTok{for}\NormalTok{ (k }\ControlFlowTok{in} \DecValTok{1}\OperatorTok{:}\NormalTok{nknots_age_cgam) \{}
\NormalTok{    ln_b_age_cgam[k] }\OperatorTok{~}\StringTok{ }\KeywordTok{dnorm}\NormalTok{(}\DecValTok{0}\NormalTok{, tau_age_cgam)}
\NormalTok{    b_age_cgam[k] <-}\StringTok{ }\KeywordTok{exp}\NormalTok{(ln_b_age_cgam[k])}
\NormalTok{  \}}
\NormalTok{  tau_age_cgam }\OperatorTok{~}\StringTok{ }\KeywordTok{dgamma}\NormalTok{(}\DecValTok{1}\NormalTok{, }\DecValTok{1}\NormalTok{)}

  \ControlFlowTok{for}\NormalTok{ (t }\ControlFlowTok{in} \DecValTok{1}\OperatorTok{:}\NormalTok{nT_age) \{}
\NormalTok{    age_effect_temp[t] <-}\StringTok{ }\KeywordTok{inprod}\NormalTok{(b_age_cgam[}\DecValTok{1}\OperatorTok{:}\NormalTok{nknots_age_cgam],}
\NormalTok{                                 Z_age_cgam[t,}\DecValTok{1}\OperatorTok{:}\NormalTok{nknots_age_cgam])}
\NormalTok{    age_effect_cgam[t] <-}\StringTok{ }\NormalTok{age_effect_temp[t] }\OperatorTok{-}\StringTok{ }\NormalTok{mu_age_cgam}
\NormalTok{  \}}
\NormalTok{  mu_age_cgam <-}\StringTok{ }\KeywordTok{mean}\NormalTok{(age_effect_temp[}\DecValTok{1}\OperatorTok{:}\NormalTok{nT_age])}

\NormalTok{  #####################}
\NormalTok{  ### Age effect spline}
\NormalTok{  #####################}

  \ControlFlowTok{for}\NormalTok{ (k }\ControlFlowTok{in} \DecValTok{1}\OperatorTok{:}\NormalTok{nknots_age_spline) \{}
\NormalTok{    b_age_spline[k] }\OperatorTok{~}\StringTok{ }\KeywordTok{ddexp}\NormalTok{(}\DecValTok{0}\NormalTok{, tau_age_spline)}
\NormalTok{    \}}
\NormalTok{  tau_age_spline }\OperatorTok{~}\StringTok{ }\KeywordTok{dgamma}\NormalTok{(.}\DecValTok{1}\NormalTok{, .}\DecValTok{1}\NormalTok{)}
  \ControlFlowTok{for}\NormalTok{ (t }\ControlFlowTok{in} \DecValTok{1}\OperatorTok{:}\NormalTok{nT_age) \{}
\NormalTok{    age_effect_spline[t] <-}\StringTok{ }\KeywordTok{inprod}\NormalTok{(b_age_spline[}\DecValTok{1}\OperatorTok{:}\NormalTok{nknots_age_spline],}
\NormalTok{                                   Z_age_spline[t, }\DecValTok{1}\OperatorTok{:}\NormalTok{nknots_age_spline])}
\NormalTok{  \}}

\NormalTok{  ########################}
\NormalTok{  ### Age effect combine}
\NormalTok{  ########################}
\NormalTok{  age_effect[}\DecValTok{1}\OperatorTok{:}\NormalTok{nT_age] <-}\StringTok{ }\NormalTok{age_effect_cgam[}\DecValTok{1}\OperatorTok{:}\NormalTok{nT_age] }\OperatorTok{+}
\StringTok{                          }\NormalTok{age_effect_spline[}\DecValTok{1}\OperatorTok{:}\NormalTok{nT_age]}

\NormalTok{  ##################################################}
\NormalTok{  ### Period effects}
\NormalTok{  ### for entire study (3 years)}
\NormalTok{  ### with separate period effects for each year,}
\NormalTok{  ### but shared hyperparams}
\NormalTok{  ##################################################}

\NormalTok{  mix2 }\OperatorTok{~}\StringTok{ }\KeywordTok{dunif}\NormalTok{(}\OperatorTok{-}\DecValTok{1}\NormalTok{, }\DecValTok{1}\NormalTok{)}
\NormalTok{  ln_sk_period }\OperatorTok{~}\StringTok{ }\KeywordTok{dnorm}\NormalTok{(}\DecValTok{0}\NormalTok{, }\DataTypeTok{sd =} \DecValTok{1}\NormalTok{)}
\NormalTok{  sdk_period <-}\StringTok{ }\KeywordTok{exp}\NormalTok{(mix2 }\OperatorTok{*}\StringTok{ }\NormalTok{ln_sk_period)}
\NormalTok{  tauk_period <-}\StringTok{ }\DecValTok{1} \OperatorTok{/}\StringTok{ }\NormalTok{sdk_period}\OperatorTok{^}\DecValTok{2}
\NormalTok{  stauk_period <-}\StringTok{ }\KeywordTok{sqrt}\NormalTok{(tauk_period)}
\NormalTok{  sda_period }\OperatorTok{~}\StringTok{ }\KeywordTok{T}\NormalTok{(}\KeywordTok{dnorm}\NormalTok{(}\DecValTok{0}\NormalTok{, }\DataTypeTok{sd =} \DecValTok{1}\NormalTok{), }\DecValTok{0}\NormalTok{, }\OtherTok{Inf}\NormalTok{)}
\NormalTok{  taua_period <-}\StringTok{ }\DecValTok{1} \OperatorTok{/}\StringTok{ }\NormalTok{sda_period}\OperatorTok{^}\DecValTok{2}
\NormalTok{  ratioinf_period <-}\StringTok{ }\NormalTok{sdk_period }\OperatorTok{/}\StringTok{ }\NormalTok{sda_period}

\NormalTok{  ###}
\NormalTok{  ### Period effects for year 1}
\NormalTok{  ###}

  \ControlFlowTok{for}\NormalTok{ (i }\ControlFlowTok{in} \DecValTok{1}\OperatorTok{:}\NormalTok{(nknots_period_yr1)) \{}
\NormalTok{    alpha_period_yr1[i] }\OperatorTok{~}\StringTok{ }\KeywordTok{dnorm}\NormalTok{(}\DecValTok{0}\NormalTok{, }\DecValTok{1}\NormalTok{)}
\NormalTok{    alphau_period_yr1[i] <-}\StringTok{ }\NormalTok{sda_period }\OperatorTok{*}\StringTok{ }\NormalTok{alpha_period_yr1[i]}
\NormalTok{  \}}
\NormalTok{  period_effect_yr1[}\DecValTok{1}\OperatorTok{:}\NormalTok{nT_yr1] <-}\StringTok{ }\KeywordTok{kernel_conv}\NormalTok{(}
    \DataTypeTok{nT =}\NormalTok{ nT_yr1,}
    \DataTypeTok{Z =}\NormalTok{ Z_period_yr1[}\DecValTok{1}\OperatorTok{:}\NormalTok{nT_yr1, }\DecValTok{1}\OperatorTok{:}\NormalTok{nknots_period_yr1],}
    \DataTypeTok{stauk =}\NormalTok{ stauk_period,}
    \DataTypeTok{nconst =}\NormalTok{ nconst,}
    \DataTypeTok{tauk =}\NormalTok{ tauk_period,}
    \DataTypeTok{nknots =}\NormalTok{ nknots_period_yr1,}
    \DataTypeTok{alphau =}\NormalTok{ alphau_period_yr1[}\DecValTok{1}\OperatorTok{:}\NormalTok{nknots_period_yr1]}
\NormalTok{  )}

\NormalTok{  ###}
\NormalTok{  ### Period effects for year 2}
\NormalTok{  ###}

  \ControlFlowTok{for}\NormalTok{ (i }\ControlFlowTok{in} \DecValTok{1}\OperatorTok{:}\NormalTok{(nknots_period_yr2)) \{}
\NormalTok{    alpha_period_yr2[i] }\OperatorTok{~}\StringTok{ }\KeywordTok{dnorm}\NormalTok{(}\DecValTok{0}\NormalTok{, }\DecValTok{1}\NormalTok{)}
\NormalTok{    alphau_period_yr2[i] <-}\StringTok{ }\NormalTok{sda_period }\OperatorTok{*}\StringTok{ }\NormalTok{alpha_period_yr2[i]}
\NormalTok{  \}}

\NormalTok{  period_effect_yr2[}\DecValTok{1}\OperatorTok{:}\NormalTok{nT_yr2] <-}\StringTok{ }\KeywordTok{kernel_conv}\NormalTok{(}
    \DataTypeTok{nT =}\NormalTok{ nT_yr2,}
    \DataTypeTok{Z =}\NormalTok{ Z_period_yr2[}\DecValTok{1}\OperatorTok{:}\NormalTok{nT_yr2, }\DecValTok{1}\OperatorTok{:}\NormalTok{nknots_period_yr2],}
    \DataTypeTok{stauk =}\NormalTok{ stauk_period,}
    \DataTypeTok{nconst =}\NormalTok{ nconst,}
    \DataTypeTok{tauk =}\NormalTok{ tauk_period,}
    \DataTypeTok{nknots =}\NormalTok{ nknots_period_yr2,}
    \DataTypeTok{alphau =}\NormalTok{ alphau_period_yr2[}\DecValTok{1}\OperatorTok{:}\NormalTok{nknots_period_yr2]}
\NormalTok{  )}

\NormalTok{  ###}
\NormalTok{  ### Period effects for year 3}
\NormalTok{  ###}

  \ControlFlowTok{for}\NormalTok{ (i }\ControlFlowTok{in} \DecValTok{1}\OperatorTok{:}\NormalTok{(nknots_period_yr3)) \{}
\NormalTok{    alpha_period_yr3[i] }\OperatorTok{~}\StringTok{ }\KeywordTok{dnorm}\NormalTok{(}\DecValTok{0}\NormalTok{, }\DecValTok{1}\NormalTok{)}
\NormalTok{    alphau_period_yr3[i] <-}\StringTok{ }\NormalTok{sda_period }\OperatorTok{*}\StringTok{ }\NormalTok{alpha_period_yr3[i]}
\NormalTok{  \}}

\NormalTok{  period_effect_yr3[}\DecValTok{1}\OperatorTok{:}\NormalTok{nT_yr3] <-}\StringTok{ }\KeywordTok{kernel_conv}\NormalTok{(}
    \DataTypeTok{nT =}\NormalTok{ nT_yr3,}
    \DataTypeTok{Z =}\NormalTok{ Z_period_yr3[}\DecValTok{1}\OperatorTok{:}\NormalTok{nT_yr3, }\DecValTok{1}\OperatorTok{:}\NormalTok{nknots_period_yr3],}
    \DataTypeTok{stauk =}\NormalTok{ stauk_period,}
    \DataTypeTok{nconst =}\NormalTok{ nconst,}
    \DataTypeTok{tauk =}\NormalTok{ tauk_period,}
    \DataTypeTok{nknots =}\NormalTok{ nknots_period_yr3,}
    \DataTypeTok{alphau =}\NormalTok{ alphau_period_yr3[}\DecValTok{1}\OperatorTok{:}\NormalTok{nknots_period_yr3]}
\NormalTok{  )}

\NormalTok{  ###}
\NormalTok{  ### Combine period effects into a single vector}
\NormalTok{  ###}

\NormalTok{  period_effect[}\DecValTok{1}\OperatorTok{:}\NormalTok{nT_yr1] <-}\StringTok{ }\NormalTok{period_effect_yr1[}\DecValTok{1}\OperatorTok{:}\NormalTok{nT_yr1]}
\NormalTok{  period_effect[(nT_yr1 }\OperatorTok{+}\StringTok{ }\DecValTok{1}\NormalTok{)}\OperatorTok{:}\NormalTok{(nT_yr1 }\OperatorTok{+}\StringTok{ }\NormalTok{nT_yr2)] <-}\StringTok{ }\NormalTok{period_effect_yr2[}\DecValTok{1}\OperatorTok{:}\NormalTok{nT_yr2]}
\NormalTok{  period_effect[(nT_yr1 }\OperatorTok{+}\StringTok{ }\NormalTok{nT_yr2 }\OperatorTok{+}\StringTok{ }\DecValTok{1}\NormalTok{)}\OperatorTok{:}\NormalTok{(nT_period_total)] <-}\StringTok{ }\NormalTok{period_effect_yr3[}\DecValTok{1}\OperatorTok{:}\NormalTok{nT_yr3]}

\NormalTok{  ### Computing state transisiton probability}
\NormalTok{  SLR[}\DecValTok{1}\OperatorTok{:}\NormalTok{records] <-}\StringTok{ }\KeywordTok{state_transition}\NormalTok{(}\DataTypeTok{records =}\NormalTok{ records,}
                                   \DataTypeTok{left =}\NormalTok{ left_age[}\DecValTok{1}\OperatorTok{:}\NormalTok{records],}
                                   \DataTypeTok{right =}\NormalTok{ right_age[}\DecValTok{1}\OperatorTok{:}\NormalTok{records],}
                                   \DataTypeTok{nT_age =}\NormalTok{ nT_age,}
                                   \DataTypeTok{age_effect =}\NormalTok{ age_effect[}\DecValTok{1}\OperatorTok{:}\NormalTok{nT_age],}
                                   \DataTypeTok{period_effect =}\NormalTok{ period_effect[}\DecValTok{1}\OperatorTok{:}\NormalTok{nT_period_total],}
                                   \DataTypeTok{age2date =}\NormalTok{ age2date[}\DecValTok{1}\OperatorTok{:}\NormalTok{records],}
                                   \DataTypeTok{beta0 =}\NormalTok{ beta0)}
\NormalTok{  ####################}
\NormalTok{  ### Likelihood}
\NormalTok{  ####################}
  \ControlFlowTok{for}\NormalTok{ (j }\ControlFlowTok{in} \DecValTok{1}\OperatorTok{:}\NormalTok{records) \{}
\NormalTok{    censor[j] }\OperatorTok{~}\StringTok{ }\KeywordTok{dbern}\NormalTok{(SLR[j])}
\NormalTok{  \}}

\NormalTok{  ##########################}
\NormalTok{  ### Derived parameters}
\NormalTok{  ##########################}

  \ControlFlowTok{for}\NormalTok{ (t }\ControlFlowTok{in} \DecValTok{1}\OperatorTok{:}\NormalTok{nT_age) \{}
\NormalTok{    llambda_age[t] <-}\StringTok{ }\NormalTok{beta0 }\OperatorTok{+}\StringTok{ }\NormalTok{age_effect[t]}
\NormalTok{    UCH0_age[t] <-}\StringTok{ }\KeywordTok{exp}\NormalTok{(llambda_age[t])}
\NormalTok{    S0_age[t] <-}\StringTok{ }\KeywordTok{exp}\NormalTok{(}\OperatorTok{-}\KeywordTok{sum}\NormalTok{(UCH0_age[}\DecValTok{1}\OperatorTok{:}\NormalTok{t]))}
\NormalTok{  \}}

  \ControlFlowTok{for}\NormalTok{ (t }\ControlFlowTok{in} \DecValTok{1}\OperatorTok{:}\NormalTok{nT_yr1) \{}
\NormalTok{    llambda_period[t, }\DecValTok{1}\NormalTok{] <-}\StringTok{ }\NormalTok{beta0 }\OperatorTok{+}\StringTok{ }\NormalTok{period_effect_yr1[t]}
\NormalTok{    UCH0_period[t, }\DecValTok{1}\NormalTok{] <-}\StringTok{ }\KeywordTok{exp}\NormalTok{(llambda_period[t, }\DecValTok{1}\NormalTok{])}
\NormalTok{    S0_period[t, }\DecValTok{1}\NormalTok{] <-}\StringTok{ }\KeywordTok{exp}\NormalTok{(}\OperatorTok{-}\KeywordTok{sum}\NormalTok{(UCH0_period[}\DecValTok{1}\OperatorTok{:}\NormalTok{t, }\DecValTok{1}\NormalTok{]))}
\NormalTok{  \}}
  \ControlFlowTok{for}\NormalTok{ (t }\ControlFlowTok{in} \DecValTok{1}\OperatorTok{:}\NormalTok{(nT_yr2)) \{}
\NormalTok{    llambda_period[t, }\DecValTok{2}\NormalTok{] <-}\StringTok{ }\NormalTok{beta0}\OperatorTok{+}\NormalTok{period_effect_yr2[t]}
\NormalTok{    UCH0_period[t, }\DecValTok{2}\NormalTok{] <-}\StringTok{ }\KeywordTok{exp}\NormalTok{(llambda_period[t, }\DecValTok{2}\NormalTok{])}
\NormalTok{    S0_period[t, }\DecValTok{2}\NormalTok{] <-}\StringTok{ }\KeywordTok{exp}\NormalTok{(}\OperatorTok{-}\KeywordTok{sum}\NormalTok{(UCH0_period[}\DecValTok{1}\OperatorTok{:}\NormalTok{t, }\DecValTok{2}\NormalTok{]))}
\NormalTok{  \}}
  \ControlFlowTok{for}\NormalTok{ (t }\ControlFlowTok{in} \DecValTok{1}\OperatorTok{:}\NormalTok{(nT_yr3)) \{}
\NormalTok{    llambda_period[t, }\DecValTok{3}\NormalTok{] <-}\StringTok{ }\NormalTok{beta0 }\OperatorTok{+}\StringTok{ }\NormalTok{period_effect_yr3[t]}
\NormalTok{    UCH0_period[t, }\DecValTok{3}\NormalTok{] <-}\StringTok{ }\KeywordTok{exp}\NormalTok{(llambda_period[t, }\DecValTok{3}\NormalTok{])}
\NormalTok{    S0_period[t, }\DecValTok{3}\NormalTok{] <-}\StringTok{ }\KeywordTok{exp}\NormalTok{(}\OperatorTok{-}\KeywordTok{sum}\NormalTok{(UCH0_period[}\DecValTok{1}\OperatorTok{:}\NormalTok{t, }\DecValTok{3}\NormalTok{]))}
\NormalTok{  \}}
  \CommentTok{#combining age+period}
  \ControlFlowTok{for}\NormalTok{ (t }\ControlFlowTok{in} \DecValTok{1}\OperatorTok{:}\NormalTok{nT_age) \{}\CommentTok{#nT_age<nT_yr1}
\NormalTok{    llambda_ageperiod[t, }\DecValTok{1}\NormalTok{] <-}\StringTok{ }\NormalTok{beta0 }\OperatorTok{+}\StringTok{ }\NormalTok{period_effect_yr1[t] }\OperatorTok{+}\StringTok{ }\NormalTok{age_effect[t]}
\NormalTok{    UCH0_ageperiod[t, }\DecValTok{1}\NormalTok{] <-}\StringTok{ }\KeywordTok{exp}\NormalTok{(llambda_ageperiod[t, }\DecValTok{1}\NormalTok{])}
\NormalTok{    S0_ageperiod[t, }\DecValTok{1}\NormalTok{] <-}\StringTok{ }\KeywordTok{exp}\NormalTok{(}\OperatorTok{-}\KeywordTok{sum}\NormalTok{(UCH0_ageperiod[}\DecValTok{1}\OperatorTok{:}\NormalTok{t, }\DecValTok{1}\NormalTok{]))}
\NormalTok{  \}}
  \ControlFlowTok{for}\NormalTok{ (t }\ControlFlowTok{in} \DecValTok{1}\OperatorTok{:}\NormalTok{(nT_yr2)) \{}
\NormalTok{    llambda_ageperiod[t, }\DecValTok{2}\NormalTok{] <-}\StringTok{ }\NormalTok{beta0 }\OperatorTok{+}\StringTok{ }\NormalTok{period_effect_yr2[t] }\OperatorTok{+}\StringTok{ }\NormalTok{age_effect[t]}
\NormalTok{    UCH0_ageperiod[t, }\DecValTok{2}\NormalTok{] <-}\StringTok{ }\KeywordTok{exp}\NormalTok{(llambda_ageperiod[t, }\DecValTok{2}\NormalTok{])}
\NormalTok{    S0_ageperiod[t, }\DecValTok{2}\NormalTok{] <-}\StringTok{ }\KeywordTok{exp}\NormalTok{(}\OperatorTok{-}\KeywordTok{sum}\NormalTok{(UCH0_ageperiod[}\DecValTok{1}\OperatorTok{:}\NormalTok{t, }\DecValTok{2}\NormalTok{]))}
\NormalTok{  \}}
  \ControlFlowTok{for}\NormalTok{ (t }\ControlFlowTok{in} \DecValTok{1}\OperatorTok{:}\NormalTok{(nT_yr3)) \{}
\NormalTok{    llambda_ageperiod[t, }\DecValTok{3}\NormalTok{] <-}\StringTok{ }\NormalTok{beta0 }\OperatorTok{+}\StringTok{ }\NormalTok{period_effect_yr3[t] }\OperatorTok{+}\StringTok{ }\NormalTok{age_effect[t]}
\NormalTok{    UCH0_ageperiod[t, }\DecValTok{3}\NormalTok{] <-}\StringTok{ }\KeywordTok{exp}\NormalTok{(llambda_ageperiod[t, }\DecValTok{3}\NormalTok{])}
\NormalTok{    S0_ageperiod[t, }\DecValTok{3}\NormalTok{] <-}\StringTok{ }\KeywordTok{exp}\NormalTok{(}\OperatorTok{-}\KeywordTok{sum}\NormalTok{(UCH0_ageperiod[}\DecValTok{1}\OperatorTok{:}\NormalTok{t, }\DecValTok{3}\NormalTok{]))}
\NormalTok{  \}}

\NormalTok{\})}\CommentTok{#end model statement}
\end{Highlighting}
\end{Shaded}

We specify data, constants and initial values for all of the parameters
in the model.

\begin{Shaded}
\begin{Highlighting}[]
\CommentTok{#Data}
\NormalTok{nimData <-}\StringTok{ }\KeywordTok{list}\NormalTok{(}\DataTypeTok{censor =}\NormalTok{ d_fit}\OperatorTok{$}\NormalTok{censor,}
                \DataTypeTok{Z_period_yr1 =}\NormalTok{ Z_period_yr1,}
                \DataTypeTok{Z_period_yr2 =}\NormalTok{ Z_period_yr2,}
                \DataTypeTok{Z_period_yr3 =}\NormalTok{ Z_period_yr3,}
                \DataTypeTok{Z_age_cgam =}\NormalTok{ Z_age_cgam,}
                \DataTypeTok{Z_age_spline =}\NormalTok{ Z_age_spline,}
                \DataTypeTok{left_age =}\NormalTok{ d_fit}\OperatorTok{$}\NormalTok{left.age,}
                \DataTypeTok{right_age =}\NormalTok{ d_fit}\OperatorTok{$}\NormalTok{right.age,}
                \DataTypeTok{age2date =}\NormalTok{ age2date,}
                \DataTypeTok{llambda_period =} \KeywordTok{matrix}\NormalTok{(}\OtherTok{NA}\NormalTok{, }\DataTypeTok{nr =}\NormalTok{ nT_period_max, }\DataTypeTok{nc =} \DecValTok{3}\NormalTok{),}
                \DataTypeTok{UCH0_period =} \KeywordTok{matrix}\NormalTok{(}\OtherTok{NA}\NormalTok{, }\DataTypeTok{nr =}\NormalTok{ nT_period_max, }\DataTypeTok{nc =} \DecValTok{3}\NormalTok{),}
                \DataTypeTok{S0_period =} \KeywordTok{matrix}\NormalTok{(}\OtherTok{NA}\NormalTok{, }\DataTypeTok{nr =}\NormalTok{ nT_period_max, }\DataTypeTok{nc =} \DecValTok{3}\NormalTok{),}
                \DataTypeTok{llambda_ageperiod =} \KeywordTok{matrix}\NormalTok{(}\OtherTok{NA}\NormalTok{, }\DataTypeTok{nr =}\NormalTok{ nT_period_max, }\DataTypeTok{nc =} \DecValTok{3}\NormalTok{),}
                \DataTypeTok{UCH0_ageperiod =} \KeywordTok{matrix}\NormalTok{(}\OtherTok{NA}\NormalTok{, }\DataTypeTok{nr =}\NormalTok{ nT_period_max, }\DataTypeTok{nc =} \DecValTok{3}\NormalTok{),}
                \DataTypeTok{S0_ageperiod =} \KeywordTok{matrix}\NormalTok{(}\OtherTok{NA}\NormalTok{, }\DataTypeTok{nr =}\NormalTok{ nT_period_max, }\DataTypeTok{nc =} \DecValTok{3}\NormalTok{),}
                \DataTypeTok{period_effect =} \KeywordTok{rep}\NormalTok{(}\OtherTok{NA}\NormalTok{,nT_period_total)}
\NormalTok{                )}

\NormalTok{nimConsts <-}\StringTok{ }\KeywordTok{list}\NormalTok{(}\DataTypeTok{records =}\NormalTok{ n_fit,}
                 \DataTypeTok{nT_age =}\NormalTok{ nT_age,}
                 \DataTypeTok{nT_period_total =}\NormalTok{ nT_period_total,}
                 \DataTypeTok{nT_yr1 =}\NormalTok{ nT_yr1,}
                 \DataTypeTok{nT_yr2 =}\NormalTok{ nT_yr2,}
                 \DataTypeTok{nT_yr3 =}\NormalTok{ nT_yr3,}
                 \DataTypeTok{nknots_age_cgam =}\NormalTok{ nknots_age_cgam,}
                 \DataTypeTok{nknots_age_spline =}\NormalTok{ nknots_age_spline,}
                 \DataTypeTok{nknots_period_yr1 =}\NormalTok{ nknots_period_yr1,}
                 \DataTypeTok{nknots_period_yr2 =}\NormalTok{ nknots_period_yr2,}
                 \DataTypeTok{nknots_period_yr3 =}\NormalTok{ nknots_period_yr3,}
                 \DataTypeTok{nconst =} \DecValTok{1} \OperatorTok{/}\StringTok{ }\KeywordTok{sqrt}\NormalTok{(}\DecValTok{2} \OperatorTok{*}\StringTok{ }\NormalTok{pi)}
\NormalTok{                 )}

\NormalTok{initsFun <-}\StringTok{ }\ControlFlowTok{function}\NormalTok{()}\KeywordTok{list}\NormalTok{(}\DataTypeTok{beta0_temp =} \KeywordTok{rnorm}\NormalTok{(}\DecValTok{1}\NormalTok{, }\OperatorTok{-}\DecValTok{5}\NormalTok{, .}\DecValTok{0001}\NormalTok{),}
                          \DataTypeTok{mix =} \DecValTok{1}\NormalTok{,}
                          \DataTypeTok{mix2 =} \DecValTok{1}\NormalTok{,}
                          \DataTypeTok{sda_period =} \KeywordTok{runif}\NormalTok{(}\DecValTok{1}\NormalTok{, }\DecValTok{0}\NormalTok{, }\DecValTok{5}\NormalTok{),}
                          \DataTypeTok{ln_sk_period =} \KeywordTok{rnorm}\NormalTok{(}\DecValTok{1}\NormalTok{, }\DecValTok{0}\NormalTok{, }\DecValTok{1}\NormalTok{),}
                          \DataTypeTok{alpha_period_yr1 =} \KeywordTok{rep}\NormalTok{(}\DecValTok{0}\NormalTok{, nknots_period_yr1),}
                          \DataTypeTok{alpha_period_yr2 =} \KeywordTok{rep}\NormalTok{(}\DecValTok{0}\NormalTok{, nknots_period_yr2),}
                          \DataTypeTok{alpha_period_yr3 =} \KeywordTok{rep}\NormalTok{(}\DecValTok{0}\NormalTok{, nknots_period_yr3),}
                          \DataTypeTok{tau_age_cgam =} \KeywordTok{runif}\NormalTok{(}\DecValTok{1}\NormalTok{, .}\DecValTok{1}\NormalTok{, }\DecValTok{1}\NormalTok{),}
                          \DataTypeTok{tau_age_spline =} \KeywordTok{runif}\NormalTok{(}\DecValTok{1}\NormalTok{, .}\DecValTok{1}\NormalTok{, }\DecValTok{1}\NormalTok{),}
                          \DataTypeTok{ln_b_age_cgam =} \KeywordTok{runif}\NormalTok{(nknots_age_cgam, }\OperatorTok{-}\DecValTok{10}\NormalTok{, }\OperatorTok{-}\DecValTok{5}\NormalTok{),}
                          \DataTypeTok{b_age_spline =} \KeywordTok{rnorm}\NormalTok{(nknots_age_spline) }\OperatorTok{*}\StringTok{ }\DecValTok{10}\OperatorTok{^-}\DecValTok{4}\NormalTok{,}
                          \DataTypeTok{llambda_period =} \KeywordTok{matrix}\NormalTok{(}\DecValTok{0}\NormalTok{,}
                                                  \DataTypeTok{nr =}\NormalTok{ nT_period_max,}
                                                  \DataTypeTok{nc =} \DecValTok{3}\NormalTok{),}
                          \DataTypeTok{UCH0_period =} \KeywordTok{matrix}\NormalTok{(}\DecValTok{0}\NormalTok{,}
                                               \DataTypeTok{nr =}\NormalTok{ nT_period_max,}
                                               \DataTypeTok{nc =} \DecValTok{3}\NormalTok{),}
                          \DataTypeTok{S0_period =} \KeywordTok{matrix}\NormalTok{(}\DecValTok{0}\NormalTok{,}
                                             \DataTypeTok{nr =}\NormalTok{ nT_period_max,}
                                             \DataTypeTok{nc =} \DecValTok{3}\NormalTok{),}
                          \DataTypeTok{llambda_ageperiod =} \KeywordTok{matrix}\NormalTok{(}\DecValTok{0}\NormalTok{,}
                                                     \DataTypeTok{nr =}\NormalTok{ nT_period_max,}
                                                     \DataTypeTok{nc =} \DecValTok{3}\NormalTok{),}
                          \DataTypeTok{UCH0_ageperiod =} \KeywordTok{matrix}\NormalTok{(}\DecValTok{0}\NormalTok{,}
                                                  \DataTypeTok{nr =}\NormalTok{ nT_period_max,}
                                                  \DataTypeTok{nc =} \DecValTok{3}\NormalTok{),}
                          \DataTypeTok{S0_ageperiod =} \KeywordTok{matrix}\NormalTok{(}\DecValTok{0}\NormalTok{,}
                                                \DataTypeTok{nr =}\NormalTok{ nT_period_max,}
                                                \DataTypeTok{nc =} \DecValTok{3}\NormalTok{),}
                          \DataTypeTok{period_effect =} \KeywordTok{rep}\NormalTok{(}\DecValTok{0}\NormalTok{, nT_period_total)}
\NormalTok{                          )}
\NormalTok{nimInits <-}\StringTok{ }\KeywordTok{initsFun}\NormalTok{()}
\end{Highlighting}
\end{Shaded}

Then we build the model, MCMC object, and run the MCMC approximation. We
save the MCMC iterations.

\begin{Shaded}
\begin{Highlighting}[]
\NormalTok{Rmodel <-}\StringTok{ }\KeywordTok{nimbleModel}\NormalTok{(}\DataTypeTok{code =}\NormalTok{ modelcode,}
                      \DataTypeTok{constants =}\NormalTok{ nimConsts,}
                      \DataTypeTok{data =}\NormalTok{ nimData,}
                      \DataTypeTok{inits =} \KeywordTok{initsFun}\NormalTok{()}
\NormalTok{                      )}

\CommentTok{#identify params to monitor}
\NormalTok{parameters <-}\StringTok{ }\KeywordTok{c}\NormalTok{(}
              \StringTok{"beta0"}\NormalTok{,}
              \StringTok{"beta0_temp"}\NormalTok{,}
              \StringTok{"mix"}\NormalTok{,}
              \StringTok{"mix2"}\NormalTok{,}
              \StringTok{"sdk_period"}\NormalTok{,}
              \StringTok{"sda_period"}\NormalTok{,}
              \StringTok{"ratioinf_period"}\NormalTok{,}
              \StringTok{"S0_age"}\NormalTok{,}
              \StringTok{"llambda_age"}\NormalTok{,}
              \StringTok{"age_effect"}\NormalTok{,}
              \StringTok{"age_effect_cgam"}\NormalTok{,}
              \StringTok{"age_effect_spline"}\NormalTok{,}
              \StringTok{"period_effect_yr1"}\NormalTok{,}
              \StringTok{"period_effect_yr2"}\NormalTok{,}
              \StringTok{"period_effect_yr3"}\NormalTok{,}
              \StringTok{"S0_period"}\NormalTok{,}
              \StringTok{"llambda_period"}\NormalTok{,}
              \StringTok{"tau_age_cgam"}\NormalTok{,}
              \StringTok{"tau_age_spline"}\NormalTok{,}
              \StringTok{"b_age_cgam"}\NormalTok{,}
              \StringTok{"b_age_spline"}\NormalTok{,}
              \StringTok{"S0_ageperiod"}\NormalTok{,}
              \StringTok{"llambda_ageperiod"}
\NormalTok{)}
\NormalTok{starttime <-}\StringTok{ }\KeywordTok{Sys.time}\NormalTok{()}
\NormalTok{confMCMC <-}\StringTok{ }\KeywordTok{configureMCMC}\NormalTok{(Rmodel,}
                          \DataTypeTok{monitors =}\NormalTok{ parameters,}
                          \DataTypeTok{thin =}\NormalTok{ n_thin,}
                          \DataTypeTok{useConjugacy =} \OtherTok{FALSE}\NormalTok{,}
                          \DataTypeTok{enableWAIC =} \OtherTok{TRUE}\NormalTok{)}
\NormalTok{nimMCMC <-}\StringTok{ }\KeywordTok{buildMCMC}\NormalTok{(confMCMC)}
\NormalTok{Cnim <-}\StringTok{ }\KeywordTok{compileNimble}\NormalTok{(Rmodel)}
\NormalTok{CnimMCMC <-}\StringTok{ }\KeywordTok{compileNimble}\NormalTok{(nimMCMC,}
                          \DataTypeTok{project =}\NormalTok{ Rmodel)}
\NormalTok{mcmcout <-}\StringTok{ }\KeywordTok{runMCMC}\NormalTok{(CnimMCMC,}
                   \DataTypeTok{niter =}\NormalTok{ reps,}
                   \DataTypeTok{nburnin =}\NormalTok{ bin,}
                   \DataTypeTok{nchains =}\NormalTok{ n_chains,}
                   \DataTypeTok{inits =}\NormalTok{ initsFun,}
                   \DataTypeTok{samplesAsCodaMCMC =} \OtherTok{TRUE}\NormalTok{,}
                   \DataTypeTok{summary =} \OtherTok{TRUE}\NormalTok{,}
                   \DataTypeTok{WAIC =} \OtherTok{TRUE}
\NormalTok{                   )}

\NormalTok{runtime <-}\StringTok{ }\KeywordTok{difftime}\NormalTok{(}\KeywordTok{Sys.time}\NormalTok{(),}
\NormalTok{                    starttime,}
                    \DataTypeTok{units =} \StringTok{"min"}\NormalTok{)}

\KeywordTok{save}\NormalTok{(runtime, }\DataTypeTok{file =} \StringTok{"results/runtime.Rdata"}\NormalTok{)}
\KeywordTok{save}\NormalTok{(mcmcout, }\DataTypeTok{file =} \StringTok{"results/mcmcout.Rdata"}\NormalTok{)}
\end{Highlighting}
\end{Shaded}

All of the code in this section is aggregated and combined in the file
S5\_04\_run\_model\_CSL\_K.R. Code for plotting this results of the
simulation from a single seed, as well as for obtaining convergence
diagnostics are be provided in the file
S5\_05\_results\_plots\_sum\_CSL\_K.

It is not possible to extensively describe all of the models. However,
they were all fit using similar code. In the model described above,
several of the basis functions were used, which are then used in
isolation for the different age or period effects for the other model
specifications.

\end{document}
